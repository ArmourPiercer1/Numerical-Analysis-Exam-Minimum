\documentclass{book}

% 中文字体支持
\usepackage{ctex}
\usepackage{xeCJK}

% 数学公式支持
\usepackage{amsmath,amssymb,amsthm}
\usepackage{mathtools}
\usepackage{bm} % 粗体数学符号

% 图表支持
\usepackage{graphicx}
\usepackage{tikz}
\usepackage{pgfplots}
\pgfplotsset{compat=1.17}
\usepackage{float}

% 页面布局
\usepackage{geometry}
\geometry{left=2.5cm,right=2.5cm,top=3cm,bottom=3cm}
\usepackage{fancyhdr}
\usepackage{lastpage}

% 目录和超链接
\usepackage{hyperref}
\hypersetup{
    colorlinks=true,
    linkcolor=blue,
    filecolor=magenta,      
    urlcolor=cyan,
    citecolor=red
}

% 代码支持
\usepackage{listings}
\usepackage{xcolor}

% 表格支持
\usepackage{booktabs}
\usepackage{multirow}
\usepackage{array}

% 特殊形状支持
\usepackage{xcolor}

% % 副标题
% \usepackage{titling}

% 定理环境
\newtheorem{theorem}{定理}[section]
\newtheorem{lemma}[theorem]{引理}
\newtheorem{corollary}[theorem]{推论}
\newtheorem{proposition}[theorem]{命题}
\newtheorem{definition}[theorem]{定义}
\newtheorem{example}[theorem]{例题}
\newtheorem{exercise}[theorem]{练习}

% 待填写内容标记
\NewDocumentCommand{\tofill}{O{} m}{
    \textbf{\textit{\textcolor{red}{待填写:\IfValueT{#1}{(#1)}#2}}}
}

% 页眉页脚设置
\setlength{\headheight}{14.5pt}
\pagestyle{fancy}
\fancyhf{}
\fancyhead[L]{数值分析课程总结}
\fancyhead[R]{\today}
\fancyfoot[C]{\thepage/\pageref{LastPage}}

% 代码样式设置
\lstset{
    basicstyle=\ttfamily\small,
    keywordstyle=\color{blue}\bfseries,
    commentstyle=\color{green!60!black},
    stringstyle=\color{red},
    showstringspaces=false,
    numbers=left,
    numberstyle=\tiny\color{gray},
    frame=single,
    breaklines=true
}

% 文档信息
\title{Numerical Analysis \\ Exam Appendix}
\author{Astral Projection}
\date{\today}

\begin{document}

\maketitle
\tableofcontents
\newpage

\chapter{数学基础知识}

\section{补充内容}
\subsection{矩阵的可分性}


\section{重要证明}

\subsection{Legender多项式的零平方误差最小性质}
\begin{theorem}
    在所有首项为1的n次多项式中,Legendre多项式$\tilde{P}_n(x)$在$[-1,1]$上与零的平方误差最小。
\end{theorem}
\subsubsection{证明:}
提示:考虑$f=\tilde{P}_n+a_{n-1}\tilde{P}_{n-1}+...+a_1\tilde{P}_1+a_0\tilde{P}_0$和误差函数$\|f-0\|^2$


\subsection{Chebyshev多项式的零一致误差最小性质}
\begin{theorem}
    在所有首项为$2^{n-1}$的n次多项式中,Chebyshev多项式$T_n(x)$在$[-1,1]$上与零的一致误差最小。
\end{theorem}
\subsubsection{证明:}


\section{例题与习题}

\chapter{函数插值与重构}
\section{重要证明}

\subsection{插值多项式的误差函数}
\begin{theorem}
    [插值多项式误差公式]
    设$f\in C^{n+1}[a,b]$,$x_0,x_1,\ldots,x_n$为区间$[a,b]$上的$n+1$个两两不同的节点,则对任意$x\in[a,b]$,存在$\xi\in(a,b)$,使得
    \begin{align}
        f(x) - p(x) = \frac{f^{(n+1)}(\xi)}{(n+1)!} \prod_{i=0}^{n} (x - x_i)
    \end{align}
    其中$p(x)$为通过节点$(x_i,f(x_i))$的插值多项式。
\end{theorem}

\subsubsection{证明:}


\subsection{分片线性插值和分片三次Hermite插值的收敛性定理}
命题:
定义
\begin{align}
    h=\max_{1\leq i \leq n}(x_i - x_{i-1})
\end{align}
则进行分片线性插值时,
\begin{itemize}
    \item 若$f\in C[a,b]$,则$\lim_{h\rightarrow 0}\|f-\phi\|_\infty\rightarrow 0$
    \item 若$f\in C^1[a,b]$,则$\|f-\phi\|_\infty \leq \frac{h}{2}\|f'\|_\infty$
    \item 若$f\in C^2[a,b]$,则$\|f-\phi\|_\infty \leq \frac{h^2}{8}\|f''\|_\infty$
\end{itemize}
进行分片三次Hermite插值时,
\begin{itemize}
    \item 若$f\in C^1[a,b]$,则$\|f-\phi\|_\infty \leq ch\|f'\|_\infty$
    \item 若$f\in C^2[a,b]$,则$\|f-\phi\|_\infty \leq ch^2\|f''\|_\infty$
    \item 若$f\in C^3[a,b]$,则$\|f-\phi\|_\infty \leq ch^3\|f'''\|_\infty$
    \item 若$f\in C^4[a,b]$,则$\|f-\phi\|_\infty \leq \frac{1}{384}h^4\|f^{(4)}\|_\infty$
\end{itemize}

\subsubsection{证明:}
Note:分片线性插值只要做泰勒展开即可;
分片三次Hermite插值在4阶光滑性时可以通过余项公式给出,低于4阶光滑性需要用到Peano核定理。



\subsection{三角插值多项式形式、相多项式形式}

\subsection{三角插值的误差分析}


\section{函数逼近}

\section{补充内容}

\subsection{Schauder基}



\section{重要证明}

\subsection{最佳平方逼近存在唯一性定理}
\begin{theorem}
    设$f\in C[a,b]$,$\Phi$为$C[a,b]$的有限维子空间,则存在唯一的$\phi^*\in \Phi$,使得
    \begin{align}
        \|f-\phi^*\| = \min_{\phi\in \Phi} \|f-\phi\|
    \end{align}
\end{theorem}
\subsubsection{证明:}

\subsection{广义傅里叶展开收敛性的证明}
1、广义傅里叶级数收敛到$(C[a,b],\|\cdot\|_2)$的完备化空间$\bar{(C[a,b],\|\cdot\|_2)}$中的一个元素。
若$\|\cdot\|_2$为权系数$\rho$的内积诱导范数,则$\bar{(C[a,b],\|\cdot\|_2)}$即为$L^2_\rho(a,b)$。

2、设$f\in C[a,b]$,$\{\phi_m\}_{m=0}^\infty$为$C[a,b]$中的规范正交函数组,则$f$在$\{\phi_m\}$下的广义傅里叶展开
    \begin{align}
        f \sim \sum_{m=0}^{\infty} a_m \phi_m(x), \quad a_m = (f,\phi_m)
    \end{align}
    在$C[a,b]$中一致收敛于$f$,即
    \begin{align}
        \lim_{n\rightarrow \infty} \left\| f - \sum_{m=0}^{n} a_m \phi_m(x) \right\|_\infty = 0
    \end{align}

    \subsubsection{证明:}
    提示:1的证明用到$\sum_{i=0}^{\infty} |a_i|^2 < \infty$,2的证明用到Weierstrass逼近定理。




\subsection{最佳一致逼近存在性定理}

\begin{theorem}
    [最佳一致逼近存在性定理]
    设$f\in C[a,b]$,则在$n$次多项式空间$P_n$中,存在一个多项式$p_n^*\in P_n$,使得
    \begin{align}
        \|f - p_n^*\|_\infty = \min_{p_n\in P_n} \|f - p_n\|_\infty
    \end{align}
    即$P_n$中关于$f\in C[a,b]$的最小偏差是可以达到的。
\end{theorem}

\subsubsection{证明:}
提示:利用偏差泛函的性质

\section{重要例题与习题}
\subsubsection{求解最佳平方逼近}
记$\Phi=\mathrm{span}(1,x^2)$,求$x\in[0,1]$上在$\Phi$中的最佳平方逼近。







\end{document}