\documentclass{book}

% 中文字体支持
\usepackage{ctex}
\usepackage{xeCJK}

% 数学公式支持
\usepackage{amsmath,amssymb,amsthm}
\usepackage{mathtools}
\usepackage{bm} % 粗体数学符号

% 图表支持
\usepackage{graphicx}
\usepackage{tikz}
\usepackage{pgfplots}
\pgfplotsset{compat=1.17}
\usepackage{float}

% 页面布局
\usepackage{geometry}
\geometry{left=2.5cm,right=2.5cm,top=3cm,bottom=3cm}
\usepackage{fancyhdr}
\usepackage{lastpage}

% 目录和超链接
\usepackage{hyperref}
\hypersetup{
    colorlinks=true,
    linkcolor=blue,
    filecolor=magenta,      
    urlcolor=cyan,
    citecolor=red
}

% 代码支持
\usepackage{listings}
\usepackage{xcolor}

% 表格支持
\usepackage{booktabs}
\usepackage{multirow}
\usepackage{array}

% 特殊形状支持
\usepackage{xcolor}

% % 副标题
% \usepackage{titling}

% 定理环境
\newtheorem{theorem}{定理}[section]
\newtheorem{lemma}[theorem]{引理}
\newtheorem{corollary}[theorem]{推论}
\newtheorem{proposition}[theorem]{命题}
\newtheorem{definition}[theorem]{定义}
\newtheorem{example}[theorem]{例题}
\newtheorem{exercise}[theorem]{练习}

% 待填写内容标记
\NewDocumentCommand{\tofill}{O{} m}{
    \textbf{\textit{\textcolor{red}{待填写:\IfValueT{#1}{(#1)}#2}}}
}

% 页眉页脚设置
\setlength{\headheight}{14.5pt}
\pagestyle{fancy}
\fancyhf{}
\fancyhead[L]{数值分析课程总结}
\fancyhead[R]{\today}
\fancyfoot[C]{\thepage/\pageref{LastPage}}

% 代码样式设置
\lstset{
    basicstyle=\ttfamily\small,
    keywordstyle=\color{blue}\bfseries,
    commentstyle=\color{green!60!black},
    stringstyle=\color{red},
    showstringspaces=false,
    numbers=left,
    numberstyle=\tiny\color{gray},
    frame=single,
    breaklines=true
}

% 文档信息
\title{Numerical Analysis \\ Exam Appendix}
\author{Astral Projection}
\date{\today}

\begin{document}

\maketitle
\tableofcontents
\newpage

\chapter{数学基础知识}

\section{补充内容}
\subsection{矩阵的可分性}


\section{重要证明}

\subsection{Legender多项式的零平方误差最小性质}
\begin{theorem}
    在所有首项为1的n次多项式中,Legendre多项式$\tilde{P}_n(x)$在$[-1,1]$上与零的平方误差最小。
\end{theorem}
\subsubsection{证明:}
提示:考虑$f=\tilde{P}_n+a_{n-1}\tilde{P}_{n-1}+...+a_1\tilde{P}_1+a_0\tilde{P}_0$和误差函数$\|f-0\|^2$


\subsection{Chebyshev多项式的零一致误差最小性质}
\begin{theorem}
    在所有首项为$2^{n-1}$的n次多项式中,Chebyshev多项式$T_n(x)$在$[-1,1]$上与零的一致误差最小。
\end{theorem}
\subsubsection{证明:}


\section{例题与习题}

\chapter{函数插值与重构}
\section{重要证明}

\subsection{插值多项式的误差函数}
\begin{theorem}
    [插值多项式误差公式]
    设$f\in C^{n+1}[a,b]$,$x_0,x_1,\ldots,x_n$为区间$[a,b]$上的$n+1$个两两不同的节点,则对任意$x\in[a,b]$,存在$\xi\in(a,b)$,使得
    \begin{align}
        f(x) - p(x) = \frac{f^{(n+1)}(\xi)}{(n+1)!} \prod_{i=0}^{n} (x - x_i)
    \end{align}
    其中$p(x)$为通过节点$(x_i,f(x_i))$的插值多项式。
\end{theorem}

\subsubsection{证明:}


\subsection{分片线性插值和分片三次Hermite插值的收敛性定理}
命题:
定义
\begin{align}
    h=\max_{1\leq i \leq n}(x_i - x_{i-1})
\end{align}
则进行分片线性插值时,
\begin{itemize}
    \item 若$f\in C[a,b]$,则$\lim_{h\rightarrow 0}\|f-\phi\|_\infty\rightarrow 0$
    \item 若$f\in C^1[a,b]$,则$\|f-\phi\|_\infty \leq \frac{h}{2}\|f'\|_\infty$
    \item 若$f\in C^2[a,b]$,则$\|f-\phi\|_\infty \leq \frac{h^2}{8}\|f''\|_\infty$
\end{itemize}
进行分片三次Hermite插值时,
\begin{itemize}
    \item 若$f\in C^1[a,b]$,则$\|f-\phi\|_\infty \leq ch\|f'\|_\infty$
    \item 若$f\in C^2[a,b]$,则$\|f-\phi\|_\infty \leq ch^2\|f''\|_\infty$
    \item 若$f\in C^3[a,b]$,则$\|f-\phi\|_\infty \leq ch^3\|f'''\|_\infty$
    \item 若$f\in C^4[a,b]$,则$\|f-\phi\|_\infty \leq \frac{1}{384}h^4\|f^{(4)}\|_\infty$
\end{itemize}

\subsubsection{证明:}
Note:分片线性插值只要做泰勒展开即可;
分片三次Hermite插值在4阶光滑性时可以通过余项公式给出,低于4阶光滑性需要用到Peano核定理。



\subsection{三角插值多项式形式、相多项式形式}

\subsection{三角插值的误差分析}


\section{函数逼近}

\section{补充内容}

\subsection{Schauder基}



\section{重要证明}

\subsection{最佳平方逼近存在唯一性定理}
\begin{theorem}
    设$f\in C[a,b]$,$\Phi$为$C[a,b]$的有限维子空间,则存在唯一的$\phi^*\in \Phi$,使得
    \begin{align}
        \|f-\phi^*\| = \min_{\phi\in \Phi} \|f-\phi\|
    \end{align}
\end{theorem}
\subsubsection{证明:}

\subsection{广义傅里叶展开收敛性的证明}
1、广义傅里叶级数收敛到$(C[a,b],\|\cdot\|_2)$的完备化空间$\bar{(C[a,b],\|\cdot\|_2)}$中的一个元素。
若$\|\cdot\|_2$为权系数$\rho$的内积诱导范数,则$\bar{(C[a,b],\|\cdot\|_2)}$即为$L^2_\rho(a,b)$。

2、设$f\in C[a,b]$,$\{\phi_m\}_{m=0}^\infty$为$C[a,b]$中的规范正交函数组,则$f$在$\{\phi_m\}$下的广义傅里叶展开
    \begin{align}
        f \sim \sum_{m=0}^{\infty} a_m \phi_m(x), \quad a_m = (f,\phi_m)
    \end{align}
    在$C[a,b]$中一致收敛于$f$,即
    \begin{align}
        \lim_{n\rightarrow \infty} \left\| f - \sum_{m=0}^{n} a_m \phi_m(x) \right\|_\infty = 0
    \end{align}

    \subsubsection{证明:}
    提示:1的证明用到$\sum_{i=0}^{\infty} |a_i|^2 < \infty$,2的证明用到Weierstrass逼近定理。




\subsection{最佳一致逼近存在性定理}

\begin{theorem}
    [最佳一致逼近存在性定理]
    设$f\in C[a,b]$,则在$n$次多项式空间$P_n$中,存在一个多项式$p_n^*\in P_n$,使得
    \begin{align}
        \|f - p_n^*\|_\infty = \min_{p_n\in P_n} \|f - p_n\|_\infty
    \end{align}
    即$P_n$中关于$f\in C[a,b]$的最小偏差是可以达到的。
\end{theorem}

\subsubsection{证明:}
提示:利用偏差泛函的性质


\subsection{Chebyshev交错点组定理}
\begin{theorem}
    [Chebyshev交错点组定理]
    设$f\in C[a,b]$,$p_n^*\in P_n$为$f$在$P_n$中的最佳一致逼近多项式,则存在$a\leq x_0 < x_1 < ... < x_{n+1} \leq b$,使得
    \begin{align}
        f(x_i) - p_n^*(x_i) = (-1)^i \|f - p_n^*\|_\infty, \quad i=0,1,...,n+1
    \end{align}
    即误差函数在$n+2$个点上达到最大偏差且符号交替。
\end{theorem}

\subsubsection{证明:}
提示:分别证明必要性和充分性,且均用到反证法。

\subsection{最佳一致逼近唯一性定理}
\begin{theorem}
    [最佳一致逼近唯一性定理]
    设$f\in C[a,b]$,则$f$在$n$次多项式空间$P_n$中的最佳一致逼近多项式$p_n^*$是唯一的。
\end{theorem}
\subsubsection{证明:}
提示:反证法。




\section{重要例题与习题}
\subsubsection{求解最佳平方逼近}
记$\Phi=\mathrm{span}(1,x^2)$,求$x\in[0,1]$上在$\Phi$中的最佳平方逼近。


\chapter{数值微积分}
\section{补充内容}

\section{重要证明}
\subsection{闭型Newton-Cotes公式的导出与误差分析}

\subsection{开型Newton-Cotes公式的导出与误差分析}


\section{重要习题与例题}


\chapter{常微分方程数值解}
\section{补充内容}

\section{重要证明}
\subsection{y的各阶导数}

当$y'=f(t,y)$时,求y的各阶导数。

符号规范:记$\partial f/\partial x_i=f_i$,即$f_i$是f对第i个变量的偏导数。在当前情境中,有 
\begin{align}
    f_1=&\frac{\partial f}{\partial t}\\
    f_2=&\frac{\partial f}{\partial y}
\end{align}

\subsubsection{求解:}
题目条件为
\begin{align}
    \frac{\mathrm{d}y}{\mathrm{d}t}=f(t,y)
\end{align}

即将对y的求导转化为了对f的全微分。于是有算符关系
\begin{align}
    \frac{\mathrm{d}}{\mathrm{d}t}=\frac{\partial }{\partial t}+\frac{\mathrm{d} y}{\mathrm{d} t}\frac{\partial }{\partial y}=\frac{\partial }{\partial t}+f\frac{\partial }{\partial y}
\end{align}

因此
\begin{align}
    y'=&f\\
    y''=&\frac{\mathrm{d}f}{\mathrm{d}t}=\frac{\partial f}{\partial t}+f\frac{\partial f}{\partial y}\\
    =&f_1+ff_2\\
    y'''=&\frac{\mathrm{d}y''}{\mathrm{d}t}=\frac{\mathrm{d}f_1}{\mathrm{d}t}+f\frac{\mathrm{d}f_2}{\mathrm{d}t}+f_2\frac{\mathrm{d}f}{\mathrm{d}t}\\
    =&\frac{\partial f_1}{\partial t}+f\frac{\partial f_1}{\partial y}+f\left( \frac{\partial f_2}{\partial t}+f\frac{\partial f_2}{\partial y} \right)+f_2\left(  \frac{\partial f}{\partial t}+f\frac{\partial f}{\partial y} \right)\\
    =&f_11+ff_{12}+f(f_{21}+ff_{22})+f_2(f_1+ff_2)\\
    =&f_{11}+2ff_{12}+f^2f_{22}+f_2(f_1+ff_2)
\end{align}


\subsection{一步误差与局部截断误差的关系}
\begin{align}
    (1-h_{n+1}L)|\tilde{R}_{n+1}|\leq  |R_{n+1}| \leq (1+h_{n+1}L)|\tilde{R}_{n+1}|
\end{align}


\section{重要习题与例题}
\subsection{求相容阶和主局部截断误差:泰勒级数展开法}
\subsubsection{欧拉法}

\subsubsection{梯形方法}

\subsection{显式RK的推导}
\subsubsection{基本思路:}

局部截断误差为
    \begin{align}
        R=&y(t+h)-y(t)-h\Phi(t,y,f)\\
            =&\sum_{k=0}^\infty \frac{1}{k!}y^{(k)}(t)h^k\\
            &-y(t)\\
            &-h\cdot\sum_{k=1}^\infty \phi_k h^k
    \end{align}


要令RK具有n阶相容性,即要令
\begin{align}
    R=p_{n+1}(t)h^{n+1}+O(h^{n+2})
\end{align}
需要消去R中低于$h^{n+1}$的各阶项,即可得到一系列关于$a_{ij},b_i,c_i$的方程。

计算方法即为:
\begin{itemize}
    \item 在t点对$t(t+h)$作泰勒展开
    \item 在$(t_n,y_n)$对$\phi(t_n,y_n,f)$作多元泰勒展开
    \item 令各阶系数相等,即 
    \begin{align}
        \phi_k=\frac{1}{(k+1)!}y^{(k+1)}
    \end{align}
\end{itemize}

\subsubsection{显式二阶推导}
显式RK参数矩阵:
\begin{align}
    \begin{array}{c|cc}
        c_1=0 & 0 & 0 \\
        c_2 & a_{21} & 0 \\
        \hline
         & b_1\,\,+ &  b_2=1
    \end{array}
\end{align}

RK斜率:
\begin{align}
    k_1=&f(t_n,y_n)\\
    k_2=&f(t_n+c_2h,y_n+a_{21}hk_1)=f(t_n+c_2h,y_n+a_{21}hf(t_n,y_n))
\end{align}

增量函数:
\begin{align}
    \phi=b_1k_1+b_2k_2
\end{align}

对$k_1$, $k_2$作泰勒展开:
\begin{align}
    k_1=&f(t_n,y_n)=f\\
    k_2=&f+f_1c_2h+f_2\cdot c_2hf+O(h^2)
\end{align}

则$\phi=b_1k_1+b_2k_2$的展开可表示为
\begin{align}
    \text{常数项:}\phi_0=&(b_1+b_2)f\\
    \text{一次项系数:}\phi_1=&b_2c_2f_1+b_2c_2ff_2
\end{align}

对y进行泰勒展开:
\begin{align}
    y'=&f(t,y)\\
    y''=&\frac{\partial f}{\partial t}+\frac{\partial f}{\partial y}\frac{\partial y}{\partial t}=f_1+ff_2
\end{align}

因此需要满足关系
\begin{align}
    (b_1+b_2)f=&y'=f\\
    b_2c_2f_1+b_2c_2ff_2=&\frac{1}{2}y''=\frac{1}{2}(f_1+ff_2)
\end{align}

即
\begin{align}
    b_1+b_2=&1\\
    b_2c_2=&\frac{1}{2}
\end{align}

\subsubsection{显式三阶推导}
RK参数矩阵 
\begin{align}
    \begin{array}{c|ccc}
        c_1=0 & 0 & 0 & 0\\
        c_2 & a_{21} & 0 & 0\\
        c_3 & a_{31} & a_{32} & 0\\
        \hline
         & b_1\,\,+ & b_2\,\,+ & b_3=1
    \end{array}
\end{align}

RK斜率:
\begin{align}
    k_1=&f(t_n,y_n)=f\\
    k_2=&f(t_n+c_2h,y_n+a_{21}hk_1)\\
        =&f+f_1c_2h+f_2\cdot a_{21}hf\\
        &+\frac{f_{11}(c_2h)^2+2f_{12}(c_2h)(a_{21}hk_1)+f_{22}(a_{21}hk_1)^2}{2}+O(h^3)\\
        =&f+c_2f_1h+a_{21}ff_2h \\
        &+\frac{c_2^2f_{11}+2a_{21}c_2ff_{12}+a_{21}^2f^2f_{22}}{2}h^2+O(h^3)\\
        =&f+c_2(f_1+ff_2)h+\frac{c_2^2}{2}(f_{11}+2ff_{12}+f^2f_{22})h^2+O(h^3)\\
    k_3=&f(t_n+c_3h,y_n+h(a_{31}k_1+a_{32}k_2))\\
       =&f+f_1c_3h+f_2h(a_{31}k_1+a_{32}k_2)\\
       &+\frac{f_{11}c_3^2h^2+2f_{12}\cdot c_3h\cdot h(a_{31}k_1+a_{32}k_2)+f_{22}(a_{31}k_1+a_{32}k_2)^2h^2}{2}+O(h^3)\\
\end{align}

此时,向$k_3$中代入$k_1$和$k_2$时,不一定要带入到$h^2$项,只要令所有的$h^2$项系数正确即可。于是有
\begin{align}
    k_3=&f+f_1c_3h+f_2h[a_{31}f+a_{32}(f+c_2(f_1+ff_2)h)]\\
    &+\frac{h^2}{2}[c_3^2f_{11} +2c_3f_{12}(a_{31}f+a_{32}f)+f_{22}(a_{31}f+a_{32}f)^2]+O(h^3)\\
    =&f+[c_3f_1+(a_{31}+a_{32})ff_2)]h\\
    &+\frac{1}{2}[2a_{32}c_2f_2(f_1+ff_2)+c_3^2f_{11}+2c_3(a_{31}+a_{32})ff_{12}+f^2f_{22}(a_{31}+a_{32})^2]h^2+O(h^3)
\end{align}
代入$a_{31}+a_{32}=c_3$后,有
\begin{align}
    k_3=&f+c_3(f_1+ff_2)h\\
    &+\frac{1}{2}[2a_{32}c_2f_2(f_1+ff_2)+c_3^2f_{11}+2c_3^2ff_{12}+f^2f_{22}c_3^2]h^2+O(h^3)\\
    =&f+c_3(f_1+ff_2)h+\frac{1}{2}[2a_{32}c_2f_2(f_1+ff_2)+c_3^2(f_{11}+2ff_{12}+f^2f_{22})]h^2+O(h^3)
\end{align}

先考虑y的各阶导数,有
\begin{align}
    y'=&f(t,y)\\
    y''=&f_1+ff_2\\
    y'''=&f_{11}+2ff_{12}+f^2f_{22}+f_2(f_1+ff_2)
\end{align}



代入增量函数$\phi=b_1k_1+b_2k_2+b_3k_3$,可得
\begin{align}
    \text{常数项:}\phi_0=&(b_1+b_2+b_3)f\\
    \text{一次项系数:}\phi_1=&(b_2c_2+b_3c_3)(f_1+ff_2)\\
    \text{二次项系数:}\phi_2=&\frac{1}{2}\left[ 2b_3a_{32}c_2f_2(f_1+ff_2)+(b_2c_2^2+b_3c_3^2)(f_{11}+2ff_{12}+f^2f_{22})   \right]
\end{align}

由 
\begin{align}
    \phi_0=&y'=f\\
    \phi_1=&\frac{1}{2}y''=\frac{1}{2}(f_1+ff_2)\\
    \phi_2=&\frac{1}{6}y'''=\frac{1}{6}[f_{11}+2ff_{12}+f^2f_{22}+f_2(f_1+ff_2)]
\end{align}
可得方程组
\begin{align}
    b_1+b_2+b_3=1\\
    b_2c_2+b_3c_3=\frac{1}{2}\\
    a_{32}b_3c_2=\frac{1}{6}\\
    b_2c_2^2+b_3c_3^2=\frac{1}{3}
\end{align}








\end{document}