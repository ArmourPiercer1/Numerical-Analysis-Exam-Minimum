\documentclass{book}

% 中文字体支持
\usepackage{ctex}
\usepackage{xeCJK}

% 数学公式支持
\usepackage{amsmath,amssymb,amsthm}
\usepackage{mathtools}
\usepackage{bm} % 粗体数学符号

% 图表支持
\usepackage{graphicx}
\usepackage{tikz}
\usepackage{pgfplots}
\pgfplotsset{compat=1.17}
\usepackage{float}

% 页面布局
\usepackage{geometry}
\geometry{left=2.5cm,right=2.5cm,top=3cm,bottom=3cm}
\usepackage{fancyhdr}
\usepackage{lastpage}

% 目录和超链接
\usepackage{hyperref}
\hypersetup{
    colorlinks=true,
    linkcolor=blue,
    filecolor=magenta,      
    urlcolor=cyan,
    citecolor=red
}

% 代码支持
\usepackage{listings}
\usepackage{xcolor}

% 表格支持
\usepackage{booktabs}
\usepackage{multirow}
\usepackage{array}

% 特殊形状支持
\usepackage{xcolor}

% % 副标题
% \usepackage{titling}

% 定理环境
\newtheorem{theorem}{定理}[section]
\newtheorem{lemma}[theorem]{引理}
\newtheorem{corollary}[theorem]{推论}
\newtheorem{proposition}[theorem]{命题}
\newtheorem{definition}[theorem]{定义}
\newtheorem{example}[theorem]{例题}
\newtheorem{exercise}[theorem]{练习}

% 待填写内容标记
\NewDocumentCommand{\tofill}{O{} m}{
    \textbf{\textit{\textcolor{red}{待填写:\IfValueT{#1}{(#1)}#2}}}
}

% 页眉页脚设置
\setlength{\headheight}{14.5pt}
\pagestyle{fancy}
\fancyhf{}
\fancyhead[L]{数值分析课程总结}
\fancyhead[R]{\today}
\fancyfoot[C]{\thepage/\pageref{LastPage}}

% 代码样式设置
\lstset{
    basicstyle=\ttfamily\small,
    keywordstyle=\color{blue}\bfseries,
    commentstyle=\color{green!60!black},
    stringstyle=\color{red},
    showstringspaces=false,
    numbers=left,
    numberstyle=\tiny\color{gray},
    frame=single,
    breaklines=true
}

% 文档信息
\title{Numerical Analysis \\ Exam Appendix}
\author{Astral Projection}
\date{\today}

\begin{document}

\maketitle
\tableofcontents
\newpage

\chapter{数学基础知识}

\section{补充内容}
\subsection{矩阵的可分性}



\section{重要证明}

\subsection{Legender多项式的零平方误差最小性质}
\begin{theorem}
    在所有首项为1的n次多项式中,Legendre多项式$\tilde{P}_n(x)$在$[-1,1]$上与零的平方误差最小。
\end{theorem}
\subsubsection{证明:}

考虑一个首项为1的n次多项式$f\in \mathbb{P}_n$,必有
\begin{align}
    f=\tilde{P}_n(x)+\sum_{k=0}^{n-1} a_k \tilde{P}_k(x)
\end{align}

考虑误差函数
\begin{align}
    \phi(\boldsymbol{a})=&\|f(x)-0\|^2_2\\
    =&\int_{-1}^1 \tilde{P}_n^2(x) \mathrm{d}x + 2\sum_{k=0}^{n-1} a_k \int_{-1}^1 \tilde{P}_n(x)\tilde{P}_k(x) \mathrm{d}x + \sum_{k=0}^{n-1} a_k^2 \int_{-1}^1 \tilde{P}_k^2(x) \mathrm{d}x
\end{align}
由Legendre多项式的正交性,
\begin{align}
    \int_{-1}^1 \tilde{P}_i(x)\tilde{P}_j(x)\mathrm{d}x=\frac{2}{2i+1}\delta_{ij}
\end{align}
于是
\begin{align}
    \phi(\boldsymbol{a})=&\int_{-1}^1 \tilde{P}_n^2(x) \mathrm{d}x + \sum_{k=0}^{n-1} a_k^2 \int_{-1}^1 \tilde{P}_k^2(x) \mathrm{d}x\\
\end{align}
当且仅当$a_k=0$时,$\phi(\boldsymbol{a})$取得最小值,即Legendre多项式在所有首项为1的n次多项式中与零的平方误差最小。


\subsection{Chebyshev多项式的零一致误差最小性质}
\begin{theorem}
    在所有首项为$1$的n次多项式中,Chebyshev多项式$T_n(x)$在$[-1,1]$上与零的一致误差最小。
\end{theorem}
\subsubsection{证明:}


\paragraph{方法1:将一致误差转化,并考虑在$\mathbb{P}_{n-1}$上的最佳一致逼近多项式}
\begin{align}
    f=\tilde{T}_n(x)+\sum_{k=0}^{n-1} a_k \tilde{T}_k(x)
\end{align}

考虑误差函数 
\begin{align}
    \Delta(f,0)=&\|f(x)-0\|_\infty\\
    =&\|\tilde{T}_n(x)+\sum_{k=0}^{n-1} a_k \tilde{T}_k(x)\|_\infty\\
    =&\|\tilde{T}_n(x)-\left( -\sum_{k=0}^{n-1} a_k \tilde{T}_k(x)\right)\|_\infty\\
    =&\Delta (\tilde{T}_n,-\sum_{k=0}^{n-1} a_k \tilde{T}_k(x))
\end{align}

该误差函数取得最小值时,即为
\begin{align}
    \min_{\boldsymbol{a}}\Delta (\tilde{T}_n,-\sum_{k=0}^{n-1} a_k \tilde{T}_k(x))=&\mathrm{dist} (\tilde{T}_n,\mathrm{span}\{\tilde{T}_0,\tilde{T}_1,...,\tilde{T}_{n-1}\})\\
    =&\mathrm{dist}(\tilde{T}_n,\mathbb{P}_{n-1})
\end{align}

则只要证$\mathrm{dist}(\tilde{T}_n,\mathbb{P}_{n-1})=\mathrm{dist}(\tilde{T}_n,0)$即可。

考虑到$\tilde{T}_n$在$[-1,1]$上恰有$n+1$个极值点,且这些极值点构成了一个n+1个点的交错点组,故0就是$\tilde{T}_n$在 $\mathbb{P}_{n-1}$中的最佳一致逼近多项式,因此
\begin{align}
    \mathrm{dist}(\tilde{T}_n,\mathbb{P}_{n-1})=\mathrm{dist}(\tilde{T}_n,0)
\end{align}

因此,当$\boldsymbol{a}=\boldsymbol{0}$时,误差函数可以取得最小值,即Chebyshev多项式在所有首项为1的n次多项式中与零的一致误差最小。


\paragraph{方法2:利用交错点组证明差多项式为0}
假设存在另一个首项为1的n次多项式$f_n(x)$,使得$\|f_n(x)-0\|_\infty < \|T_n(x)-0\|_\infty$。

考虑
\begin{align}
    Q(x)=f_n(x)-\tilde{T}_n(x)\in \mathbb{P}_{n-1}
\end{align}

在$\tilde{T}_n$的n+1个极值点$x_k$处,设$\tilde{T}_n$取得极值$\pm M$,则由于$\|f_n(x)-0\|_\infty < \|T_n(x)-0\|_\infty$,
必有 
\begin{align}
    |f_n(x_k)|<M
\end{align}
因此
\begin{align}
    Q(x_k)=\begin{cases}
        M-p(x_k)>0,&k\text{为偶数}\\
        -M-p(x_k)<0,&k\text{为奇数}
    \end{cases}
\end{align}
于是Q在区间内n+1个点上符号交替,故Q至少有n个零点。由于$Q\in \mathbb{P}_{n-1}$,$Q$只能为0. 于是$f_n(x)=\tilde{T}_n(x)$,与假设矛盾。


\section{例题与习题}

\chapter{函数插值与重构}
\section{重要证明}

\subsection{插值多项式的误差函数}
\begin{theorem}
    [插值多项式误差公式]
    设$f\in C^{n+1}[a,b]$,$x_0,x_1,\ldots,x_n$为区间$[a,b]$上的$n+1$个两两不同的节点,则对任意$x\in[a,b]$,存在$\xi\in(a,b)$,使得
    \begin{align}
        f(x) - p(x) = \frac{f^{(n+1)}(\xi)}{(n+1)!} \prod_{i=0}^{n} (x - x_i)
    \end{align}
    其中$p(x)$为通过节点$(x_i,f(x_i))$的插值多项式。
\end{theorem}

\subsubsection{证明:}


\subsection{分片线性插值和分片三次Hermite插值的收敛性定理}
命题:
定义
\begin{align}
    h=\max_{1\leq i \leq n}(x_i - x_{i-1})
\end{align}
则进行分片线性插值时,
\begin{itemize}
    \item 若$f\in C[a,b]$,则$\lim_{h\rightarrow 0}\|f-\phi\|_\infty\rightarrow 0$
    \item 若$f\in C^1[a,b]$,则$\|f-\phi\|_\infty \leq \frac{h}{2}\|f'\|_\infty$
    \item 若$f\in C^2[a,b]$,则$\|f-\phi\|_\infty \leq \frac{h^2}{8}\|f''\|_\infty$
\end{itemize}
进行分片三次Hermite插值时,
\begin{itemize}
    \item 若$f\in C^1[a,b]$,则$\|f-\phi\|_\infty \leq ch\|f'\|_\infty$
    \item 若$f\in C^2[a,b]$,则$\|f-\phi\|_\infty \leq ch^2\|f''\|_\infty$
    \item 若$f\in C^3[a,b]$,则$\|f-\phi\|_\infty \leq ch^3\|f'''\|_\infty$
    \item 若$f\in C^4[a,b]$,则$\|f-\phi\|_\infty \leq \frac{1}{384}h^4\|f^{(4)}\|_\infty$
\end{itemize}

\subsubsection{证明:}
Note:分片线性插值只要做泰勒展开即可;
分片三次Hermite插值在4阶光滑性时可以通过余项公式给出,低于4阶光滑性需要用到Peano核定理。



\subsection{三角插值多项式形式、相多项式形式}

\subsection{三角插值的误差分析}


\section{函数逼近}

\section{补充内容}

\subsection{Schauder基}



\section{重要证明}

\subsection{最佳平方逼近存在唯一性定理}
\begin{theorem}
    设$f\in C[a,b]$,$\Phi$为$C[a,b]$的有限维子空间,则存在唯一的$\phi^*\in \Phi$,使得
    \begin{align}
        \|f-\phi^*\| = \min_{\phi\in \Phi} \|f-\phi\|
    \end{align}
\end{theorem}
\subsubsection{证明:}

\subsection{广义傅里叶展开收敛性的证明}
1、广义傅里叶级数收敛到$(C[a,b],\|\cdot\|_2)$的完备化空间$\bar{(C[a,b],\|\cdot\|_2)}$中的一个元素。
若$\|\cdot\|_2$为权系数$\rho$的内积诱导范数,则$\bar{(C[a,b],\|\cdot\|_2)}$即为$L^2_\rho(a,b)$。

2、设$f\in C[a,b]$,$\{\phi_m\}_{m=0}^\infty$为$C[a,b]$中的规范正交函数组,则$f$在$\{\phi_m\}$下的广义傅里叶展开
    \begin{align}
        f \sim \sum_{m=0}^{\infty} a_m \phi_m(x), \quad a_m = (f,\phi_m)
    \end{align}
    在$C[a,b]$中一致收敛于$f$,即
    \begin{align}
        \lim_{n\rightarrow \infty} \left\| f - \sum_{m=0}^{n} a_m \phi_m(x) \right\|_\infty = 0
    \end{align}

    \subsubsection{证明:}
    提示:1的证明用到$\sum_{i=0}^{\infty} |a_i|^2 < \infty$,2的证明用到Weierstrass逼近定理。




\subsection{最佳一致逼近存在性定理}

\begin{theorem}
    [最佳一致逼近存在性定理]
    设$f\in C[a,b]$,则在$n$次多项式空间$P_n$中,存在一个多项式$p_n^*\in P_n$,使得
    \begin{align}
        \|f - p_n^*\|_\infty = \min_{p_n\in P_n} \|f - p_n\|_\infty
    \end{align}
    即$P_n$中关于$f\in C[a,b]$的最小偏差是可以达到的。
\end{theorem}

\subsubsection{证明:}
提示:利用偏差泛函的性质


\subsection{Chebyshev交错点组定理}
\begin{theorem}
    [Chebyshev交错点组定理]
    设$f\in C[a,b]$,$p_n^*\in P_n$为$f$在$P_n$中的最佳一致逼近多项式,则存在$a\leq x_0 < x_1 < ... < x_{n+1} \leq b$,使得
    \begin{align}
        f(x_i) - p_n^*(x_i) = (-1)^i \|f - p_n^*\|_\infty, \quad i=0,1,...,n+1
    \end{align}
    即误差函数在$n+2$个点上达到最大偏差且符号交替。
\end{theorem}

\subsubsection{证明:}
提示:分别证明必要性和充分性,且均用到反证法。

\subsection{最佳一致逼近唯一性定理}
\begin{theorem}
    [最佳一致逼近唯一性定理]
    设$f\in C[a,b]$,则$f$在$n$次多项式空间$P_n$中的最佳一致逼近多项式$p_n^*$是唯一的。
\end{theorem}
\subsubsection{证明:}
提示:反证法。




\section{重要例题与习题}
\subsubsection{求解最佳平方逼近}
记$\Phi=\mathrm{span}(1,x^2)$,求$x\in[0,1]$上在$\Phi$中的最佳平方逼近。


\chapter{数值微积分}
\section{补充内容}

\section{重要证明}
\subsection{闭型Newton-Cotes公式的导出与误差分析}

\subsection{开型Newton-Cotes公式的导出与误差分析}


\section{重要习题与例题}


\chapter{常微分方程数值解}
\section{补充内容}
\subsection{一致Lipschitz常数的求解}
若$f(t,y)$关于y连续可微,则Lipschitz常数即为区间内$f$对$y$的偏导数的最大值,即
\begin{align}
    L = \max_{(t,y)\in D} \left| \frac{\partial f}{\partial y} \right|
\end{align}


\section{重要证明}
\subsection{y的各阶导数}

当$y'=f(t,y)$时,求y的各阶导数。

符号规范:记$\partial f/\partial x_i=f_i$,即$f_i$是f对第i个变量的偏导数。在当前情境中,有 
\begin{align}
    f_1=&\frac{\partial f}{\partial t}\\
    f_2=&\frac{\partial f}{\partial y}
\end{align}

\subsubsection{求解:}
题目条件为
\begin{align}
    \frac{\mathrm{d}y}{\mathrm{d}t}=f(t,y)
\end{align}

即将对y的求导转化为了对f的全微分。于是有算符关系
\begin{align}
    \frac{\mathrm{d}}{\mathrm{d}t}=\frac{\partial }{\partial t}+\frac{\mathrm{d} y}{\mathrm{d} t}\frac{\partial }{\partial y}=\frac{\partial }{\partial t}+f\frac{\partial }{\partial y}
\end{align}

因此
\begin{align}
    y'=&f\\
    y''=&\frac{\mathrm{d}f}{\mathrm{d}t}=\frac{\partial f}{\partial t}+f\frac{\partial f}{\partial y}\\
    =&f_1+ff_2\\
    y'''=&\frac{\mathrm{d}y''}{\mathrm{d}t}=\frac{\mathrm{d}f_1}{\mathrm{d}t}+f\frac{\mathrm{d}f_2}{\mathrm{d}t}+f_2\frac{\mathrm{d}f}{\mathrm{d}t}\\
    =&\frac{\partial f_1}{\partial t}+f\frac{\partial f_1}{\partial y}+f\left( \frac{\partial f_2}{\partial t}+f\frac{\partial f_2}{\partial y} \right)+f_2\left(  \frac{\partial f}{\partial t}+f\frac{\partial f}{\partial y} \right)\\
    =&f_11+ff_{12}+f(f_{21}+ff_{22})+f_2(f_1+ff_2)\\
    =&f_{11}+2ff_{12}+f^2f_{22}+f_2(f_1+ff_2)
\end{align}


\subsection{一步误差与局部截断误差的关系}
\begin{align}
    (1-h_{n+1}L)|\tilde{R}_{n+1}|\leq  |R_{n+1}| \leq (1+h_{n+1}L)|\tilde{R}_{n+1}|
\end{align}


\section{重要习题与例题}
\subsection{求相容阶和主局部截断误差:泰勒级数展开法}
\subsubsection{欧拉法}

\subsubsection{梯形方法}

\subsection{显式RK的推导}
\subsubsection{基本思路:}

局部截断误差为
    \begin{align}
        R=&y(t+h)-y(t)-h\Phi(t,y,f)\\
            =&\sum_{k=0}^\infty \frac{1}{k!}y^{(k)}(t)h^k\\
            &-y(t)\\
            &-h\cdot\sum_{k=1}^\infty \phi_k h^k
    \end{align}


要令RK具有n阶相容性,即要令
\begin{align}
    R=p_{n+1}(t)h^{n+1}+O(h^{n+2})
\end{align}
需要消去R中低于$h^{n+1}$的各阶项,即可得到一系列关于$a_{ij},b_i,c_i$的方程。

计算方法即为:
\begin{itemize}
    \item 在t点对$t(t+h)$作泰勒展开
    \item 在$(t_n,y_n)$对$\phi(t_n,y_n,f)$作多元泰勒展开
    \item 令各阶系数相等,即 
    \begin{align}
        \phi_k=\frac{1}{(k+1)!}y^{(k+1)}
    \end{align}
\end{itemize}

\subsubsection{显式二阶推导}
显式RK参数矩阵:
\begin{align}
    \begin{array}{c|cc}
        c_1=0 & 0 & 0 \\
        c_2 & a_{21} & 0 \\
        \hline
         & b_1\,\,+ &  b_2=1
    \end{array}
\end{align}

RK斜率:
\begin{align}
    k_1=&f(t_n,y_n)\\
    k_2=&f(t_n+c_2h,y_n+a_{21}hk_1)=f(t_n+c_2h,y_n+a_{21}hf(t_n,y_n))
\end{align}

增量函数:
\begin{align}
    \phi=b_1k_1+b_2k_2
\end{align}

对$k_1$, $k_2$作泰勒展开:
\begin{align}
    k_1=&f(t_n,y_n)=f\\
    k_2=&f+f_1c_2h+f_2\cdot c_2hf+O(h^2)
\end{align}

则$\phi=b_1k_1+b_2k_2$的展开可表示为
\begin{align}
    \text{常数项:}\phi_0=&(b_1+b_2)f\\
    \text{一次项系数:}\phi_1=&b_2c_2f_1+b_2c_2ff_2
\end{align}

对y进行泰勒展开:
\begin{align}
    y'=&f(t,y)\\
    y''=&\frac{\partial f}{\partial t}+\frac{\partial f}{\partial y}\frac{\partial y}{\partial t}=f_1+ff_2
\end{align}

因此需要满足关系
\begin{align}
    (b_1+b_2)f=&y'=f\\
    b_2c_2f_1+b_2c_2ff_2=&\frac{1}{2}y''=\frac{1}{2}(f_1+ff_2)
\end{align}

即
\begin{align}
    b_1+b_2=&1\\
    b_2c_2=&\frac{1}{2}
\end{align}

\subsubsection{显式三阶推导}
RK参数矩阵 
\begin{align}
    \begin{array}{c|ccc}
        c_1=0 & 0 & 0 & 0\\
        c_2 & a_{21} & 0 & 0\\
        c_3 & a_{31} & a_{32} & 0\\
        \hline
         & b_1\,\,+ & b_2\,\,+ & b_3=1
    \end{array}
\end{align}

RK斜率:
\begin{align}
    k_1=&f(t_n,y_n)=f\\
    k_2=&f(t_n+c_2h,y_n+a_{21}hk_1)\\
        =&f+f_1c_2h+f_2\cdot a_{21}hf\\
        &+\frac{f_{11}(c_2h)^2+2f_{12}(c_2h)(a_{21}hk_1)+f_{22}(a_{21}hk_1)^2}{2}+O(h^3)\\
        =&f+c_2f_1h+a_{21}ff_2h \\
        &+\frac{c_2^2f_{11}+2a_{21}c_2ff_{12}+a_{21}^2f^2f_{22}}{2}h^2+O(h^3)\\
        =&f+c_2(f_1+ff_2)h+\frac{c_2^2}{2}(f_{11}+2ff_{12}+f^2f_{22})h^2+O(h^3)\\
    k_3=&f(t_n+c_3h,y_n+h(a_{31}k_1+a_{32}k_2))\\
       =&f+f_1c_3h+f_2h(a_{31}k_1+a_{32}k_2)\\
       &+\frac{f_{11}c_3^2h^2+2f_{12}\cdot c_3h\cdot h(a_{31}k_1+a_{32}k_2)+f_{22}(a_{31}k_1+a_{32}k_2)^2h^2}{2}+O(h^3)\\
\end{align}

此时,向$k_3$中代入$k_1$和$k_2$时,不一定要带入到$h^2$项,只要令所有的$h^2$项系数正确即可。于是有
\begin{align}
    k_3=&f+f_1c_3h+f_2h[a_{31}f+a_{32}(f+c_2(f_1+ff_2)h)]\\
    &+\frac{h^2}{2}[c_3^2f_{11} +2c_3f_{12}(a_{31}f+a_{32}f)+f_{22}(a_{31}f+a_{32}f)^2]+O(h^3)\\
    =&f+[c_3f_1+(a_{31}+a_{32})ff_2)]h\\
    &+\frac{1}{2}[2a_{32}c_2f_2(f_1+ff_2)+c_3^2f_{11}+2c_3(a_{31}+a_{32})ff_{12}+f^2f_{22}(a_{31}+a_{32})^2]h^2+O(h^3)
\end{align}
代入$a_{31}+a_{32}=c_3$后,有
\begin{align}
    k_3=&f+c_3(f_1+ff_2)h\\
    &+\frac{1}{2}[2a_{32}c_2f_2(f_1+ff_2)+c_3^2f_{11}+2c_3^2ff_{12}+f^2f_{22}c_3^2]h^2+O(h^3)\\
    =&f+c_3(f_1+ff_2)h+\frac{1}{2}[2a_{32}c_2f_2(f_1+ff_2)+c_3^2(f_{11}+2ff_{12}+f^2f_{22})]h^2+O(h^3)
\end{align}

先考虑y的各阶导数,有
\begin{align}
    y'=&f(t,y)\\
    y''=&f_1+ff_2\\
    y'''=&f_{11}+2ff_{12}+f^2f_{22}+f_2(f_1+ff_2)
\end{align}



代入增量函数$\phi=b_1k_1+b_2k_2+b_3k_3$,可得
\begin{align}
    \text{常数项:}\phi_0=&(b_1+b_2+b_3)f\\
    \text{一次项系数:}\phi_1=&(b_2c_2+b_3c_3)(f_1+ff_2)\\
    \text{二次项系数:}\phi_2=&\frac{1}{2}\left[ 2b_3a_{32}c_2f_2(f_1+ff_2)+(b_2c_2^2+b_3c_3^2)(f_{11}+2ff_{12}+f^2f_{22})   \right]
\end{align}

由 
\begin{align}
    \phi_0=&y'=f\\
    \phi_1=&\frac{1}{2}y''=\frac{1}{2}(f_1+ff_2)\\
    \phi_2=&\frac{1}{6}y'''=\frac{1}{6}[f_{11}+2ff_{12}+f^2f_{22}+f_2(f_1+ff_2)]
\end{align}
可得方程组
\begin{align}
    b_1+b_2+b_3=1\\
    b_2c_2+b_3c_3=\frac{1}{2}\\
    a_{32}b_3c_2=\frac{1}{6}\\
    b_2c_2^2+b_3c_3^2=\frac{1}{3}
\end{align}

\subsection{课本习题}

\subsubsection{1、显式Euler方法求解初值问题并分析误差}
用显式Euler方法来求解初值问题,列出数值解与解析解的误差.
\begin{enumerate}
\item[(1)] 
\begin{align*}
y' = 1 + \frac{y}{x},\ 1 \leq x \leq 2,\ y(1) = 2,\ \text{取} \ h = 0.25\ (\text{相应解析解为} \ y(x) = x\ln x + 2x).
\end{align*}
\item[(2)] 
\begin{align*}
y' = \cos x + \sin 3x,\ 0 \leq x \leq 1,\ y(0) = 1,\ \text{取} \ h = 0.25\ (\text{相应解析解为} \ y(x) = \frac{1}{2}\sin 2x - \frac{1}{3}\cos 3x + \frac{4}{3}).
\end{align*}
\end{enumerate}
\paragraph{证明:}
显式Euler方法的格式为:
\begin{align}
    y_{n+1}&=y_n+h_{n+1}f(t_n,y_n)
\end{align}

(1)
\begin{align}
    x_0=1,\quad y_0=&y(1)=2\\
    x_1=1.25,\quad y_1=&y_0+h f(x_0,y_0)=2+0.25\times \left(1+\frac{2}{1}\right)=2+0.25\times 3=2.75\\
    x_2=1.5,\quad y_2=&y_1+hf(x_1,y_1)=2.75+0.25\times (1+\frac{2.75}{1.25})=2.75+0.25\times 3.2=3.55\\
    x_3=1.75,\quad y_3=&y_2+hf(x_2,y_2)=3.55+0.25\times (1+\frac{3.55}{1.5})=3.55+0.25\times 3.3667=4.3917\\
    x_4=2.0,\quad y_4=&y_3+hf(x_3,y_3)=4.3917+0.25\times (1+\frac{4.3917}{1.75})=4.3917+0.25\times 3.5084=5.2698
\end{align}

(2)
\begin{align}
    x_0=0,\quad y_0=&y(0)=1\\
    x_1=0.25,\quad y_1=&y_0+hf(x_0,y_0)=1+0.25\times (\cos 0 + \sin 0)=1+0.25\times 1=1.25\\
    x_2=0.5,\quad y_2=&y_1+hf(x_1,y_1)=1.25+0.25\times (\cos 0.25 + \sin 0.75)\\=&1.25+0.25\times (0.9689 + 0.6816)=1.25+0.25\times 1.6505=1.6626\\
    x_3=0.75,\quad y_3=&y_2+hf(x_2,y_2)=1.6626+0.25\times (\cos 0.5 + \sin 1.5)\\=&1.6626+0.25\times (0.8776 + 0.9975)=1.6626+0.25\times 1.8751=2.1304\\
    x_4=1.0,\quad y_4=&y_3+hf(x_3,y_3)=2.1304+0.25\times (\cos 0.75 + \sin 2.25)\\=&2.1304+0.25\times (0.7317 + 0.7781)=2.1304+0.25\times 1.5098=2.5079
\end{align}





\subsubsection{2、改进Euler方法求解初值问题并分析误差}
用改进的Euler方法解第1题中的问题,并列出数值解和相应解析解的误差.
\paragraph{证明:}





\subsubsection{3、梯形方法迭代格式的收敛性证明}
用梯形方法解初值问题
\begin{align*}
\begin{cases}
y' = e^x\sin(xy),\ x \in [0,1],\\
y(0) = 1.
\end{cases}
\end{align*}
若迭代初值为\( y^{(0)}_{s+1} = y_s + hf(x_s,y_s) \),试证逐步长\( h \)使迭代格式
\begin{align*}
y^{(k+1)}_{s+1} = y_s + \frac{h}{2}\left[ f(x_s,y_s) + f(x_{s+1},y^{(k)}_{s+1}) \right],\ s = 0,1,\cdots
\end{align*}
是收敛的.
\paragraph{证明:}

求解过程为:
\begin{enumerate}
    \item 微分方程数值解问题:用显欧拉法给出一套迭代初值$\{y_k^{(0)}\}$
    \item 非线性方程数值解问题:在每个点上,用给出的迭代格式进行迭代,求解非线性方程的数值解
\end{enumerate}

因此,要证明第二步非线性方程的迭代格式是收敛的,只要证明它满足压缩映像原理。

\begin{align}
    |y_{n+1}^{(s+1)} - y_{n+1}^{(s)}| &= \left| \frac{h}{2} \left[ f(x_{n+1},y_{n+1}^{(s)}) - f(x_{n+1},y_{n+1}^{(s-1)}) \right] \right| \\
    =&\frac{h}{2}|f(x_{n+1},y_{n+1}^{(s)}) - f(x_{n+1},y_{n+1}^{(s-1)})| 
\end{align}

该微分方程有唯一数值解,即满足Lipschitz条件。对应的Lipschitz常数L可以将$\Delta f$与$\Delta y$联系起来。
\begin{align}
    L=\max_{x\in[0,1],y\in\mathbb{R}} \left| \frac{\partial f}{\partial y} \right| = \max_{x\in[0,1],y\in\mathbb{R}} |xe^x\cos(xy)| \leq e
\end{align}

因此迭代过程有
\begin{align}
    |y_{n+1}^{(s+1)} - y_{n+1}^{(s)}| &\leq \frac{h}{2}L |y_{n+1}^{(s)} - y_{n+1}^{(s-1)}| \\
    &\leq \frac{he}{2} |y_{n+1}^{(s)} - y_{n+1}^{(s-1)}|
\end{align}
压缩映像原理要求
\begin{align}
    \frac{h}{2}e<1
\end{align}
即
\begin{align}
    h<\frac{2}{e}
\end{align}






\subsubsection{4、梯形方法求解特定初值问题的结论证明}
用梯形方法解初值问题\( y' = -y,\ y(1) = 1 \),试证明:
\begin{enumerate}
\item[(1)] 取\( y_0 = y(1) = 1 \),有\( y_n = \left( \frac{2 - h}{2 + h} \right)^n \).
\item[(2)] 当\( h \to 0,x = nh \)不变时,\( y_n \)收敛于初值问题的准确解\( e^{ - x_n} \).
\end{enumerate}
\paragraph{证明:}

(1)梯形方法迭代格式:
\begin{align}
    y_{n+1}=y_n+\frac{h}{2}[f(t_n,y_n)+f(t_{n+1},y_{n+1})]
\end{align}
由于该问题是自治的,不需要求解非线性方程,
\begin{align}
    &y_{n+1}=y_n+\frac{h}{2}[-y_n-y_{n+1}]\\
    \Rightarrow &y_{n+1}=\frac{2-h}{2+h}y_n
\end{align}
初值为$y_0=1$,因此
\begin{align}
    y_n=\left( \frac{2-h}{2+h} \right)^n
\end{align}


(2)代入$n=x/h$,求极限即可。注意,要用到对数求极限。





\subsubsection{5、单步法的局部截断误差与绝对稳定性分析}
试求出单步法
\begin{align*}
\begin{cases}
y_{n+1} = y_n + hf(x_{n+1},y_n+hf(x_n,y_n)),\\
y_0 = y(x_0)
\end{cases}
\end{align*}
的局部截断误差主项及绝对稳定性区间.
\paragraph{证明:}

增量函数
\begin{align}
    \phi(x_n,x_{n+1};y_n,y_{n+1};f)=f(x_{n+1},y_n+hf(x_n,y_n))
\end{align}

局部截断误差:
\begin{align}
    R_{n+1}=&y(x+h)-y(x)-h\phi(x_n,x_{n+1};y_n,y_{n+1};f)\\
    =&y(x+h)-y(x)-hf(x_{n+1},y_n+hf(x_n,y_n))\\
    =&y(x+h)-y(x)-hf(x+h,y+hf)
\end{align}

求y的各阶导数:
\begin{align}
    &y'=f\\
    &y''=f_1+ff_2
\end{align}

求$\phi$的各阶导数:
\begin{align}
    \phi=f(x+h,y+hf)=f+(f_1+ff_2)h+O(h^2)
\end{align}

于是有
\begin{align}
    R_{n+1}=&y(x)+fh+\frac{1}{2}(f_1+ff_2)h^2+O(h^3)-y(x)\\
    &-h(f+(f_1+ff_2)h+O(h^2))\\
    =&-\frac{1}{2}h^2(f_1+ff_2)+O(h^3)\\
    =&-\frac{h^2}{2}y''+O(h^3)
\end{align}
故单步法的局部截断误差主项为$-\frac{h^2}{2}y''$。

绝对稳定性分析:将单步法格式代入试验方程
\begin{align}
    y'=f=\lambda y
\end{align}
得到
\begin{align}
    y_{n+1}=&y_n+h\cdot\lambda(y_n+h\cdot \lambda y_n)\\
    =&[(h\lambda)^2+(h\lambda)+1]y_n
\end{align}
绝对稳定性要求
\begin{align}
    |E(h\lambda)| = |(h\lambda)^2+(h\lambda)+1| < 1
\end{align}
解得绝对稳定区间为复平面上满足
\begin{align}
    |z^2+z+1|<1
\end{align}
的区域。





\subsubsection{6、中点公式与Heun方法求解初值问题并分析误差}
应用中点公式及Heun方法
\begin{align*}
y_{n+1} = y_n + \frac{1}{4}hf(x_n,y_n) + \frac{3}{4}hf\left( x_n + \frac{2}{3}h,y_n + \frac{2}{3}hf(x_n,y_n) \right)
\end{align*}
计算初值问题
\begin{align*}
\begin{cases}
y' = y - x^2 + 1,\ x \in [0,1],\\
y(0) = 0.5.
\end{cases}
\end{align*}
取\( h = 0.2 \),列出数值解及相应的误差(问题的解析解为\( y(x) = (x+1)^2 - \frac{1}{2}e^{x} \)).
\paragraph{证明:}

他都不让带计算器了,我认为不会考这玩意






\subsubsection{7、多种数值方法求解初值问题的结果对比}
应用显式Euler方法、改进的Euler方法以及四阶经典Runge-Kutta方法计算初值问题
\begin{align*}
\begin{cases}
y' = -2y + 2x^2 + 2x,\ x \in [0,0.5],\\
y(0) = 1.
\end{cases}
\end{align*}
这三种方法步长\( h \)依次取为\( 0.25,0.05,0.1 \),列出相应计算结果、解析解\( y(x) = e^{-2x} + x^2 \)的结果及相应的误差.
\paragraph{证明:}

呃呃




\subsubsection{8、中点公式的局部截断误差分析}
试求出中点公式
\begin{align*}
y_{n+1} = y_n + hf\left( x_n + \frac{h}{2},y_n + \frac{1}{2}hf(x_n,y_n) \right)
\end{align*}
的局部截断误差主项.
\paragraph{证明:}

对y逐次求导:

算子:
\begin{align}
    \frac{\mathrm{d}}{\mathrm{d}x_1}=\frac{\partial}{\partial x_1}+f\frac{\partial }{\partial x_2}
\end{align}
\begin{align}
    y'=&f\\
    y''=&\frac{\mathrm{d}f}{\mathrm{d}x}=f_1+ff_2\\
    y'''=&\frac{\mathrm{d}f}{\mathrm{d}x}=\frac{\partial f_1}{\partial x_1}+f\frac{\partial f_1}{\partial x_2}+f\left(\frac{\partial f_2}{\partial x_1}+f\frac{\partial f_2}{\partial x_2}\right)+\left(\frac{\partial f}{\partial x_1}+f\frac{\partial f}{\partial x_2}\right) f_2\\
        =&f_{11}+ff_{21}+f(f_{12}+ff_{22})+f_2(f_1+ff_2)\\
        =&f_{11}+f(2f_{12}+ff_{22})+f_2(f_1+ff_2)
\end{align}


\begin{align}
    R_{n+1}=&y(x+h)-y(x)-h\phi(x_n,x_{n+1};y_n,y_{n+1};f)\\
    =&y(x+h)-y(x)-hf\left( x+\frac{h}{2},y+\frac{1}{2}hf \right)\\
    =& y(x)+fh+\frac{1}{2}(f_1+ff_2)h^2 + \frac{1}{6}[f_{11}+f(2f_{12}+ff_{22})+f_2(f_1+ff_2)]h^3+O(h^4)\\&-y(x)\\
    &-h(f+f_1\frac{h}{2}+f_2\frac{h}{2}f +\frac{f_{11}\left(\dfrac{h}{2}\right)^2+2f_{12}\left(\dfrac{hf}{2}\right)\left(\dfrac{h}{2}\right)+f_{22}\left(\dfrac{hf}{2}\right)^2}{2}   +O(h^3))\\
    =&\frac{1}{6}[f_{11}+f(2f_{12}+ff_{22})+f_2(f_1+ff_2)]h^3-\frac{1}{8}(f_{11}+2ff_{12}+f^2f_{22})h^3+O(h^4)\\
    =&\frac{1}{24}[f_{11}+f(2f_{12}+ff_{22})+4f_2(f_1+ff_2)]h^3+O(h^4)
\end{align}

故局部截断误差主项为 
\begin{align}
    \frac{1}{24}[f_{11}+f(2f_{12}+ff_{22})+4f_2(f_1+ff_2)]h^3
\end{align}



\subsubsection{9、特定初值问题下数值方法的近似值一致性证明}
对于初值问题
\begin{align*}
\begin{cases}
y' = -y + x + 1,\ x \in [0,1],\\
y(0) = 1,
\end{cases}
\end{align*}
试证明用中点方法、改进的Euler方法以及Heun方法(见第6题)求解,对任意的步长\( h \)均有相同的近似值.
\paragraph{证明:}

只要证明其一阶泰勒展开式相等。
\begin{itemize}
    \item 中点方法:
    \begin{align}
        y_{n+1}=&y_n+hf(x_n+\frac{1}{2}h,y_n+\frac{1}{2}hf)\\
        \approx & y_n+h(f+\frac{1}{2}hf_1+\frac{1}{2}hff_2) \\
        =&y_n+hf+\frac{1}{2}(f_1+ff_2)h^2
    \end{align}
    \item Heun方法:
    \begin{align}
        y_{n+1} =& y_n + \frac{1}{4}hf(x_n,y_n) + \frac{3}{4}hf\left( x_n + \frac{2}{3}h,y_n + \frac{2}{3}hf(x_n,y_n) \right)\\
        \approx & y_n+\frac{1}{4}hf+\frac{3}{4}h(f+\frac{2}{3}hf_1+\frac{2}{3}hff_2)\\
        =&y_n+hf+\frac{1}{2}(f_1+ff_2)h^2
    \end{align}
    \item 改进Euler方法:
    \begin{align}
        y_{n+1}=&y_n+\frac{h}{2}[f+f(t+h,y+hf)]\\
        \approx &y_n+\frac{h}{2}[f+f+hf_1+hff_2]\\
        =&y_n+hf+\frac{1}{2}(f_1+ff_2)h^2
    \end{align}
\end{itemize}

故三者具有相同的近似值。



\subsubsection{10、隐式中点方法的绝对稳定性区间分析}
试求出隐式中点方法
\begin{align*}
y_{n+1} = y_n + hf\left( x_n + \frac{1}{2}h,\frac{1}{2}(y_n + y_{n+1}) \right)
\end{align*}
的绝对稳定性区间(推广§3.5的方法).
\paragraph{证明:}

试验方程:
\begin{align}
    y'=f(y)=\lambda y
\end{align}

代入隐式中点方法有
\begin{align}
    &y_{n+1}=y_n+h\cdot \frac{\lambda}{2}(y_n+y_{n+1})\\
    \Rightarrow & y_{n+1}=\frac{2+(h\lambda)}{2-(h\lambda)}y_n 
\end{align}

则绝对稳定性区间为
\begin{align}
    &|E(h\lambda)| = \left| \frac{2+(h\lambda)}{2-(h\lambda)} \right| < 1\\
    \Rightarrow& \Re(h\lambda)<0
\end{align}
即绝对稳定性区间为负半实轴。




\subsubsection{11、二步显式Adams方法的局部截断误差推导}
试推导二步显式Adams方法与二步隐式Adams方法的局部截断误差.

\paragraph{证明:}
二步显式Adams方法:
\begin{align}
    &y_{n+1}=y_n+h\left(\frac{3}{2}f(t_n,y_n)-\frac{1}{2}f(t_{n-1},y_{n-1})\right)\\
    \Leftrightarrow& y_{n+1}-y_n=h\left(\frac{3}{2}f(t_n,y_n)-\frac{1}{2}f(t_{n-1},y_{n-1})\right)
\end{align}

局部截断误差:
\begin{align}
    R_{n+1}=&y(t_{n+1})-y(t_n)-h\left(\frac{3}{2}f(t_n,y_n)-\frac{1}{2}f(t_{n-1},y_{n-1})\right)\\
    =&y(t+h)-y(t)-h\left(  \frac{3}{2}f-\frac{1}{2}y'(t-h)  \right)\\
    =&y(t)+y'h+\frac{1}{2}y''h^2+\frac{1}{6}y'''h^3+O(h^4)\\
    &-y(t)\\
    &-h[\frac{3}{2}y'-\frac{1}{2}(y'-y''h+\frac{1}{2}y'''h^2)+O(h^3)]\\
    =&\frac{1}{6}y'''h^3+\frac{1}{4}y'''h^3+O(h^4)\\
    =&-\frac{5}{12}y'''h^3+O(h^4)
\end{align}

二步隐式Adams方法:
\begin{align}
    &y_{n+1}=y_{n}+\frac{1}{12}h(5f(t_{n+1},y_{n+1})+8f(t_{n},y_{n}) - f(t_{n-1},y_{n-1}))\\
    \Leftrightarrow& y_n-y_{n-1}=\frac{1}{12}h(5f(t_{n+1},y_{n+1})+8f(t_{n},y_{n}) - f(t_{n-1},y_{n-1}))
\end{align}

局部截断误差:
\begin{align}
    R_{n+1}=& y(t_{n+1})-y(t_{n})-\frac{1}{12}h(5f(t_{n+1},y_{n+1})+8f(t_{n},y_{n}) - f(t_{n-1},y_{n-1}))\\
    =&y(t+h)-y(t)-\frac{1}{12}h(5y'(t+h)+8y'(t) - y'(t-h))\\
    =&y(t)+y'h+\frac{1}{2}y''h^2+\frac{1}{6}y'''h^3+\frac{1}{24}y''''h^4+O(h^5)\\
    -&y(t)\\
    &-\frac{1}{12}h[5(y'+y''h+\frac{1}{2}y'''h^2+\frac{1}{6}y''''h^3+O(h^4))+8y'-(y'-y''h+\frac{1}{2}y'''h^2-\frac{1}{6}y''''h^3+Oh^(4))]\\
    =&-\frac{1}{24}y''''h^4+O(h^5)
\end{align}


\subsubsection{12、Hamming公式的局部截断误差分析}
试推导Hamming公式
\begin{align*}
y_{n+3} = \frac{1}{8}(9y_{n+2} - y_{n}) + \frac{3}{8}h\left[ f(x_{n+3},y_{n+3}) + 2f(x_{n+2},y_{n+2}) - f(x_{n+1},y_{n+1}) \right]
\end{align*}
的局部截断误差主项.
\paragraph{证明:}

活不了了



\subsubsection{13、数值积分法推导二步数值方法}
试用数值积分方法直接推导二步方法
\begin{align*}
y_{n+2} - y_{n+1} = \frac{h}{12}\left[ 5f(x_{n+2},y_{n+2}) + 8f(x_{n+1},y_{n+1}) - f(x_n,y_n) \right].
\end{align*}
\paragraph{证明:}

记$x_n=-h,\,x_{n+1}=0,\,x_{n+2}=h$,则可以
构造Lagrange插值多项式:
\begin{align}
    L_0=\frac{x^2-hx}{2h^2}\\
    L_1=\frac{h^2-x^2}{h^2}\\
    L_2=\frac{x^2+hx}{2h^2}
\end{align}

三者在$[0,h]$上的积分是
\begin{align}
    \int_{0}^hL_0\mathrm{d}x=-\frac{1}{12}h\\
    \int_0^h L_1\mathrm{d}x=\frac{2}{3}h=\frac{8}{12}h\\
    \int_0^h L_2\mathrm{d}x=\frac{5}{12}h
\end{align}

于是有 
\begin{align}
    y_{n+1}-y_n=&\int_{x_{n}}^{x_{n+1}}f(x,y(x))\mathrm{d}x=\sum_{i=0}^2 f(x_{n+i},y_{n+i})\int_0^h L_i\mathrm{d}x\\
    =&-\frac{h}{12}f_{n}+\frac{8}{12}f_{n+1}+\frac{5}{12}f_{n+2}\\
    =&\frac{h}{12}[5f(x_{n+2},y_{n+2}) + 8f(x_{n+1},y_{n+1}) - f(x_n,y_n)]
\end{align}




\subsubsection{14、Adams方法求解初值问题并与解析解对比}
用四阶的显式Adams方法和隐式Adams方法解初值问题
\begin{align*}
\begin{cases}
y' = -y + x + 1,\ x \in [0,1],\\
y(0) = 2,
\end{cases}
\end{align*}
取\( h = 0.2 \),将结果与解析解作比较(解析解\( y(x) = (x+1) + \frac{1}{2}e^{-x} \)).
\paragraph{证明:}
呜咕



\subsubsection{15、线性二步法的阶数证明}
证明线性二步法
\begin{align*}
y_{n+2} + (b - 1)y_{n+1} - by_{n} = \frac{1}{4}h\left[ (b + 3)f(x_{n+2},y_{n+2}) + (3b + 1)f(x_{n},y_{n}) \right]
\end{align*}
当\( b \neq -1 \)时是二阶的,当\( b = -1 \)时是三阶的.
\paragraph{证明:}





\subsubsection{16、线性多步法的阶数参数确定}
试确定\( \alpha \)使线性多步法
\begin{align*}
y_{n+3} + \alpha(y_{n+2} - y_{n+1}) - y_{n} = \frac{1}{2}(3 + \alpha)h\left[ f(x_{n+2},y_{n+2}) + f(x_{n+1},y_{n+1}) \right]
\end{align*}
是四阶的.
\paragraph{证明:}
以$t_{n+1}$为展开点,
\begin{align}
    R_{n+1}=& y(t+h)+(b-1)y(t)-by(t-h)-\frac{1}{4}h[(b+3)y'(t+h)+(3b+1)y'(t-h)]\\
    =&y+y'h+\frac{1}{2}y''h^2+\frac{1}{6}y'''h^3+\frac{1}{24}y''''h^4+O(h^5)\\
    &+(b-1)y\\
    &-b(y-y'h+\frac{1}{2}y''h^2-\frac{1}{6}y'''h^3+\frac{1}{24}y''''h^4+O(h^5))\\
    &-\frac{h}{4}[(b+3)(y'+y''h+\frac{1}{2}y'''h^2+\frac{1}{6}y''''h^3+O(h^4))\\
    &\quad\quad\quad+(3b+1)(y'-y''h+\frac{1}{2}y'''h^2-\frac{1}{6}y''''h^3+O(h^4))]\\
    =&(1+b-1-b)y+y'h(1+b-\frac{b+3}{4}-\frac{3b+1}{4})+y''h^2(\frac{1}{2}-\frac{1}{2}b-\frac{b+3}{4}+\frac{3b+1}{4})\\
    &+y'''h^3(\frac{1}{6}+\frac{b}{6}-\frac{b+3}{8}-\frac{3b+1}{8})+y''''h^4(\frac{1}{24}-\frac{b}{24}-\frac{b+3}{24}-\frac{3b+1}{24})+O(h^5)
\end{align}

可见$h^0,h^1,h^2$项恒为0,要求$h^3$项为0,有
\begin{align}
    \frac{1}{6}+\frac{b}{6}-\frac{b+3}{8}-\frac{3b+1}{8}=0
\end{align}
解得$b=-1$,此时$h^4$项也为0,因此该线性多步法在$b=-1$时是四阶的。






\subsubsection{17、线性多步法的收敛性分析}
讨论线性多步法
\begin{align*}
y_{n+3} + \frac{1}{4}y_{n+2} - \frac{1}{2}y_{n+1} - \frac{3}{4}y_n = \frac{1}{8}h\left[ 19f(x_{n+2},y_{n+2}) + 5f(x_n,y_n) \right]
\end{align*}
的收敛性.
\paragraph{证明:}

要让多步法收敛,只要求其满足相容性。

收敛性要求多步法满足强稳定性和相容性。

强稳定性:只要证特征方程的根全部在单位圆内,且模为1的根是单根。

特征方程:
\begin{align}
    \rho(x)=&x^3+\frac{1}{4}x^2-\frac{1}{2}x-\frac{3}{4}\\
    \sigma(x)=&\frac{19}{8}x^2+\frac{5}{8}
\end{align}

解得特征方程的根为
\begin{align}
    &x_1=1\\
    &x_2=\frac{-5-i\sqrt{23}}{8}\\
    &x_3=\frac{-5+i\sqrt{23}}{8}
\end{align}

三个根的模分别为
\begin{align}
    &|x-1|=1\\
    &|x_2|=|x_3|=\frac{48}{64}<1
\end{align}
故多步法是具有强稳定性的。

相容性:
\begin{align}
    \rho'(x)=3x^2+\frac{1}{2}x-\frac{1}{2}\neq \sigma(x)
\end{align}
故多步法不具有相容性,多步法不收敛。





\subsubsection{18、线性多步法的收敛性分析}
讨论线性多步法
\begin{align*}
y_{n+2} + y_{n+1} - 2y_{n} = \frac{1}{4}h\left[ f(x_{n+2},y_{n+2}) + 8f(x_{n+1},y_{n+1}) + 3f(x_n,y_n) \right]
\end{align*}
的收敛性
\paragraph{证明:}

\begin{align}
    \rho(x)=x^2+x-2\\
    \sigma(x)=\frac{1}{4}x^2+2x+\frac{3}{4}
\end{align}

强稳定性:特征多项式的根为
\begin{align}
    &x_1=1\\
    &x_2=-2
\end{align}
故不具有强稳定性。多步法不收敛。






\subsubsection{19、线性多步法的相容性分析}
讨论线性多步法
\begin{align}
    y_{n+2}-y_{n+1}=\frac{h}{2}[3f(x_{n+1},y_{n+1})-f(x_n,y_n)]
\end{align}
的相容性。
\paragraph{证明:}
在$x_{n+1}$处展开:
\begin{align}
    R_{n+1}=&y(x+h)-y-\frac{h}{2}[3y'-y'(x-h)]\\
    =&y+y'h+\frac{1}{2}y''h^2+\frac{1}{6}y'''h^3\\
    &-y\\
    &-\frac{h}{2}[3y'-(y'-y''h+\frac{1}{2}y'''h^2)]\\
    =&\frac{5}{12}h^3y'''+O(h^4)
\end{align}
故具有二阶相容性。






\subsubsection{20、A-稳定性}
试证明隐式Euler方法是A-稳定的.
\paragraph{证明:}

隐式Euler方法:
\begin{align}
    y_{n+1}=y_n+h f(x_{n+1},y_{n+1})
\end{align}
代入试验方程
\begin{align}
    y'=f=\lambda y 
\end{align}
得到 
\begin{align}
    y_{n+1}=&y_n+h\lambda y_{n+1}\\
    \Rightarrow & y_{n+1}=\frac{1}{1-h\lambda}y_n
\end{align}
故绝对稳定性要求
\begin{align}
    \bigg|\frac{1}{1-\mu}\bigg|<1
\end{align}
令$\mu=x+iy$,则有
\begin{align}
    (x-1)^2+y^2>1
\end{align}
只要实部$x<0$,则上式恒成立,因此隐式Euler方法是A-稳定的。








\chapter{圣遗物,考前必看}

% 第一份试题:2022-2023学年数值分析期末考试题

\section*{圣遗物1}

\begin{enumerate}
    \item (共10分) 对于线性代数方程组 \(Ax = b\),残量定义为 \(r = b - Ax\)。回答下述问题:
    \begin{enumerate}
        \item 写出Richardson迭代格式,分析格式的收敛性。
        \item 设 \(M\) 为可逆矩阵,写出以 \(M\) 为预处理矩阵的Richardson迭代格式。
        \item 设 \(A\) 为对称正定矩阵,选择预处理矩阵为数乘矩阵,即 \(M = \mu I\) (\(I\) 为单位矩阵,\(\mu\) 为标量)。则当(且仅当) \(\mu\) 满足何种条件时,此预处理方法是收敛的?
    \end{enumerate}
    \textbf{证明:}

    \item (共10分) 给定系数矩阵为对称正定矩阵的线性代数方程组 \(Ax = b\)。请回答下述问题:
    \begin{enumerate}
        \item 定义函数
        \[
        \varphi(x) = \frac{1}{2}(Ax, x) - (b, x),
        \]
        其中内积 \((x, y)\) 为标准的欧式内积。证明在全空间中存在唯一的临界点;该临界点为全局最小值点,且恰好是 \(Ax = b\) 的根。
        \item 设当前近似向量为 \(x\)。写出从 \(x\) 出发,平行于方向p做一维极小搜索得到的新近似向量的表达式。
        \item 证明新近似向量的残量与搜索方向p是正交的(关于欧式内积)。
    \end{enumerate}
    \textbf{证明:}

    \item (共10分) 设 \(A\) 是一个实对称矩阵,则其特征值和特征向量都是实的。若特征向量满足 \(\|x\|_2 = 1\),则称 \(x\) 为归一化特征向量。请回答下述问题:
    \begin{enumerate}
        \item 写出Newton法计算特征值及其对应的归一化特征向量的计算格式。
        \item 证明当特征值是单特征值时,上述计算方法是局部二阶收敛的。
    \end{enumerate}

    \textbf{解:}


    \item (共10分) 设实矩阵序列 \(\{X_k\}\) 满足
    \[
    X_{k+1} = X_k + X_k(I - AX_k), \quad k = 0, 1, \cdots.
    \]
    证明如下结论:
    \begin{enumerate}
        \item 令 \(E_k = I - AX_k\),则 \(E_{k+1} = E_k^2\)。
        \item \(\{X_k\}\) 至少局部二阶收敛到 \(A^{-1}\)。
    \end{enumerate}
    \textbf{证明:}



    \item (共10分) 考虑常微分方程组 \(y' = f(t, y)\) 的初值问题。请回答下述问题:
    \begin{enumerate}
        \item 写出一般步长情况下的显式Euler方法、隐式Euler方法以及梯形方法的计算格式。
        \item 分析梯形方法的相容性,写出该方法的主局部截断误差。
    \end{enumerate}
    \textbf{证明:}



    \item (共10分) 考虑常微分方程组 \(y' = f(t, y)\) 的初值问题。用如下的等步长Runge-Kutta方法
    \[
    y_{n+1} = y_n + hK,
    \]
    其中 \(K\) 满足
    \[
    K = f(t_n + h/2, y_n + hK/2).
    \]
    请回答下述问题:
    \begin{enumerate}
        \item 证明该方法恰好是二阶收敛的(需要说明其不是三阶方法)。
        \item 确定该方法的绝对稳定区域。
    \end{enumerate}
\end{enumerate}
\textbf{证明:}



\newpage

\section*{圣遗物2}

\begin{enumerate}
    \item 证明:
        设得到$Ax=b$的一个近似解$\bar{x}$,其残量为 
        \begin{align}
            r=b-A\bar{x}
        \end{align}
        设$x$和$\bar{x}$分别是准确解和近似解,证明:
        \begin{align}
            \frac{1}{\mathrm{cond}(A)}\frac{\|r\|}{\|b\|}\leq \frac{\|x-\bar{x}\|}{\|x\|}\leq \mathrm{cond}(A)\frac{\|r\|}{\|b\|}
        \end{align}
        \textbf{证明:}



    \item 矩阵A不可分且弱对角占优,证明:
    \begin{enumerate}
        \item A对角元素不等于0
        \item 矩阵A非奇异
        \item Ax=b的Jacobi迭代收敛
    \end{enumerate}
    \textbf{证明:}




    \item 有以下迭代格式:
    \[
    x_{k + 1} = \frac{x_{k}(3 + x_{k}^{2})}{3x_{k}^{2} + 1}
    \]
    \begin{enumerate}
        \item 求该迭代的不动点
        \item 分析各不动点的局部收敛性和收敛阶
        \item 求各不动点的收敛域
    \end{enumerate}
    \textbf{证明:}




    \item 经典再放送\\
    (共10分)设A是一个实对称矩阵,则其特征值和特征向量都是实的.若特征向量x满足 \(\| x\|_{2} = 1\) ,则称x为归一化特征向量.请回答下述问题:
    \begin{enumerate}
        \item 写出Newton法计算特征值及其对应的归一化特征向量的计算格式
        \item 证明当特征值是单特征值时,上述计算方法是局部二阶收敛的
    \end{enumerate}
    \textbf{证明:}




    \item 微分方程: \(y^{\prime} = f(t,y)\) ,迭代式:
    \[
    y_{n + 1} = y_{n} + h\big(\theta f(t_{n} + y_{n}) + (1 - \theta)f(t_{n + 1},y_{n + 1})\big), \quad \theta \in [0,1]
    \]
    \begin{enumerate}
        \item 该迭代的相容阶
        \item 绝对稳定区间
    \end{enumerate}
\end{enumerate}
\textbf{证明:}





\chapter{圣遗物答案}

\section*{圣遗物1}

\begin{enumerate}
    \item (共10分) 对于线性代数方程组 \(Ax = b\),残量定义为 \(r = b - Ax\)。回答下述问题:
    \begin{enumerate}
        \item 写出Richardson迭代格式,分析格式的收敛性。
        \item 设 \(M\) 为可逆矩阵,写出以 \(M\) 为预处理矩阵的Richardson迭代格式。
        \item 设 \(A\) 为对称正定矩阵,选择预处理矩阵为数乘矩阵,即 \(M = \mu I\) (\(I\) 为单位矩阵,\(\mu\) 为标量)。则当(且仅当) \(\mu\) 满足何种条件时,此预处理方法是收敛的?
    \end{enumerate}
    \textbf{证明:}

    \item (共10分) 给定系数矩阵为对称正定矩阵的线性代数方程组 \(Ax = b\)。请回答下述问题:
    \begin{enumerate}
        \item 定义函数
        \[
        \varphi(x) = \frac{1}{2}(Ax, x) - (b, x),
        \]
        其中内积 \((x, y)\) 为标准的欧式内积。证明在全空间中存在唯一的临界点;该临界点为全局最小值点,且恰好是 \(Ax = b\) 的根。
        \item 设当前近似向量为 \(x\)。写出从 \(x\) 出发,平行于方向p做一维极小搜索得到的新近似向量的表达式。
        \item 证明新近似向量的残量与搜索方向p是正交的(关于欧式内积)。
    \end{enumerate}
    \textbf{证明:}

    \item (共10分) 设 \(A\) 是一个实对称矩阵,则其特征值和特征向量都是实的。若特征向量满足 \(\|x\|_2 = 1\),则称 \(x\) 为归一化特征向量。请回答下述问题:
    \begin{enumerate}
        \item 写出Newton法计算特征值及其对应的归一化特征向量的计算格式。
        \item 证明当特征值是单特征值时,上述计算方法是局部二阶收敛的。
    \end{enumerate}

    \textbf{解:}


\noindent 设 $A \in \mathbb{R}^{n \times n}$ 是一个实对称矩阵。我们需要寻找特征对 $(\lambda, x)$,使得 $Ax = \lambda x$ 且满足归一化条件 $\|x\|_2 = 1$。

\vspace{1em}
\noindent \textit{(1)使用 Newton 法计算特征值及其归一化特征向量的计算格式}

\vspace{0.5em}
\noindent 为了使用 Newton 法,定义非线性映射 $F: \mathbb{R}^{n+1} \to \mathbb{R}^{n+1}$ 如下:
\begin{equation*}
F(z) = F(x, \lambda) = 
\begin{pmatrix} 
Ax - \lambda x \\ 
\frac{1}{2}(x^T x - 1) 
\end{pmatrix} = 0
\end{equation*}
其中 $z = [x^T, \lambda]^T$。其相应的 Jacobian 矩阵 $J(x, \lambda)$ 为:
\begin{equation*}
J(x, \lambda) = 
\begin{pmatrix} 
A - \lambda I & -x \\ 
x^T & 0 
\end{pmatrix}
\end{equation*}

\noindent 给定初始近似值 $x_0$ 和 $\lambda_0$,第 $k$ 步迭代格式为:
\begin{enumerate}
    \item 解线性方程组得到增量 $\Delta z_k = [\Delta x_k^T, \Delta \lambda_k]^T$:
    \begin{equation*}
    \begin{pmatrix} 
    A - \lambda_k I & -x_k \\ 
    x_k^T & 0 
    \end{pmatrix} 
    \begin{pmatrix} 
    \Delta x_k \\ 
    \Delta \lambda_k 
    \end{pmatrix} = 
    - \begin{pmatrix} 
    Ax_k - \lambda_k x_k \\ 
    \frac{1}{2}(x_k^T x_k - 1) 
    \end{pmatrix}
    \end{equation*}
    \item 更新解:$x_{k+1} = x_k + \Delta x_k$,$\lambda_{k+1} = \lambda_k + \Delta \lambda_k$。
\end{enumerate}

\vspace{1em}
\noindent \textit{(2)证明当特征值是单特征值时,上述计算方法是局部二阶收敛的}

\vspace{0.5em}
\begin{proof}
根据 Newton 法的性质,若 $J(z^*)$ 在解点处非奇异,则具有局部二阶收敛性。考虑齐次方程 $J(z^*) [u^T, v]^T = 0$:
\begin{align}
(A - \lambda^* I)u - v x^* &= 0 \label{eq1} \\
(x^*)^T u &= 0 \label{eq2}
\end{align}

将式 \eqref{eq1} 左乘 $(x^*)^T$,利用 $A$ 的对称性及 $(A - \lambda^* I)x^* = 0$,得:
\begin{equation*}
(x^*)^T (A - \lambda^* I)u - v (x^*)^T x^* = 0 \implies 0 - v \cdot 1 = 0 \implies v = 0
\end{equation*}

将 $v=0$ 代入式 \eqref{eq1} 得 $(A - \lambda^* I)u = 0$。由于 $\lambda^*$ 是单特征值,其特征空间维数为 1,故 $u = \alpha x^*$。代入式 \eqref{eq2} 有:
\begin{equation*}
(x^*)^T (\alpha x^*) = \alpha \|x^*\|_2^2 = \alpha = 0 \implies u = 0
\end{equation*}

由于齐次方程只有零解,Jacobian 矩阵 $J(z^*)$ 非奇异。因此,该方法在单特征值附近是局部二阶收敛的。
\end{proof}
    

    \item (共10分) 设实矩阵序列 \(\{X_k\}\) 满足
    \[
    X_{k+1} = X_k + X_k(I - AX_k), \quad k = 0, 1, \cdots.
    \]
    证明如下结论:
    \begin{enumerate}
        \item 令 \(E_k = I - AX_k\),则 \(E_{k+1} = E_k^2\)。
        \item \(\{X_k\}\) 至少局部二阶收敛到 \(A^{-1}\)。
    \end{enumerate}
    \textbf{证明:}



    \item (共10分) 考虑常微分方程组 \(y' = f(t, y)\) 的初值问题。请回答下述问题:
    \begin{enumerate}
        \item 写出一般步长情况下的显式Euler方法、隐式Euler方法以及梯形方法的计算格式。
        \item 分析梯形方法的相容性,写出该方法的主局部截断误差。
    \end{enumerate}
    \textbf{证明:}



    \item (共10分) 考虑常微分方程组 \(y' = f(t, y)\) 的初值问题。用如下的等步长Runge-Kutta方法
    \[
    y_{n+1} = y_n + hK,
    \]
    其中 \(K\) 满足
    \[
    K = f(t_n + h/2, y_n + hK/2).
    \]
    请回答下述问题:
    \begin{enumerate}
        \item 证明该方法恰好是二阶收敛的(需要说明其不是三阶方法)。
        \item 确定该方法的绝对稳定区域。
    \end{enumerate}
\end{enumerate}
\textbf{证明:}



\newpage

\section*{圣遗物2}

\begin{enumerate}
    \item 证明:
        设得到$Ax=b$的一个近似解$\bar{x}$,其残量为 
        \begin{align}
            r=b-A\bar{x}
        \end{align}
        设$x$和$\bar{x}$分别是准确解和近似解,证明:
        \begin{align}
            \frac{1}{\mathrm{cond}(A)}\frac{\|r\|}{\|b\|}\leq \frac{\|x-\bar{x}\|}{\|x\|}\leq \mathrm{cond}(A)\frac{\|r\|}{\|b\|}
        \end{align}
        \textbf{证明:}



    \item 矩阵A不可分且弱对角占优,证明:
    \begin{enumerate}
        \item A对角元素不等于0
        \item 矩阵A非奇异
        \item Ax=b的Jacobi迭代收敛
    \end{enumerate}
    \textbf{证明:}




    \item 有以下迭代格式:
    \[
    x_{k + 1} = \frac{x_{k}(3 + x_{k}^{2})}{3x_{k}^{2} + 1}
    \]
    \begin{enumerate}
        \item 求该迭代的不动点
        \item 分析各不动点的局部收敛性和收敛阶
        \item 求各不动点的收敛域
    \end{enumerate}
    \textbf{证明:}




    \item 经典再放送\\
    (共10分)设A是一个实对称矩阵,则其特征值和特征向量都是实的.若特征向量x满足 \(\| x\|_{2} = 1\) ,则称x为归一化特征向量.请回答下述问题:
    \begin{enumerate}
        \item 写出Newton法计算特征值及其对应的归一化特征向量的计算格式
        \item 证明当特征值是单特征值时,上述计算方法是局部二阶收敛的
    \end{enumerate}
    \textbf{证明:}




    \item 微分方程: \(y^{\prime} = f(t,y)\) ,迭代式:
    \[
    y_{n + 1} = y_{n} + h\big(\theta f(t_{n} + y_{n}) + (1 - \theta)f(t_{n + 1},y_{n + 1})\big), \quad \theta \in [0,1]
    \]
    \begin{enumerate}
        \item 该迭代的相容阶
        \item 绝对稳定区间
    \end{enumerate}
\end{enumerate}
\textbf{证明:}









\end{document}