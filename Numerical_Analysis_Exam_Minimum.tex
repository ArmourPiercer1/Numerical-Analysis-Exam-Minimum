\documentclass{book}

% 中文字体支持
\usepackage{ctex}
\usepackage{xeCJK}

% 标题字号设置(使用 titlesec,依次减小字号)
\usepackage{titlesec}
\titleformat{\chapter}{\zihao{-2}\bfseries}{\thechapter}{1em}{}
\titleformat{\section}{\zihao{3}\bfseries}{\thesection}{0.8em}{}
\titleformat{\subsection}{\zihao{4}\bfseries}{\thesubsection}{0.7em}{}
\titleformat{\subsubsection}{\zihao{-4}\bfseries}{\thesubsubsection}{0.6em}{}
\titleformat{\paragraph}{\zihao{5}\bfseries}{\theparagraph}{0.5em}{}
% 各级标题编号格式
% chapter: 一 二 三 ...
\renewcommand{\thechapter}{\zhnum{chapter}}
% section: (一)(二)(三)...
\renewcommand{\thesection}{(\zhnum{section})}
% subsection: 1. 2. 3. ...
\renewcommand{\thesubsection}{\arabic{subsection}.}
% subsubsection: (1)(2)(3)...
\renewcommand{\thesubsubsection}{(\arabic{subsubsection})}
% paragraph: a. b. c. ...
\renewcommand{\theparagraph}{\alph{paragraph}.}
% 允许编号到 paragraph 层级
\setcounter{secnumdepth}{4}
% 数学公式支持
\usepackage{amsmath,amssymb,amsthm}
\usepackage{mathtools}
\usepackage{bm} % 粗体数学符号

% 取消所有公式编号:通过修改计数器深度实现
\usepackage{etoolbox}
\AtBeginEnvironment{equation}{\nonumber}
\AtBeginEnvironment{equation*}{\nonumber}
\AtBeginEnvironment{align}{\nonumber}
\AtBeginEnvironment{align*}{\nonumber}
\AtBeginEnvironment{gather}{\nonumber}
\AtBeginEnvironment{gather*}{\nonumber}
\AtBeginEnvironment{multline}{\nonumber}
\AtBeginEnvironment{multline*}{\nonumber}

% 图表支持
\usepackage{graphicx}
\usepackage{tikz}
\usepackage{pgfplots}
\pgfplotsset{compat=1.17}
\usepackage{float}

% 页面布局
\usepackage{geometry}
\geometry{left=2.5cm,right=2.5cm,top=3cm,bottom=3cm}
\usepackage{fancyhdr}
\usepackage{lastpage}
\usepackage{mathtools}

% 目录和超链接
\usepackage{hyperref}
\hypersetup{
    colorlinks=true,
    linkcolor=blue,
    filecolor=magenta,      
    urlcolor=cyan,
    citecolor=red
}

% 代码支持
\usepackage{listings}
\usepackage{xcolor}

% 表格支持
\usepackage{booktabs}
\usepackage{multirow}
\usepackage{array}

% 特殊形状支持
\usepackage{xcolor}

% % 副标题
% \usepackage{titling}

% 定理环境(取消编号)
\newtheorem*{theorem}{定理}
\newtheorem*{lemma}{引理}
\newtheorem*{corollary}{推论}
\newtheorem*{proposition}{命题}
\newtheorem*{definition}{定义}
\newtheorem*{example}{例题}
\newtheorem*{exercise}{练习}
\newtheorem*{remark}{备注}
\newtheorem*{proofstep}{说明}
\newtheorem*{property}{性质}




% 待填写内容标记
\NewDocumentCommand{\tofill}{O{} m}{
    \textbf{\textit{\textcolor{red}{待填写:\IfValueT{#1}{(#1)}#2}}}
}

% 页眉页脚设置
\setlength{\headheight}{14.5pt}
\pagestyle{fancy}
\fancyhf{}
\fancyhead[L]{数值分析课程总结}
\fancyhead[R]{\today}
\fancyfoot[C]{\thepage/\pageref{LastPage}}

% 代码样式设置
\lstset{
    basicstyle=\ttfamily\small,
    keywordstyle=\color{blue}\bfseries,
    commentstyle=\color{green!60!black},
    stringstyle=\color{red},
    showstringspaces=false,
    numbers=left,
    numberstyle=\tiny\color{gray},
    frame=single,
    breaklines=true
}

% 文档信息
\title{Numerical Analysis \\ Exam Minimum}
\author{Astral Projection, Yiuko}
\date{\today}


\begin{document}

\maketitle
\tableofcontents
\newpage

\chapter {数学基础知识}

\section {核心概念与理论}

\subsection {线性空间}
\subsubsection {定义与性质}


\begin{definition}[线性空间]
    设S是一个集合,P是一个数域($\mathbb{R}$或$\mathbb{C}$). 定义两种映射关系:
    \begin{itemize}
        \item 向量加法:$+: S \times S \to S$
        \item 数乘:$\cdot : P \times S \to S$
    \end{itemize}
    如果对任意的$u,v,w \in S$和$a,b \in P$,满足以下八条公理,则称(S,P)为一个线性空间(向量空间):
    \begin{enumerate}
        \item 加法交换律:$u + v = v + u$
        \item 加法结合律:$(u + v) + w = u + (v + w)$
        \item 存在加法单位元:存在零向量$0 \in S$,使得对任意$v \in S$,有$v + 0 = v$
        \item 存在加法逆元:对任意$v \in S$,存在$-v \in S$,使得$v + (-v) = 0$
        \item 数乘结合律:$a(bv) = (ab)v$
        \item 数乘分配律1:$a(u + v) = au + av$
        \item 数乘分配律2:$(a + b)v = av + bv$
        \item 数乘单位元:$1v = v$
        \end{enumerate}
        则称(S,P)构成一个线性空间。
\end{definition}
此外,如果对于给定空间的运算法则和数域是不言自明的,则通常简写为S是一个线性空间。
如我们说$\mathbb{R}^n$是一个线性空间,通常指$(\mathbb{R}^n,\mathbb{R})$是一个线性空间或$(\mathbb{R}^n,\mathbb{C})$是一个线性空间,具体取决于数域的选择。

\subsubsection {线性无关与相关}
\tofill[定义]{线性无关与线性相关}

\subsubsection {基、框架与维数}

\tofill[定义]{基、框架与维数}

性质:
\begin{itemize}
    \item 空间的维度是一个内蕴量,与基的选择无关
    \item 多项式空间$P_N$中,$\{1,x,x^2,\ldots,x^N\}$构成其一组基,维数为$\dim P_N=N+1$
    \item 连续函数空间$C[a,b]$中,$\forall N$,$\{1,x,x^2,\ldots,x^N\}$是线性无关的,但不能构成其基,因其维数为无穷大
\end{itemize}

\subsection {度量与赋范空间}
\subsubsection {距离空间}
\begin{definition}[距离空间]
    设M是一个集合,$d: M \times M \to \mathbb{R}$是一个映射,如果对任意的$x,y,z \in M$,满足以下三条公理,则称(M,d)为一个距离空间:
    \begin{enumerate}
        \item 非负性与分离性:$d(x,y) \geq 0$,且当且仅当$x=y$时,$d(x,y)=0$
        \item 对称性:$d(x,y) = d(y,x)$
        \item 三角不等式:$d(x,z) \leq d(x,y) + d(y,z)$
    \end{enumerate}
    则称(M,d)构成一个距离空间。
\end{definition}

\subsubsection{距离空间的完备性}

\tofill[定义]{完备性}

$\mathbb{R}$是完备的,且任意有限维赋范空间都是完备的。

\paragraph{构造方法:距离空间的完备化}
设$(M,d)$是一个距离空间,可以按照如下过程构造其完备化空间:
\begin{enumerate}
    \item 构造对偶的柯西列空间
    \begin{align}
        \tilde{M}={\{(x_n)\,|\,x_n \in M,\quad (x_n) \text{为柯西列}\}}
    \end{align}
    \item 在柯西列空间$\tilde{M}$中定义等价关系
    \begin{align}
        \tilde{x}\sim\tilde{y}\leftrightarrow \lim_{n\to\infty} d(x_n,y_n)=0
    \end{align}
    即这两个柯西列按照角标顺序,交叉放在一起,还是柯西列。
    \item 构造商空间:
    \begin{align}
        \hat{M}=\tilde{M}/\sim=\{ [\tilde{x}] \}
    \end{align}
    式中,$[\tilde{x}]$表示柯西列$\tilde{x}$的等价类,即$[\tilde{x}]$是一个集合,
    集合中的所有元素在等价关系$\sim$下都是等价的。
    \item 在商空间$\hat{M}$中定义距离
    \begin{align}
        \hat{d}([\tilde{x}],[\tilde{y}])=\lim_{n\to\infty} d(x_n,y_n)
    \end{align}
    \item 则$(\hat{M},\hat{d})$即为距离空间$(M,d)$的完备化空间。
\end{enumerate}

嵌入映射:可以在原空间$M$与完备化空间$\hat{M}$之间定义一个单射$i$:
\begin{align}
    i: M \to \hat{M},\quad i(x)=[(x,x,x,\ldots)]
\end{align}
该映射将原空间中的每个点$x$映射为完备化空间中由常值序列$(x,x,x,\ldots)$所构成的等价类,
且\textbf{映射前后任意两元素的距离不变}





\subsubsection {赋范空间与 Banach 空间}
\tofill[定义]{赋范空间} 

完备的赋范空间称为Banach空间,或者B空间。


\subsubsection {等价范数}
\begin{definition}[范数的等价性]
    设$V$是一个线性空间,$\|\cdot\|_1$和$\|\cdot\|_2$是$V$上的两个范数,如果存在正常数$c$和$C$,使得对任意$v \in V$,都有
    \begin{align}
        c\|v\|_1 \leq \|v\|_2 \leq C\|v\|_1
    \end{align}
    则称这两个范数是等价的。

    若存在正数C,使得对任意$v \in V$,都有
    \begin{align}
        \|v\|_2 \leq C\|v\|_1
    \end{align}
    则称范数$\|\cdot\|_1$强于$\|\cdot\|_2$。
\end{definition}

性质:
\begin{itemize}
    \item 在有限维线性空间上,任意两个范数都是等价的
    \item 在无限维线性空间上,范数不一定是等价的
    \item 若一个点列在较强的范数下是Cauchy列,则在较弱的范数下也是Cauchy列;反之不必然。
\end{itemize}

\subsubsection{常用的范数}
\paragraph{$\mathbb{R}^n$上的范数} 记$x=(x_1,x_2,\ldots,x_n)^T \in \mathbb{R}^n$,则常用的范数有:
\begin{itemize}
    \item 无穷范数:所有元素的最大值
    \begin{align}
        \|x\|_{\infty}=\max_{1 \leq i \leq n} |x_i|
    \end{align}
    \item 1-范数:所有元素的绝对值之和
    \begin{align}
        \|x\|_{1}=\sum_{i=1}^{n} |x_i|
    \end{align}
    \item 2-范数:欧几里得范数,即所有元素的平方和的平方根
    \begin{align}
        \|x\|_{2}=\left(\sum_{i=1}^{n} |x_i|^2\right)^{\frac{1}{2}}
    \end{align}
\end{itemize}

\paragraph{$C[a,b]$(有界闭区间上连续函数空间)上的范数}
\begin{itemize}
    \item 无穷范数:函数在区间上的最大绝对值
    \begin{align}
        \|f\|_{\infty}=\max_{a \leq x \leq b} |f(x)|
    \end{align}
    \item 1-范数:函数在区间上的绝对值积分
    \begin{align}
        \|f\|_{1}=\int_{a}^{b} |f(x)| \, dx
    \end{align}
    \item 2-范数:函数在区间上的平方积分的平方根
    \begin{align}
        \|f\|_{2}=\left(\int_{a}^{b} |f(x)|^2 \, dx\right)^{\frac{1}{2}}
    \end{align}
\end{itemize}

$C[a,b]$上三个范数的性质:

\begin{itemize}
    \item 任意两个范数不等价
    \item 无穷范数强于2范数,2范数强于1范数
    \item 只有无穷范数对应的赋范空间是完备的
    \item 1-范数对应的完备化空间为$L^1(a,b)$,2-范数对应的完备化空间为$L^2(a,b)$ 
        \footnote{$L^1(a,b)$为(a,b)上的可积函数空间,$L^2(a,b)$为(a,b)上的平方可积函数空间。}
\end{itemize}

\subsection {内积空间}
\subsubsection {内积定义与性质}
\begin{definition}[内积]
    设$(S,P)$是一个线性空间,如果对任意的$u,v,w \in S$和$a,b \in P$,存在一个映射$S\times S\rightarrow P$,满足 
    \begin{enumerate}
        \item 共轭对称性:$\langle u,v \rangle = \overline{\langle v,u \rangle}$
        \item 线性性:$\langle au + bv,w \rangle = a\langle u,w \rangle + b\langle v,w \rangle$
        \item 正定性:$\langle v,v \rangle \geq 0$,且当且仅当$v=0$时,$\langle v,v \rangle = 0$
    \end{enumerate}
    则称该映射为内积,$(S,P)$构成一个内积空间。
\end{definition}

若$\langle x,y \rangle =0$,则称$x$与$y$正交。

几个常用空间上的内积:
\begin{itemize}
    \item $\mathbb{R}^n$或$\mathbb{C}^n$上的内积
    \begin{align}
        \langle x,y \rangle = \sum_{i=1}^{n} x_i \overline{y_i}
    \end{align}
    \item $C[a,b]$(有界闭区间上连续函数空间)上的内积
    \begin{align}
        \langle f,g \rangle = \int_{a}^{b} f(x) \overline{g(x)} \, dx
    \end{align}
    \item $C[a,b]$上的带权内积:
    \begin{align}
        (f,g)=\int_a^b \rho(x) f(x) \overline{g(x)} \, dx
    \end{align}
    其中,权函数$\rho(x)$需要满足条件:
    \begin{itemize}
        \item $\rho(x)\in C[a,b]$
        \item $\rho(x)$几乎处处为正
        \item $\int_a^b \rho(x) \, dx < +\infty$
        \item $\forall q(x) \in P_n$,$\int_a^b \rho(x) |q(x)|\mathrm{d}x < \infty$
    \end{itemize}
    带权内积所研究的空间称为加权内积空间:
    \begin{align}
        L^2_{\rho}(a,b) = \{ f(x) \,|\, \int_a^b \rho(x) |f(x)|^2 \, dx < +\infty \}
    \end{align}
\end{itemize}

常用的权函数有:
\begin{align}
    \rho(x) = 1, & \quad [a,b] = [-1,1]\\
    \rho(x)=\frac{1}{1-x^2}, & \quad [a,b] = [-1,1]\\
\end{align}


\subsubsection {正交性与 Schmidt 正交化}
\tofill[定义]{正交性}

\tofill[方法]{用Grammer矩阵判断内积空间中向量组的线性无关性}

\tofill[方法]{Schmidt 正交化过程:从一个线性无关向量组构造一个正交向量组:让每个向量减去与已有空间垂直的分量}

用Schmidt正交化过程得到的正交向量组具有以下性质:
\begin{align}
    \Phi_{k-1} \subset \Phi_k\\
    y_k \perp \Phi_{k-1}
\end{align}

\subsubsection {由内积诱导的范数}
\begin{definition}[诱导范数]
    设$(S,P)$是一个内积空间,则可以定义范数$\|\cdot\|$如下:
    \begin{align}
        \|v\|=\sqrt{\langle v,v \rangle}
    \end{align}
    则称该范数为由内积诱导的范数。
\end{definition}

\begin{itemize}
    \item 任何内积均能诱导对应的范数
    \item 当且仅当范数满足平行四边形法则时
    \begin{align}
        \|f+g\|^2 + \|f-g\|^2 = 2\|f\|^2 + 2\|g\|^2
    \end{align}
    范数可以诱导内积:
    \begin{align}
        (x,y)=\frac{1}{4}(\|x+y\|^2 - \|x-y\|^2 + i\|x+iy\|^2 - i\|x-iy\|^2)
    \end{align}
\end{itemize}


\subsection {正交多项式}
\begin{definition}
[正交多项式]
    设$\{\phi_n(x)\}$是定义在区间$[a,b]$上的一组多项式,且每个多项式的次数为$n$,如果对任意$m \neq n$,都有
    \begin{align}
        \int_a^b \rho(x) \phi_m(x) \phi_n(x) \, dx = 0
    \end{align}
    则称$\{\phi_n(x)\}$为区间$[a,b]$上关于权函数$\rho(x)$的正交多项式。
\end{definition}

正交多项式的性质:
\begin{itemize}%[]
    \item $\deg \phi_i=i$
    \item $(\phi_i,\,\phi_j)=0,\quad \forall i\neq j$
    \item $\phi_n$为实系数多项式
    \item $\phi_n$在开区间$(a,b)$内恰有n个实单根
\end{itemize}

\tofill[证明]{$\phi_n$在开区间$(a,b)$内恰有n个实单根的证明。证法:分别证明实根、单根、全在$(a,b)$内。3个命题均可用反证法。}


\subsubsection {$\rho=1$:Legendre 多项式}
产生方法:
\begin{itemize}
    \item 权函数$\rho(x)=1$
    \item 区间$[-1,1]$
\end{itemize}

表达式:
\begin{align}
    P_n(x)=\frac{1}{2^nn!}\frac{\mathrm{d}^n}{\mathrm{d}x^n}[(x^2-1)^n]
\end{align}

性质:
\begin{itemize}
    \item 首项系数:
    \begin{align}
        k_n=\frac{(2n)!}{2^n(n!)^2}
    \end{align}
    \item 正交归一化:
    \begin{align}
        \int_{-1}^{1} P_n(x)P_m(x) \, dx = \frac{2}{2n+1} \delta_{mn}
    \end{align}
    \item 三项递推关系:
    \begin{align}
        (n+1)P_{n+1}(x)=(2n+1)xP_n(x)-nP_{n-1}(x)
    \end{align}
    \item 奇偶性:
    \begin{align}
        P_n(-x)=(-1)^n P_n(x)
    \end{align}
    \item 导数关系:
    \begin{align}
        \frac{\mathrm{d}}{\mathrm{d}x} P_n(x) = \frac{n}{x^2-1} [xP_n(x) - P_{n-1}(x)]
    \end{align}
    \item 前五项:
    \begin{align}
        &P_0(x)=1\\
        &P_1(x)=x\\
        &P_2(x)=\frac{1}{2}(3x^2-1)\\
        &P_3(x)=\frac{1}{2}(5x^3-3x)\\
        &P_4(x)=\frac{1}{8}(35x^4-30x^2+3)
    \end{align}
    \item 零平方误差最小:
    \begin{theorem}
        在所有首项为1的n次多项式中,Legendre多项式$\tilde{P}_n(x)$在$[-1,1]$上与零的平方误差最小。
    \end{theorem}
\end{itemize}



\subsubsection {$\rho(x)=\frac{1}{\sqrt{1-x^2}}$:Chebyshev 多项式}
产生方法:
\begin{itemize}
    \item 权函数$\rho(x)=\frac{1}{\sqrt{1-x^2}}$
    \item 区间$[-1,1]$
\end{itemize}

表达式:
\begin{align}
    T_n(x)=\cos(n \arccos x)
\end{align}

性质:
\begin{itemize}
    \item 首项系数:$2^{n-1}$
    \item 正交归一化:
    \begin{align}
        \int_{-1}^{1} \frac{T_n(x)T_m(x)}{\sqrt{1-x^2}} \, dx = \begin{cases}
            \pi, & n=m=0\\
            \frac{\pi}{2}, & n=m\neq 0\\
            0, & n\neq m
        \end{cases}
    \end{align}
    \item 三项递推关系:
    \begin{align}
        T_{n+1}=2xT_n-T_{n-1}
    \end{align}
    \item 奇偶性:
    \begin{align}
        T_n(-x)=(-1)^n T_n(x)
    \end{align}
    \item 前五项:
    \begin{align}
        &T_0(x)=1\\
        &T_1(x)=x\\
        &T_2(x)=2x^2-1\\
        &T_3(x)=4x^3-3x\\
        &T_4(x)=8x^4-8x^2+1
    \end{align}
    \item 零点:
    \begin{align}
        x_k=\cos\left(\frac{2k-1}{2n}\pi\right),\quad k=1,2,\ldots,n
    \end{align}
    \item 极值点:
    \begin{align}
        x_k=\cos\left(\frac{k\pi}{n}\right),\quad k=0,1,\ldots,n
    \end{align}
    \item 简单表达式:当$|x|\geq 1$时,
    \begin{align}
        T_n(x)=\frac{1}{2}\left[ \left( x + \sqrt{x^2-1} \right)^n + \left( x - \sqrt{x^2-1} \right)^n \right]
    \end{align}
\end{itemize}



\subsection{矩阵分析回顾}

\subsubsection{矩阵代数与初等变换 (Elementary Transformations)}
\textbf{初等矩阵}:由单位矩阵 $I$ 经过\textbf{一次}初等变换得到的矩阵。
\begin{enumerate}
    \item \textbf{交换矩阵 (Permutation Matrix)} $P_{ij}$:
    交换 $I$ 的第 $i$ 行和第 $j$ 行。
    \begin{itemize}
        \item \textbf{作用}:左乘交换矩阵的行,右乘交换矩阵的列。
        \item \textbf{性质}:$P_{ij}^{-1} = P_{ij}^T = P_{ij}$(对称且正交)。
        \item \textbf{应用}:Gauss 消去法中的选主元 ($PA=LU$)。
    \end{itemize}
    
    \item \textbf{倍乘矩阵 (Scaling Matrix)} $D_i(k)$:
    将 $I$ 的第 $i$ 行乘以非零常数 $k$。
    \begin{itemize}
        \item \textbf{性质}:$D_i(k)$ 是对角阵。$D_i(k)^{-1} = D_i(1/k)$。
    \end{itemize}
    
    \item \textbf{倍加矩阵 (Elimination/Shear Matrix)} $E_{ij}(k)$ ($i \ne j$):
    将 $I$ 的第 $j$ 行的 $k$ 倍加到第 $i$ 行。
    \[
        E_{ij}(k) = I + k e_i e_j^T
    \]
    \begin{itemize}
        \item \textbf{作用}:左乘 $A$ 将第 $j$ 行的 $k$ 倍加到第 $i$ 行(消元操作)。
        \item \textbf{性质}:
        1. 逆矩阵形式简单:$(E_{ij}(k))^{-1} = E_{ij}(-k)$。
        2. Gauss 消去法中的消元矩阵 $L_k$ 就是一系列倍加矩阵的乘积。
    \end{itemize}
\end{enumerate}

\subsubsection{矩阵特征值理论}
\begin{itemize}
    \item \textbf{特征分解}:$Ax=\lambda x$。
    \item \textbf{SVD分解}:$A=U\Sigma V^H$。
    \item \textbf{Schur分解}:$U^H A U = T$。
\end{itemize}

\subsubsection{特殊矩阵类}
\begin{itemize}
    \item \textbf{Hermite 矩阵 (实对称矩阵)}:$A^H = A$。
        \begin{itemize}
            \item 特征值全为实数。
            \item 存在酉矩阵 $U$ 使得 $U^H AU = \Lambda$ (可酉对角化)。
            \item \textbf{Rayleigh 商}:$\lambda_{\min} \le \frac{x^H Ax}{x^H x} \le \lambda_{\max}$。
        \end{itemize}
        
    \item \textbf{对称正定矩阵 (Symmetric Positive Definite, SPD)}:
        \begin{definition}[SPD 定义]
            实对称矩阵 $A$ 称为正定的,如果对任意非零向量 $x \in \mathbb{R}^n, x \ne 0$,都有:
            \[ x^T A x > 0 \]
            这定义了一个\textbf{能量范数}(Energy Norm):$||x||_A = \sqrt{x^T A x}$。
        \end{definition}

        性质:
            \begin{enumerate}
                \item \textbf{特征值}:所有特征值均严格大于 0 ($\lambda_i > 0$)。
                \item \textbf{行列式}:$\det(A) = \prod \lambda_i > 0$,故 $A$ 必可逆。
                \item \textbf{主子式}:所有顺序主子式均大于 0(Sylvester 准则)。
                \item \textbf{对角元}:$a_{ii} > 0$,且 $\max_{i,j} |a_{ij}| = \max_i a_{ii}$(最大元素必在对角线上)。
                \item \textbf{逆矩阵}:若 $A$ 是 SPD,则 $A^{-1}$ 也是 SPD。
                \item \textbf{Cholesky 分解}:$A$ 存在唯一的分解 $A=LL^T$,其中 $L$ 为对角元为正的下三角阵。
            \end{enumerate}

        
    \item \textbf{酉矩阵 (正交矩阵)}:$U^H U = I$。保持向量 2-范数不变 ($||Ux||_2 = ||x||_2$)。
\end{itemize}




\subsection {矩阵空间}
\tofill[性质]{矩阵空间的基本性质:线性空间、乘法运算、代数性质}


\subsubsection {矩阵范数}
\begin{definition}[矩阵范数]
    矩阵空间$\mathbb{C}^{n\times n}$上的范数$\|\cdot\|$称为矩阵范数,如果对任意的$A,B \in \mathbb{C}^{n\times n}$和$a \in \mathbb{C}$,满足以下性质:
    \begin{enumerate}
        \item 非负性与分离性:$\|A\| \geq 0$,且当且仅当$A=0$时,$\|A\|=0$
        \item 齐次性:$\|aA\| = |a| \|A\|$
        \item 三角不等式:$\|A+B\| \leq \|A\| + \|B\|$
        \item 次乘性:$\|AB\| \leq \|A\| \cdot \|B\|$
    \end{enumerate}
\end{definition}

Note: 矩阵范数是定义在矩阵代数而非矩阵空间上的,必须与矩阵乘法相容。

\begin{definition}[矩阵范数与向量范数的相容性]
    设$\|\cdot\|_v$是向量空间$\mathbb{C}^n$上的一个范数,$\|\cdot\|_m$是矩阵空间$\mathbb{C}^{n\times n}$上的一个范数,如果对任意的$A \in \mathbb{C}^{n\times n}$和$x \in \mathbb{C}^n$,都有
    \begin{align}
        \|Ax\|_v \leq \|A\|_m \cdot \|x\|_v
    \end{align}
    则称矩阵范数$\|\cdot\|_m$与向量范数$\|\cdot\|_v$是相容的。
\end{definition}

矩阵范数的两种常见构造方法:
\begin{itemize}
    \item 直接构造:Frobenius范数
    \begin{align}
        \|A\|_F=\left( \sum_{i=1}^{n} \sum_{j=1}^{n} |a_{ij}|^2 \right)^{\frac{1}{2}}
    \end{align}
    Frobenius范数的性质:
    \begin{itemize}
        \item Frobenius范数与向量2-范数相容
        \item $\|I\|_F=\sqrt{n}$
    \end{itemize}
    \item 向量范数诱导:算子范数
    \begin{definition}
    [算子范数]
        设$\|\cdot\|_v$是向量空间$\mathbb{C}^n$上的一个范数,则可以定义矩阵空间$\mathbb{C}^{n\times n}$上的算子范数$\|\cdot\|_m$如下:
        \begin{align}
            \|A\|_m = \max_{x \neq 0} \frac{\|Ax\|_v}{\|x\|_v} = \max_{\|x\|_v=1} \|Ax\|_v
        \end{align}
        则称该范数为由向量范数$\|\cdot\|_v$诱导的算子范数。
    \end{definition}
\end{itemize}

常用的几个算子范数:
\begin{itemize}
    \item 无穷范数:行和最大值
    \begin{align}
        \|A\|_{\infty} = \max_{1 \leq i \leq n} \sum_{j=1}^{n} |a_{ij}|
    \end{align}
    \item 1-范数:列和最大值
    \begin{align}
        \|A\|_{1} = \max_{1 \leq j \leq n} \sum_{i=1}^{n} |a_{ij}|
    \end{align}
    \item 2-范数(谱范数):$A$的最大奇异值,即$A^HA$的最大特征值的平方根
    \begin{align}
        \|A\|_{2} = \sqrt{\lambda_{\max}(A^HA)}
    \end{align}
\end{itemize}


\subsubsection {谱半径}
\begin{definition}[谱半径]
    谱半径定义为矩阵所有特征值模的最大值,即
    \begin{align}
        \rho(A) = \max_{1 \leq i \leq n} |\lambda_i|
    \end{align}
\end{definition}

谱半径和矩阵范数的关系:
\begin{itemize}
    \item 矩阵范数下界:
    \begin{theorem}
        对任意$A\in\mathbb{C}^{n\times n}$,有
        \begin{align}
            \rho(A) \leq \|A\|
        \end{align}
    \end{theorem}
    \item 无穷接近范数的存在性:
    \begin{theorem}
        对任意$A\in\mathbb{C}^{n\times n}$,存在一个矩阵范数$\|\cdot\|$,使得
        \begin{align}
            \rho(A) \leq \|A\| \leq \rho(A) + \varepsilon
        \end{align}
        其中,$\varepsilon$为任意给定的正常数。
    \end{theorem}
\end{itemize}


\subsubsection {可逆矩阵相关定理}
\begin{theorem}[扰动引理I]
    给定$B\in\mathbb{C}^{n\times n}$。设$\|B\|<1$,则$I+B$可逆,且
    \begin{align}
        \|(I+B)^{-1}\| \leq \frac{1}{1-\|B\|}
    \end{align}
\end{theorem}

\begin{theorem}
    [扰动引理II]
    设$A,\,C\in\mathbb{C}^{n\times n}$,且$A$可逆。若 
    \begin{align}
        \|C-A\| < \frac{1}{\|A^{-1}\|}
    \end{align}
    则$C$也可逆,且
    \begin{align}
        \|C^{-1}\| \leq \frac{\|A^{-1}\|}{1 - \|A^{-1}\| \cdot \|C-A\|}
    \end{align}
\end{theorem}


\begin{theorem}
    [扰动定理II]
    设$A,\,\delta A \in\mathbb{C}^{n\times n}$,且$A$可逆。若$\|A^{-1}\delta A\| < 1$,则$A + \delta A$也可逆,且
    \begin{align}
        \|(A + \delta A)^{-1}\| \leq \frac{\|A^{-1}\|}{1 - \|A^{-1}\| \cdot \|\delta A\|}
    \end{align}
\end{theorem}





\chapter {函数插值与重构}

\section*{总结:基本方法是利用插值基函数构造插值多项式,从而实现对函数的近似与重构。}

问题1:求解插值函数
\begin{itemize}
    \item 整个区间上的连续插值:Lagrange插值、Newton插值、Hermite插值
    \begin{itemize}
        \item Lagrange插值基函数:
        \begin{align}
            L_\alpha(x)=\prod_{\substack{\beta\in I\\ \beta\neq \alpha}} \frac{x - x_\beta}{x_\alpha - x_\beta}, \quad \alpha\in I
        \end{align}
        \item  Newton插值和Hermite插值:构造均差表,列表计算。
        \begin{itemize}
            \item 均差递推关系:
            \begin{align}
                f[x_i] &= f(x_i)\\
                f[x_i,x_{i+1},\ldots,x_{i+k}] &= \frac{f[x_{i+1},\ldots,x_{i+k}] - f[x_i,\ldots,x_{i+k-1}]}{x_{i+k} - x_i}
            \end{align}
            \item 均差构造插值多项式:各项为$f[x_0,x_1,\ldots, x_k]\cdot (x-x_0)(x-x_1)\ldots(x-x_{k-1})$
        \end{itemize}
    \end{itemize}
    \item 分段插值:分片线性插值、分片三次Hermite插值 
    \begin{itemize}
        \item 分片线性插值基函数:
        \begin{align}
            L_{\alpha}(x) =& \begin{cases}
                \frac{x - x_{\alpha-1}}{x_\alpha - x_{\alpha-1}}, & x \in [x_{\alpha-1}, x_\alpha]\\
                \frac{x_{\alpha+1} - x}{x_{\alpha+1} - x_\alpha}, & x \in [x_\alpha, x_{\alpha+1}]\\
                0, & \text{else}
            \end{cases}\\
            \phi(x)=&\sum_{\alpha=0}^{n} f_\alpha L_\alpha(x)
        \end{align}
        \item 分片三次Hermite插值基函数:
        \begin{align}
            \alpha_k =& \left(  1+2\frac{x-x_k}{x_{k+1}-x_k}  \right) \left(  \frac{x_{k+1}-x}{x_{k+1}-x_k}  \right)^2\\
            \beta_k =& (x-x_k) \left(  \frac{x_{k+1}-x}{x_{k+1}-x_k}  \right)^2\\
            \alpha_{k+1} =& \left(  1+2\frac{x_{k+1}-x}{x_{k+1}-x_k}  \right) \left(  \frac{x-x_k}{x_{k+1}-x_k}  \right)^2\\
            \beta_{k+1} =& -(x_{k+1} - x) \left(  \frac{x - x_k}{x_{k+1}-x_k}  \right)^2\\
            \phi(x) =& \sum_{k=0}^{n} \left[ f_k \alpha_k + f_k' \beta_k  \right],\quad x\in[x_k,x_{k+1}]
        \end{align}
    \end{itemize}
\end{itemize}


\section {通用理论}
\subsection {问题模型}
\tofill[数学描述]{采样泛函视角下的插值问题数学描述}
\subsection {插值空间}
常用的插值空间:多项式函数空间、样条函数空间、三角多项式函数空间
\subsection {误差分析与收敛性}

\section {具体插值方法}
\subsection {一维多项式插值}
问题:给定插值数据(采样数据)$(x_\alpha, f_\alpha)$,$\alpha\in I$,确定多项式$P(x)\in P_n$,
$n=|I|-1$,满足插值条件
\begin{align}
    x_\alpha(P)=P(x_\alpha)=f_\alpha, \quad \alpha\in I
\end{align}

\begin{theorem}
    [多项式插值基本定理]

    给定$n+1$个插值条件
    \begin{align}
        (x_\alpha,\,f_\alpha),\quad \alpha\in I, \quad x_\alpha\neq x_\beta \text{ for }\alpha\neq\beta
    \end{align}
    则存在唯一的插值多项式$P\in P_n$满足插值条件。
\end{theorem}

Note:若$x_\alpha$取之于复平面,上述定理依然成立;且上述定理与采样节点的排序无关。 

\subsubsection {Lagrange 插值}
\paragraph {基函数构造}
\begin{definition}
    [Lagrange插值基函数]
    定义
    \begin{align}
        L_\alpha(x)=L_{\alpha;I}(x)=\prod_{\substack{\beta\in I\\ \beta\neq \alpha}} \frac{x - x_\beta}{x_\alpha - x_\beta}, \quad \alpha\in I
    \end{align}
    称为插值基函数。
\end{definition}

若给定3个插值条件$(x_0,f_0)$,$(x_1,f_1)$,$(x_2,f_2)$,则对应的插值基函数为
\begin{align}
    L_0(x) &= \frac{(x - x_1)(x - x_2)}{(x_0 - x_1)(x_0 - x_2)}\\
    L_1(x) &= \frac{(x - x_0)(x - x_2)}{(x_1 - x_0)(x_1 - x_2)}\\
    L_2(x) &= \frac{(x - x_0)(x - x_1)}{(x_2 - x_0)(x_2 - x_1)}
\end{align}

插值基函数天然满足性质:
\begin{align}
    x_\beta(L_\alpha)=L_\alpha(x_\beta)=\delta_{\alpha\beta}
\end{align}

\paragraph {插值公式}
在计算出插值基函数的基础上,插值多项式可写为:
\begin{align}
    P(x)=\sum_{\alpha\in I} f_\alpha L_\alpha(x)
\end{align}

\paragraph {余项}





\paragraph {均差定义}
\begin{definition}
    [均差的递推公式]
    设$f(x)$在区间$[a,b]$上有定义,且给定插值节点$x_0,x_1,\ldots,x_n$,则定义如下均差:
    \begin{align}
        f[x_i] &= f(x_i)\\
        f[x_i,x_{i+1},\ldots,x_{i+k}] &= \frac{f[x_{i+1},\ldots,x_{i+k}] - f[x_i,\ldots,x_{i+k-1}]}{x_{i+k} - x_i}
    \end{align}
    其中,$i=0,1,\ldots,n-k$,$k=1,2,\ldots,n$。
\end{definition}

均差的性质:
\begin{itemize}
    \item $f_{i_0i_1...i_k}$与节点$x_{i_0},x_{i_1},\ldots,x_{i_k}$的顺序无关
    \item 设f是N次多项式,若$k>N$,则对任意节点$x_{i_0},x_{i_1},\ldots,x_{i_k}$,都有
    \begin{align}
        f_{i_0i_1\ldots i_k} = 0
    \end{align}
\end{itemize}


\paragraph {插值公式}
\begin{align}
    P_{i_0i_1...i_k}(x) &= f_{i_0} + f_{i_0i_1}(x - x_{i_0}) + f_{i_0i_1i_2}(x - x_{i_0})(x - x_{i_1}) + \ldots \nonumber\\
    &+ f_{i_0i_1\ldots i_k}(x - x_{i_0})(x - x_{i_1})\cdots(x - x_{i_{k-1}})
\end{align}






\paragraph{Newton插值多项式的列表计算}

以给定4个节点时$x_0,\,x_1,\,x_2,\,x_3$的插值问题为例。可以按照如下表格从左向右逐列填写计算均差:

\begin{center}
\renewcommand{\arraystretch}{1.5}
\begin{tabular}{|c|cccc|}
\hline
$x_i$ & 0阶均差 & 1阶均差 & 2阶均差 & 3阶均差 \\
\hline
$x_0$ & \fcolorbox{red}{white}{$f(x_0)$} & & & \\
& & \fcolorbox{red}{white}{$f[x_0,x_1]=\dfrac{f(x_1)-f(x_0)}{x_1 - x_0}$} & & \\
$x_1$ & $f(x_1)$ & & \fcolorbox{red}{white}{$f[x_0,x_1,x_2]=\dfrac{f[x_1,x_2]-f[x_0,x_1]}{x_2 - x_0}$} & \\
& & $f[x_1,x_2]=\dfrac{f(x_2)-f(x_1)}{x_2 - x_1}$ & & \fcolorbox{red}{white}{$f[x_0,x_1,x_2,x_3]=...$} \\
$x_2$ & $f(x_2)$ & & $f[x_1,x_2,x_3]=\dfrac{f[x_2,x_3]-f[x_1,x_2]}{x_3 - x_1}$ & \\
& & $f[x_2,x_3]=\dfrac{f(x_3)-f(x_2)}{x_3 - x_2}$ & & \\
$x_3$ & $f(x_3)$ & & & \\
\hline
\end{tabular}
\end{center}
之后用插值表最上方一行的均差值逐个组装Newton插值多项式。


\subsubsection {Hermite 插值}
\paragraph{Hermite插值问题}
给定$\xi_i$, $f_i^{(k)}$, $i=0,1,...,m$,  $k=0,1,...,n_i-1$, 其中$\xi_i$两两不同,且 
\begin{align}
    \xi_0<\xi_1<...<\xi_m 
\end{align}
希望确定一个次数为n的多项式函数
\begin{align}
    P_n(x),\quad n=\sum_{i=0}^{m} n_i - 1
\end{align}
满足插值条件
\begin{align}
    P^{(k)}(\xi_i)=f_i^{(k)}, \quad i=0,1,...,m \quad k=0,1,...,n_i-1
\end{align}


\paragraph{拓展均差}
\begin{definition}
    [拓展均差]
    设$f\in C^n(I(x_0,\,x_1,\,...,\,x_n))$,定义
    \begin{align*}
        f[x_0,\,x_1,\,x_n]=\int_0^{t_0} \mathrm{d}t_1 &\int_0^{t_1}\mathrm{d}t_2 \cdots \int_0^{t_{n-1}} \mathrm{d}t_n \\\\
        &f^{(n)}(t_n[x_n-x_{n-1}]+t_{n-1}[x_{n-1}-x_{n-2}]+...+t_1[x_1-x_0]+t_0x_0)
    \end{align*}
    式中,$n\geq 1$且$t_0=1$。
\end{definition}
Note: 这一积分实际上表示了一个单位标准n-维标准型上的积分,或者说积分区域始终是一个插值节点构造的凸组合。
这隐含了一个要求是$1=t_0\geq t_1 \geq t_2 \geq ... \geq t_n \geq 0$。

拓展均差的性质:
\begin{itemize}
    \item 若$x_i$两两不一,则拓展均差等价于普通均差
    \begin{align}
        f[x_0,\,x_1,\,x_n]=f_{x_0,x_1,...,x_n}
    \end{align}
    且具有相同的递推关系:
    \begin{align}
        f[x_0,\,x_1,\,...,\,x_n]=\frac{f[x_0,\,x_1,\,...,\,x_{n-2},\,x_n]-f[x_0,\,x_1,\,...,\,x_{n-1}]}{x_n - x_0}
    \end{align}
    \item 极限性质:若$f$足够光滑,则
    \begin{align}
        \lim_{\epsilon_i\rightarrow 0}f[x_0+\epsilon_0,\,x_1+\epsilon_1,\,...,\,x_n+\epsilon_n]=f[x_0,\,x_1,\,...,\,x_n]
    \end{align}
    \item 导数与重节点:可从极限性质导出
    \begin{align}
        \frac{\mathrm{d}}{\mathrm{d}x}f[x_0,x_1,...,x_n,x]=f[x_0,x_1,...,x_n,x,x]
    \end{align}
    \item 介值定理:若$f\in C^n[a,b]$,$x_0,\,x_1,\,...,\,x_n\in [a,b]$,则存在$\xi\in I(x_0,\,x_1,\,...,\,x_n)$,使得
    \begin{align}
        f[x_0,\,x_1,\,...,\,x_n]=\frac{f^{(n)}(\xi)}{n!}
    \end{align}
    特别地,
    \begin{align}
         f[\underbrace{x, x, \dots, x}_{n+1\text{个}}]=\frac{f^{(n)}(x)}{n!}
    \end{align}
\end{itemize}

\paragraph{Hermite插值多项式}
Hermite插值多项式可表示为
\begin{align}
    P(x)=f[x_0]+f[x_0,x_1](x-x_0)+...+f[x_0,x_1,...,x_n](x-x_0)(x-x_1)...(x-x_{n-1})
\end{align}
其中,$x_0,\,...,\,x_n$为下面序列的任意置换:
\begin{align}
    \underbrace{\xi_0,\,\xi_0,\,...,\,\xi_0}_{n_0\text{个}},\,\underbrace{\xi_1,\,\xi_1,\,...,\,\xi_1}_{n_1\text{个}},\, ...,\,\underbrace{\xi_m,\,\xi_m,\,...,\,\xi_m}_{n_m\text{个}}
\end{align}

\paragraph{列表法求Hermite插值多项式}
假设给定2个节点$\xi_0,\,\xi_1$,对应的插值条件分别为$f_0,\,f_0',\,f_0''$,$f_1$,则可按下表计算均差:


Hermite 插值均差表(节点序列:$\xi_0, \xi_0, \xi_0, \xi_1$):
\[
\begin{array}{c|c|c|c|c}
\text{节点} & \text{0 阶均差} & \text{1 阶均差} & \text{2 阶均差} & \text{3 阶均差} \\
\hline
\xi_0 & f[\xi_0] = f_0 & & & \\
\xi_0 & f[\xi_0] = f_0 & f[\xi_0,\xi_0] = f_0' & & \\
\xi_0 & f[\xi_0] = f_0 & f[\xi_0,\xi_0] = f_0' & f[\xi_0,\xi_0,\xi_0] = \dfrac{f_0''}{2} & \\
\xi_1 & f[\xi_1] = f_1 & f[\xi_0,\xi_1] = \dfrac{f_1 - f_0}{\xi_1 - \xi_0} & 
f[\xi_0,\xi_0,\xi_1] = \dfrac{f[\xi_0,\xi_1] - f[\xi_0,\xi_0]}{\xi_1 - \xi_0} &
f[\xi_0,\xi_0,\xi_0,\xi_1] = \dfrac{f[\xi_0,\xi_0,\xi_1] - f[\xi_0,\xi_0,\xi_0]}{\xi_1 - \xi_0}
\end{array}
\]

对应的插值多项式为:
\[
\begin{aligned}
P(x) &= f[\xi_0] + f[\xi_0,\xi_0](x - \xi_0) + f[\xi_0,\xi_0,\xi_0](x - \xi_0)^2 + f[\xi_0,\xi_0,\xi_0,\xi_1](x - \xi_0)^3 \\
&= f_0 + f_0'(x - \xi_0) + \frac{f_0''}{2}(x - \xi_0)^2 \\
&\quad + \frac{2(f_1 - f_0 - f_0'(\xi_1 - \xi_0)) - f_0''(\xi_1 - \xi_0)^2}{2(\xi_1 - \xi_0)^3}(x - \xi_0)^3
\end{aligned} 
\]


\subsubsection{Lagrange插值和Hermite插值的收敛分析}
\paragraph{插值余项}
若f在区间$[a,b]$上具有$n+1$阶连续导数,则对任意$x\in[a,b]$,存在$\xi\in I(x, x_0, x_1, \ldots, x_n)$,使得
\begin{align}
    R(x) = f(x) - P(x) = \frac{f^{(n+1)}(\xi)}{(n+1)!} \prod_{i=0}^{n} (x - x_i)= \frac{f^{(n+1)}(\xi)}{(n+1)!} \omega_{01...n}(x)
\end{align}

此外,使用$n+1$个节点$x_0,\,x_1,\,\ldots,\,x_n$进行Hermite插值时,总有
    
\begin{align}
    R(x)=f(x)-P(x)=f[x_0,x_1,\ldots ,x_n,x](x-x_0)(x-x_1)\cdots (x-x_n)   
\end{align}


\paragraph{收敛性}
收敛性的定义:当给定插值点的最大间距$h\rightarrow 0$时,插值余项$R(x)\rightarrow 0$,则称插值多项式序列在区间$[a,b]$上收敛于函数$f(x)$。


\begin{definition}
    [多项式插值收敛定义]
    设$f\in C^{\infty}[a,b]$,且存在正常数$M>0$,使得对任意的$n\geq 0$,都有
    \begin{align}
        \max_{x\in[a,b]} |f^{(n)}(x)| \leq M
    \end{align}
    则对任意在$[a,b]$上的插值节点序列$\{x_i^{(n)}\}_{i=0}^{n}$,对应的插值多项式序列$\{P_n(x)\}$在$[a,b]$上均匀收敛于$f(x)$。
\end{definition}


收敛的充分条件:
\begin{theorem}
    记$\delta =|I(x_0,\,x_1,\,x_2, \ldots, x_n)|$, $\tilde{x}$为I的中心。若$f$在$B(\tilde{x},w\delta)$上复解析,则对任意$\bar{x}\in I$,插值法收敛。
\end{theorem}

\subsection {分段插值}
\subsubsection {分段线性插值}
\tofill{分片线性插值问题描述}

\paragraph{分片线性插值的插值基函数}
\begin{definition}
    [分片线性插值基函数]
    设给定插值节点$x_0<x_1<...<x_n$,则定义分片线性插值基函数为
    \begin{align}
        l_i(x) = \begin{cases}
            \frac{x - x_{i-1}}{x_i - x_{i-1}}, & x\in [x_{i-1}, x_i]\\
            \frac{x_{i+1} - x}{x_{i+1} - x_i}, & x\in [x_i, x_{i+1}]\\
            0, & \text{otherwise}
        \end{cases}
    \end{align}
    其中,$i=0,1,...,n$,且约定$x_{-1}=x_0$,$x_{n+1}=x_n$。
\end{definition}
则线性插值基函数为 
\begin{align}
    \phi(x)=\sum_{k=1}^nf_k l_k(x)
\end{align}

\paragraph{分片线性插值的收敛性定理}
定义
\begin{align}
    h=\max_{1\leq i \leq n}(x_i - x_{i-1})
\end{align}

\begin{itemize}
    \item 若$f\in C[a,b]$,则$\lim_{h\rightarrow 0}\|f-\phi\|_\infty\rightarrow 0$
    \item 若$f\in C^1[a,b]$,则$\|f-\phi\|_\infty \leq \frac{h}{2}\|f'\|_\infty$
    \item 若$f\in C^2[a,b]$,则$\|f-\phi\|_\infty \leq \frac{h^2}{8}\|f''\|_\infty$
\end{itemize}

\subsubsection {分段三次 Hermite 插值}
\tofill[]{分片三次Hermite插值的数学描述}


\paragraph{插值基函数}
分段三次Hermite插值的基函数满足如下条件:
\begin{align}
    \alpha_k(x_i)=& \delta_{ik},\quad \alpha_k'(x_i)=0\\
    \beta_k(x_i)=& 0,\quad \beta_k'(x_i)= \delta_{ik}
\end{align}

在单元$[x_k, x_{k+1}]$上,三次Hermite插值基函数为
\begin{align}
    \alpha_k =& \left(  1+2\frac{x-x_k}{x_{k+1}-x_k}  \right) \left(  \frac{x_{k+1}-x}{x_{k+1}-x_k}  \right)^2\\
    \beta_k =& (x-x_k) \left(  \frac{x_{k+1}-x}{x_{k+1}-x_k}  \right)^2\\
    \alpha_{k+1} =& \left(  1+2\frac{x_{k+1}-x}{x_{k+1}-x_k}  \right) \left(  \frac{x-x_k}{x_{k+1}-x_k}  \right)^2\\
    \beta_{k+1} =& -(x_{k+1} - x) \left(  \frac{x - x_k}{x_{k+1}-x_k}  \right)^2
\end{align}

\begin{itemize}
    \item $\alpha_k$满足单位分解性:$\alpha_k+\alpha_{k+1}=1$
    \item $\alpha_k(x_i)=\delta_{ik}$,$\alpha_k'(x_i)=0$
    \item $\beta_k(x_i)=0$,$\beta_k'(x_i)=\delta_{ik}$
\end{itemize}



\paragraph{三次Hermite插值多项式}
\begin{align}
    \phi=\sum_{k=0}^{n} \left[ f_k \alpha_k(x) + f_k' \beta_k(x) \right]
\end{align}

\paragraph{收敛性定理}
\begin{definition}
    [分段三次Hermite插值收敛性定理]
    设$f\in C^1[a,b]$,则分段三次Hermite插值多项式$\phi$满足
    \begin{align}
        \|f - \phi\|_\infty \leq ch\|f'\|_\infty
    \end{align}
\end{definition}

若f有更好的光滑性,则:
\begin{itemize}
    \item 若$f\in C^2[a,b]$,则$\|f-\phi\|_\infty \leq ch^2\|f''\|_\infty$
    \item 若$f\in C^3[a,b]$,则$\|f-\phi\|_\infty \leq ch^3\|f'''\|_\infty$
    \item 若$f\in C^4[a,b]$,则$\|f-\phi\|_\infty \leq \frac{1}{384}h^4\|f^{(4)}\|_\infty$
\end{itemize}

此外,对于不高于三次的多项式,分段三次Hermite插值是精确的。



\subsection {Fourier 插值}
\subsubsection{离散傅里叶变换}
\tofill[定义]{离散傅里叶变换式、变换式系数表达式}

如果f的光滑性满足$f\in C_{per}^M$,则有
\begin{align}
    a_n=O(n^{-M})\\
    b_n=O(n^{-M})
\end{align}
且
\begin{align}
    \|f(x)-\left[  \frac{a_0}{2}+\sum_{n=1}^N a_n\cos nx+\sum_{n=1}^N b_n \sin nx \right]\|_\infty = O(N^{-M})
\end{align}

\subsubsection{三角多项式插值空间}
三角多项式插值空间为:
\begin{align}
    \Phi_{2M+1}:=&\left\{  \frac{A_0}{2} + \sum_{n=1}^M \left( A_n \cos nx + B_n \sin nx \right) \right\}\\
    \Phi_{2M}:=&\left\{  \frac{A_0}{2} + \sum_{n=1}^{M-1} \left( A_n \cos nx + B_n \sin nx \right) + A_M \cos Mx \right\}
\end{align}



\subsubsection {三角多项式插值与一般多项式插值}
\tofill[问题]{三角多项式插值问题的数学描述}

\tofill[问题]{辅助插值问题:找相多项式}

\tofill[理论]{两个插值问题之间的联系:欧拉公式}

\subsubsection{插值定理与三角插值多项式}
\begin{theorem}
    [三角多项式插值定理]
    设给定插值节点
    \begin{align}
        x_k = \frac{2k\pi}{N}, \quad k=0,1,\ldots,N-1
    \end{align}
    则对任意插值数据$f_k$,存在唯一的三角多项式
    \begin{align}
        P(x) = \frac{A_0}{2} + \sum_{n=1}^{M} \left( A_n \cos nx + B_n \sin nx \right)
    \end{align}
    (当N为奇数时,$M=\frac{N-1}{2}$;当N为偶数时,$M=\frac{N}{2}$)满足插值条件
    \begin{align}
        P(x_k) = f_k, \quad k=0,1,\ldots,N-1
    \end{align}
    同样,存在唯一的相多项式
    \begin{align}
        Q(x) = \sum_{n=0}^{N-1} \beta_n e^{inx}
    \end{align}
    满足插值条件
    \begin{align}
        Q(x_k) = f_k, \quad k=0,1,\ldots,N-1
    \end{align}
\end{theorem}


\begin{itemize}
    \item 三角插值多项式:
    \begin{align}
        A_j=\frac{2}{N}\sum_{k=0}^{N-1} f_k \cos\left(  \frac{2\pi jk}{N}  \right), \quad j=0,1,\ldots,M\\
        B_j=\frac{2}{N}\sum_{k=0}^{N-1} f_k \sin\left(  \frac{2\pi jk}{N}  \right), \quad j=1,2,\ldots,M
    \end{align}
    \item 相插值多项式:
    \begin{align}
        \beta_j=\frac{1}{N}\sum_{k=0}^{N-1} f_k w^{-kj}, \quad j=0,1,\ldots,N-1
    \end{align}
    式中,$w=e^{i\frac{2\pi}{N}}$。
\end{itemize}


% \subsubsection {离散 Fourier 变换关联}



\chapter {函数逼近}

\section*{总结}
Note:这里为了让总结看起来更顺畅,没有遵循原PPT和书中的符号规范,转而应用了比较统一的符号。这些符号规范仅在总结一部分使用。
\begin{itemize}
    \item 最佳平方逼近求解:
    \begin{itemize}
        \item 给定一组规范正交函数基时:
        \begin{itemize}
            \item 写出规范正交函数基$\{\phi_m\}_{m=0}^n$
            \item 计算系数$a_m=\dfrac{(f,\phi_m)}{(\phi_m,\phi_m)}$
            \item 写出逼近多项式$\phi^*=\sum_{m=0}^n a_m \phi_m$
        \end{itemize}
        \item 给定一组非规范正交函数基时:
        \begin{itemize}
            \item 写出非规范正交函数基$\{\phi_m\}_{m=0}^n$
            \item 构建法方程组:
            \begin{align}
                m_{ij}=(\phi_j,\phi_i),\quad b_i=(f,\phi_i)\\
            \end{align}
            \item 求解线性方程组$\mathbf{M}\mathbf{a}=\mathbf{b}$,得到系数$a_m$
            \item 写出逼近多项式$\phi^*=\sum_{m=0}^n a_m \phi_m$
        \end{itemize}
    \end{itemize}
    \item Legendre多项式作最佳平方逼近的收敛速度:高阶时控制在$1/\sqrt{n}$以下
    \item 最小二乘逼近求解:计算过程可视作最佳平方逼近的离散形式
    \begin{itemize}
        \item 给定基底$\{\phi_m\}_{m=0}^n$和采样点$\{x_k\}_{k=0}^N$
        \item 估计一个误差系数$\rho(x_i)$
        \item 用半内积构建法方程组:
        \begin{align}
            m_{ij}=\sum_{k=0}^N \rho(x_k) \phi_j(x_k)\phi_i(x_k),\quad b_i=\sum_{k=0}^N \rho(x_k) f(x_k)\phi_i(x_k)\\
        \end{align}
        \item 求解线性方程组$\mathbf{M}\mathbf{a}=\mathbf{b}$,得到系数$a_m$
        \item 写出逼近多项式$\phi^*=\sum_{m=0}^n a_m \phi_m$
    \end{itemize}
    \item 一致逼近求解:
    \begin{itemize}
        \item 给定原函数$f\in C[a,b]$和基底$\{\phi_m\}_{m=0}^n$
        \item 写出待定逼近多项式$\phi(x)=\sum_{m=0}^n a_m \phi_m(x)$
        \item 写出误差函数$E(x)=f(x)-\phi(x)$
        \item 根据切比雪夫交错点组定理,写出求解条件:
        \begin{itemize}
            \item 交错条件:$E(x_i)=-e(x_{x-1})$
            \item 偏差点条件:除端点作为偏差点的情况,必有$f'(x_i)=p_n'(x_i)$
        \end{itemize}
        \item 由偏差点条件可以解出一系列偏差点$x_i(\boldsymbol{a})$,再代入交错条件中,解出系数$a_m$
    \end{itemize}
    
\end{itemize}
\section {通用理论}

\subsection {问题模型}
\tofill[问题]{函数逼近问题的数学模型}

最佳逼近问题:找$\phi^*\in \Phi$,使得 
\begin{align}
    \|f-\phi^*\| = \min_{\phi\in \Phi} \|f-\phi\|
\end{align}


\subsection {逼近准则}
\subsubsection {最小二乘准则(L₂范数)}
\subsubsection {一致逼近准则(L∞范数)}

\subsection {核心定理}

\section {具体逼近方法}

\subsection {最优平方逼近}
\subsubsection{基础理论}

问题:给定一个线性子空间$\Phi$,找$\phi^*\in \Phi$,使得
\begin{align}
    \|f-\phi^*\|_2 = \min_{\phi\in \Phi} \|f-\phi\|_2
\end{align}

问题等价于
\begin{align}
    \|f-\phi^*\|_2^2 = \min_{\phi\in \Phi} \|f-\phi\|_2^2
\end{align}

于是构造出一个辅助函数:
\begin{align}
    I=\|f-\phi\|_2^2=\left( f-\sum_{i=0}^n a_i\phi_i, f-\sum_{i=0}^n a_i\phi_i \right)\\
    =  \sum_{i=0}^n \sum_{j=0}^n a_i a_j (\phi_i, \phi_j) - 2\sum_{i=0}^n a_i (f,\phi_i) +(f,f)
\end{align}

式中,$\{\phi_i\}$为$\Phi$的一组基。

\paragraph{法方程}

多元函数取得最小值的条件:对各变量偏导为0、在这一点的二阶偏导数构成的矩阵正定。

各变量偏导为0即导出法方程:
\begin{align}
    \sum_{j=0}^{n} a_j ( \phi_j, \phi_i ) = ( f, \phi_i ), \quad i=0,1,\ldots,n
\end{align}
或写成矩阵形式,
\begin{align}
    \begin{bmatrix}
        ( \phi_0, \phi_0 ) & ( \phi_1, \phi_0 ) & \cdots & ( \phi_n, \phi_0 ) \\
        ( \phi_0, \phi_1 ) & ( \phi_1, \phi_1 ) & \cdots & ( \phi_n, \phi_1 ) \\
        \vdots & \vdots & \ddots & \vdots \\
        ( \phi_0, \phi_n ) & ( \phi_1, \phi_n ) & \cdots & ( \phi_n, \phi_n )
    \end{bmatrix}
    \begin{bmatrix}
        a_0 \\ a_1 \\ \vdots \\ a_n
    \end{bmatrix}
    =
    \begin{bmatrix}
        ( f, \phi_0 ) \\ ( f, \phi_1 ) \\ \vdots \\ ( f, \phi_n )
    \end{bmatrix}
\end{align}


\begin{theorem}
    [最小二乘逼近法方程]
    设$\Phi$为线性子空间,$\{\phi_0, \phi_1, \ldots, \phi_n\}$为$\Phi$的一组基,则函数$f$在$\Phi$中的最优平方逼近$\phi^*$可表示为
    \begin{align}
        \phi^* = \sum_{j=0}^{n} a_j^* \phi_j
    \end{align}
    其中,系数$c_j^*$满足法方程组
    \begin{align}
        \sum_{j=0}^{n} a_j ( \phi_j, \phi_i ) = ( f, \phi_i ), \quad i=0,1,\ldots,n
    \end{align}
\end{theorem}

\paragraph{最佳平方逼近的性质}
\begin{lemma}
    [最佳平方逼近的正交性质]
    设$\phi^*$为函数$f$在子空间$\Phi$中的最佳平方逼近,则对任意$\phi\in \Phi$,都有
    \begin{align}
        f-\phi^* \perp \Phi
    \end{align}
\end{lemma}

\begin{corollary}[最佳平方逼近的勾股定理]
    设$\phi^*$为函数$f$在子空间$\Phi$中的最佳平方逼近,则有
    \begin{align}
        \|f\|_2^2 = \|f-\phi^*\|_2^2 + \|\phi^*\|_2^2
    \end{align}
\end{corollary}


\subsubsection{幂函数基的逼近}
考虑连续函数在n-次多项式空间$P_n$中的最佳平方逼近问题。

若选取$\{x^m\}_{m=0}^n$作为基函数,在$[0,1]$区间上求解最佳平方逼近,则法方程组的系数矩阵为Hilbert矩阵:
\begin{definition}
    [Hilbert矩阵]
    阶数为$n+1$的Hilbert矩阵定义为
    \begin{align}
        H_{ij} = \frac{1}{i+j-1}, \quad i,j=1,2,\ldots,n+1
    \end{align}
    即
    \begin{align}
        H_n=\begin{bmatrix}
        1 & \frac{1}{2} & \frac{1}{3} & \cdots & \frac{1}{n+1} \\
        \frac{1}{2} & \frac{1}{3} & \frac{1}{4} & \cdots & \frac{1}{n+2} \\
        \frac{1}{3} & \frac{1}{4} & \frac{1}{5} & \cdots & \frac{1}{n+3} \\
        \vdots & \vdots & \vdots & \ddots & \vdots \\
        \frac{1}{n+1} & \frac{1}{n+2} & \frac{1}{n+3} & \cdots & \frac{1}{2n+1}
        \end{bmatrix}
    \end{align}
\end{definition}

这一逼近形式的问题:
\begin{itemize}
    \item Hilbert矩阵病态,随着n的增大,条件数迅速增大,导致数值解不稳定
    \item 幂函数基不正交,导致法方程系数矩阵接近奇异
\end{itemize}


\subsubsection {正交多项式逼近}

\paragraph{广义傅里叶展开}
在$C[a,b]$中,取一个规范正交函数组$\{\phi_m\}_{m=0}^n$,即确定了一个用于逼近的线性子空间$\Phi_n=\text{span}\{\phi_0, \phi_1, \ldots, \phi_n\}$。

相应地,对于任意给定函数f,存在唯一的最佳平方逼近$\phi^*\in \Phi_n$,且
\begin{align}
    \phi^* = \sum_{i=0}^n a_i^* \phi_i, \quad a_i^* = (f, \phi_i)
\end{align}

\begin{definition}
    [广义傅里叶展开]
    设$\{\phi_m\}_{m=0}^\infty$为$C[a,b]$中的规范正交函数组,则对任意$f\in C[a,b]$,都有
    \begin{align}
        f \sim f_\infty= \sum_{i=0}^\infty a_i \phi_i, \quad a_i = (f, \phi_i)
    \end{align}
    $f_\infty$称为函数f在$\{\phi_m\}$下的广义傅里叶展开。
\end{definition}

\begin{theorem}
    [广义傅里叶展开的收敛性]
    设$\{\phi_m\}_{m=0}^\infty$为$C[a,b]$中的规范正交函数组,则对任意$f\in C[a,b]$,其广义傅里叶展开在$L_2$范数下收敛于f,即
    \begin{align}
        \lim_{n\rightarrow \infty} \|f - f_n\|_2 = 0
    \end{align}
    其中,$f_n = \sum_{i=0}^n a_i \phi_i$。
\end{theorem}


\paragraph{Legendre多项式作最佳平方逼近}
设内积为$[-1,1]$上的权1内积,给定$f\in C[-1,1]$,则f在Legendre多项式空间$P_n$中的最佳平方逼近为
\begin{align}
    I_nf(x)=\sum_{k=0}^n \frac{(f,\phi_k)}{\|\phi_k\|^2}\phi_k(x)
\end{align}

性质:
\begin{itemize}
    \item 收敛性:$I_nf\rightarrow f,\quad n\rightarrow \infty$,且收敛到$L^2((-1,1))$空间中
    \item 收敛速度估计:若$f\in C^2[-1,1]$,则$\forall \epsilon>0$,$\exists N>)$,使得
    \begin{align}
        \|f - I_nf\|_\infty \leq \frac{\epsilon}{\sqrt{n}},\quad n\geq N
    \end{align}
    即对于n次的Legendre多项式逼近,误差会被控制在$O(\frac{1}{\sqrt{n}})$范围内
\end{itemize}

\subsubsection{Legender多项式的零平方误差最小性质}
\begin{theorem}
    [Legendre多项式的零平方误差最小性质]
    设$f\in C[-1,1]$,则在所有满足$\deg(p_n)\leq n$且$(p_n, x^k)=0,\quad k=0,1,\ldots,n-1$的多项式中,Legendre多项式$P_n(x)$使得平方误差$\|f - P_n\|_2$最小。
\end{theorem}




\subsection {最小二乘逼近:最优平方逼近的离散化形式}
\tofill[问题]{最小二乘逼近的问题模型:确定的输入-输出关系叠加系统噪声和随机误差,希望拟合输入-输出关系的具体形式}

\subsubsection{基础理论}
核心假设:观测数据$f_i$和正确输 $\phi^*(x_i)$之间的误差$a_i\xi_i$是一个独立同分布的零均值随机变量。

假设导出:当拟合结果$\phi$取得正确的关系$\phi=\phi^*$时,(归一化)误差的方差最小。

数学化:$\phi^*$是以下优化问题的极小解:
\begin{align}
    \phi^*=\arg\min_{\phi\in \Phi} \sigma^2=\arg\min_{\phi\in \Phi} \frac{1}{m+1}\sum_{i=0}^m [f_i - \phi(x_i)]^2
\end{align}


\subsubsection{数学方法:“半”内积}

\begin{theorem}
    [Haar条件]
    若给定一组采样点$\{x_i\}_{i=0}^m$,且$\Phi$中的任意非0元素在这组采样点上的采样结果不全为0,则称这组采样点满足Haar条件。
\end{theorem}


满足Haar条件时,可以利用在采样点上的采样结果,定义一个“半”内积:
\begin{align}
    (f,g)_m = \sum_{i=0}^m \frac{f(x_i) g(x_i)}{a_i^2}
\end{align}
此时这个“半”内积在$\Phi$上是一个真正的内积。因此上述问题等价于求解法方程AU=b:
\begin{align}
    \begin{bmatrix}
        (\phi_0, \phi_0)_m & (\phi_1, \phi_0)_m & \cdots & (\phi_n, \phi_0)_m \\
        (\phi_0, \phi_1)_m & (\phi_1, \phi_1)_m & \cdots & (\phi_n, \phi_1)_m \\
        \vdots & \vdots & \ddots & \vdots \\
        (\phi_0, \phi_n)_m & (\phi_1, \phi_n)_m & \cdots & (\phi_n, \phi_n)_m
    \end{bmatrix}
    \begin{bmatrix}
        u_0 \\ u_1 \\ \vdots \\ u_n
    \end{bmatrix}
    =
    \begin{bmatrix}
        ( f, \phi_0 )_m \\ ( f, \phi_1 )_m \\ \vdots \\ ( f, \phi_n )_m
    \end{bmatrix}
\end{align}

Note: 我们注意到,在求内积过程中,分母上表示采样点随机误差幅度的$a_i^2$仍然是不确定的,此时需要基于经验
或先验知识给出估计。常用的估计有$a_i\propto 1$、$a_i\propto f_i$、$a_i\propto x_i$、$a_i\propto x_i^2$等。


\subsection{最佳一致逼近}
\tofill[问题]{最佳一致逼近的数学描述}

最佳一致逼近中用到一些符号:
\begin{itemize}
    \item $\Delta (f,p_n)=\|f-p_n\|_\infty$称为$p_n$关于f的偏差
    \item $E_n=\inf_{p_n\in P_n} \Delta (f,p_n)$称为f在$P_n$中的最小偏差
    \item 若$f(x_i)-p_n(x_i)=\sigma \Delta (f, p_n)$, $\sigma=\pm 1$,则称$x_i$为$p_n$关于f的偏差点。
    \begin{itemize}
        \item 若$\sigma=1$,则称$x_i$为正偏差点
        \item 若$\sigma=-1$,则称$x_i$为负偏差点
    \end{itemize}
\end{itemize}



\subsubsection{偏差泛函与最优逼近多项式存在性定理}
假设给定一个逼近多项式$p_n=a_0+a_1x+a_2x^2+\ldots +a_nx^n$,则可定义偏差泛函:
\begin{align}
    \phi(f,\boldsymbol{a})=\|f-p_n\|_\infty=\|f-(a_0+a_1x+\ldots +a_nx^n)\|_\infty
\end{align}

这个泛函具有以下性质:
\begin{itemize}
    % \item 正定性:$\phi(f,\boldsymbol{a})\geq 0$,且当且仅当$f=p_n$时取等号
    % \item 齐次性:$\forall \lambda \in \mathbb{R}$,都有
    % \begin{align}
    %     \phi(f, \lambda \boldsymbol{a}) = |\lambda| \phi(f, \boldsymbol{a})
    % \end{align}
    \item 连续性:$\phi$对$\boldsymbol{a}$连续
    \item 对$\boldsymbol{a}$下凸:$\forall \boldsymbol{a_1}, \boldsymbol{a_2}\in \mathbb{R}^{n+1}$,$\forall t\in [0,1]$,都有
    \begin{align}
        \phi(f, t\boldsymbol{a_1} + (1-t)\boldsymbol{a_2}) \leq t\phi(f,\boldsymbol{a_1}) + (1-t)\phi(f,\boldsymbol{a_2})
    \end{align}
    \item 正定性:对于任意$\boldsymbol{a}\neq \boldsymbol{0}$,都有$\phi(f,\boldsymbol{a})>0$
    
    正定性等价于:$\phi(0;\mathbf{a})$在单位球面$\sum_{i=0}^n a_i^2=1$上有正的最小值$\mu$
\end{itemize}

利用偏差泛函的性质,可以证明最佳一直逼近多项式的存在性:

\begin{theorem}
    [最佳一致逼近存在性定理]
    设$f\in C[a,b]$,则在$n$次多项式空间$P_n$中,存在一个多项式$p_n^*\in P_n$,使得
    \begin{align}
        \|f - p_n^*\|_\infty = \min_{p_n\in P_n} \|f - p_n\|_\infty
    \end{align}
    即$P_n$中关于$f\in C[a,b]$的最小偏差是可以达到的。
\end{theorem}

\subsubsection{最佳一致逼近多项式的性质}
\begin{definition}
    [Chebyshev交错点组]
    若$x_1<x_2<\ldots <x_m$是$p_n$关于f的轮流为正负的偏差点,则称其为一个Chebyshev交错点组。
\end{definition}

\begin{theorem}
    [最佳一致逼近多项式的Chebyshev交错定理]
    $p^*_n$是f的最佳一致逼近多项式的充要条件是存在一个元素个数为n+2的Chebyshev交错点组$x_0<x_1<\ldots <x_{n+1}$
\end{theorem}

\begin{corollary}
    最佳一致逼近多项式是一个Lagrange插值多项式,其插值节点在$(a,b)$内。
\end{corollary}


\begin{theorem}
    [最佳一致逼近多项式的唯一性定理]
    设$f\in C[a,b]$,则$f$在$n$次多项式空间$P_n$中的最佳一致逼近多项式$p_n^*$是唯一的。
\end{theorem}


\subsubsection{Chebyshev多项式的零一致误差最小性质}
\begin{theorem}
    [Chebyshev多项式的零一致误差最小性质]
    设$T_n(x)$为n次Chebyshev多项式,则在$[-1,1]$上,任意单位首项系数的n次多项式$p_n(x)$均满足
    \begin{align}
        \|p_n\|_\infty \geq \|T_n\|_\infty = \frac{1}{2^{n-1}}
    \end{align}
    即Chebyshev多项式在$[-1,1]$上具有最小的无穷范数。
\end{theorem}

\subsubsection{交错点组求解方法}
\paragraph{解析求解:利用交错点组定理}
\begin{align}
    E(x_i)=-E(x_{i-1}),\quad \forall x_i\\
    f'(x_i)=p_n'(x_i),\quad \forall x_i\neq a,b
\end{align}

\paragraph{数值求解:Remez算法}
\begin{itemize}
    \item 选择初始交错点组$\{x_i^{(0)}\}_{i=0}^{n+1}$
    \item 计算当前逼近多项式$p_n^{(k)}$,并计算偏差$E^{(k)}(x)=f(x)-p_n^{(k)}(x)$
    \item 找到新的交错点组$\{x_i^{(k+1)}\}_{i=0}^{n+1}$,即$E^{(k)}(x)$的极值点
    \item 若$\|E^{(k+1)}\|_\infty$与$\|E^{(k)}\|_\infty$足够接近,则停止迭代,否则返回步骤2
\end{itemize}





\chapter {数值微积分}
\section*{总结}
机械积分公式:
\begin{align}
    I_n(f) = \sum_{k=0}^n A_k f(x_k)
\end{align}

\subsection*{数值积分}
\begin{itemize}
    \item 数值积分方法的代数精度
    \begin{itemize}
        \item 定义:可以准确积分n次多项式,不能准确积分n+1次
        \item 判据:当有n个积分节点时,
        \begin{itemize}
            \item 以下公式确保至少达到n-1阶:
                \begin{align}
                A_k=I(L_k)=\int_a^b L_k(x)\rho(x)\mathrm{d}x
                \end{align}
            \item 最多达到2n-1阶:当且仅当积分节点为权函数$\rho(x)$对应的正交多项式的n个不同实根时
        \end{itemize}
    \end{itemize}
    \item 数值积分方法的代数稳定性
    \begin{itemize}
        \item 定义:对任意n阶的求积公式,求积系数的绝对值之和有确定的上界
        \item 判据:对任意的阶数n,所有的求积系数均大于0,则该方法是一致稳定的
        \item Newton-Cotes公式不是一致稳定的,但复合积分方法和Gauss求积公式是一致稳定的。
    \end{itemize}
    \item Newton-Cotes 公式:用多项式函数插值近似函数,再对插值多项式积分
    \item Newton-Cotes公式的具体应用
    \begin{itemize}
        \item 闭型:Newton-Cotes公式系数和余项表
        \item 开型:中点公式和余项
    \end{itemize}
    \item 复合求积方法(以下规定$h_k=x_{k+1}-x_k$,且$h_{-1}=h_n=0$)
    \begin{itemize}
        \item 复合中点公式:$A_k=h_k$
        \item 复合梯形公式:$A_k=\dfrac{h_{k-1}+h_k}{2}$
        \item 复合Simpson公式:即$S_{2n}=\frac{1}{3}T_n+\frac{2}{3}H_n=\dfrac{4T_{2n}-T_n}{3}$
    \end{itemize}
    \item 外推与Romberg方法
    \begin{itemize}
        \item Neville递推式:\textcolor{blue}{需记忆!}
        \begin{align}
            T_{ik}= T_{i,k-1}+\frac{T_{i,k-1}-T_{i-1,k-1}}{\dfrac{h_{i-k}^2}{h_i^2}-1}
        \end{align}
        \item Romberg方法:Neville在二分加密过程的特殊情况\textcolor{blue}{需记忆!}
        \begin{align}
            T_{ik}=T_{i,k-1}+\frac{T_{i,k-1}-T_{i-1,k-1}}{4^k-1}
        \end{align}
    \end{itemize}
    \item Gauss求积公式:在正交多项式零点处取积分节点,且系数为$A_k^{(n)}=I(L_k^{(n)})$,即第k个节点处的采样系数为该节点处的Lagrange基函数的\textbf{带权}积分
\end{itemize}

\subsection*{数值微分}
\begin{itemize}
    \item 基础方法:差商近似导数
    \item 插值型数值微分:
    \begin{itemize}
        \item 线性插值:
        \begin{align}
            p_1(x)=&f[x_0,x_1]x+f[x_0]\\
            p_1'(x)=&f[x_0,x_1]
        \end{align}
        \item 抛物插值:
        \begin{align}
            p_2(x)=f[x_0]+f[x_0,x_1](x-x_0)+f[x_0,x_1,x_2](x-x_0)(x-x_1)\\
            p_2'(x)=f[x_0,x_1]+f[x_0,x_1,x_2](2x - x_0 - x_1)\\
            p_2''(x)=2f[x_0,x_1,x_2]
        \end{align}
    \end{itemize}
    \item 外推方法:等距节点、二分加密下,按照
    \begin{align}
        G_1(h)=&G(h)\\
        G_{m+1}(h)=&G_m(h/2)+\frac{G_m(h/2)-G_m(h)}{2^{m}-1}=\frac{4^mG_m(h/2)-G_m(h)}{4^m-1}
    \end{align}
    \item 误差分析:
    \begin{itemize}
        \item 基础方法:向前、向后差商误差为$O(h)$,中心差商误差为$O(h^2)$。此外,中心差商求二阶导也可以达到$O(h^2)$
        \item 插值型数值微分:n次插值多项式的导数误差为$O(h^{n+1-k})$,其中k为求导阶数。此外,抛物插值在中心点处的二阶导数精度可以加一阶。
        \item 外推方法:每次外推使误差阶数增加2
    \end{itemize}
\end{itemize}


\section {数值积分}
\subsection {通用理论}
\subsubsection {积分问题模型}
\tofill[问题]{关注的对象是\textbf{带权积分}}

\begin{itemize}
    \item 目标:构造一种不依赖于函数f具体表达形式的近似积分方法
    \item 基本思想:用简单函数近似被积函数,再计算简单函数的准确积分
\end{itemize}

\subsubsection {求积公式核心}
\subsubsection {代数精度}
\begin{definition}
    如果 
    \begin{align}
        \tilde{I}(f)=I(f), \quad \forall f\in P_n
    \end{align}
    则称$\tilde{I}$的代数精度至少为n。进一步,若存在一个$P_{n+1}$中的多项式$f$,使得$\tilde{I}(f)\neq I(f)$,则称$\tilde{I}$的代数精度为n。
\end{definition}

\begin{theorem}
    [至少n-1阶代数精度的判据]
    设$L_k(x)$为以$x_1,\ldots,x_n$为节点时的第k个插值基函数,则当且仅当
    \begin{align}
        A_k=I(L_k)=\int_a^b L_k(x)\rho(x)\mathrm{d}x 
    \end{align}
    时,$I_n(\cdot)$具有至少$n-1$阶代数精度。
\end{theorem}

\begin{theorem}
    [n个求积节点的最大可能代数精度]
    n个求积节点上的求积公式的最大可能代数精度为2n-1,当且仅当求积节点恰为权函数$\rho(x)$对应的正交多项式的n个不同实根时,达到这一代数精度。
\end{theorem}



\subsubsection {一致稳定性}
设$\tilde{I}_n$是一个线性积分法,n为渐进参数(对应自由度个数)。

% 考虑给出的数据$f_k=f(x_k)$可能存在误差的情况:

% 设$\|f-g\|_\infty \leq \delta$,则


\begin{definition}
    [一致稳定性]
    设对于一个方法导出的n阶机械求积公式为
\begin{align}
    \tilde{I}_n(f)=\sum_{k=0}^n A_k^{(n)}f(x_k)
\end{align}
若存在一个$M>0$,使得对任意n阶的求积公式,均能满足
\begin{align}
    \sum_{k=0}^n |A_k^{(n)}| \leq M
\end{align}
则称该数值积分法是一致稳定的。
\end{definition}


\begin{theorem}
    若某个积分法对任意的阶数n和序数k,均有$A_k^{(n)} >0$,则这一积分法是一致稳定的。
\end{theorem}

考虑一致稳定性的意义在于,当数据存在误差时,有偏数据积分结果的误差可以由一直稳定性的参数M和误差的上界$\delta$的乘积控制。

\begin{itemize}
    \item Simpson公式是一致稳定的。
    \item Gauss求积公式总是一致稳定的。
\end{itemize}

\subsection {具体求积方法}
\subsubsection {Newton-Cotes 公式}
\paragraph{导出思路}
用多项式函数整体插值近似求积区间内的函数值,再对插值多项式进行积分,得到近似积分公式。

即如下过程导出:
\begin{itemize}
    \item 选取插值节点$\{x_i\}_{i=0}^n$。注意,这些节点不一定全在积分区间内。
    \item 构造n-次Lagrange插值多项式
    \begin{align}
        P_n(x) = \sum_{i=0}^n f(x_i) l_i(x)
    \end{align}
    其中,$l_i(x)$为Lagrange基函数:
    \begin{align}
        l_i(x) = \prod_{\substack{j=0 \\ j\neq i}}^n \frac{x - x_j}{x_i - x_j}
    \end{align}
    \item 对插值多项式进行积分,得到近似积分公式
    \begin{align}
        \tilde{I}(f)=I(P_n)=\sum_{k=0}^n f(x_k)I(L_k)=\sum_{k=0}^n A_k^{(n)} f(x_k)
    \end{align}
\end{itemize}

\paragraph{性质:}
\begin{itemize}
    \item 近似积分的精度与插值多项式逼近精度相关
    \item 积分法只与函数f在插值节点处的取值有关
\end{itemize}

\paragraph{(闭型)Newton-Cotes 公式}
取$x_i$为等距节点,且$a=x_0<x_1<\ldots <x_n=b$,则称所得求积公式为闭型Newton-Cotes公式。

积分系数为:
\begin{align}
    A_k^{(n)}=(b-a)C_k^(n)
\end{align}

\textcolor{blue}{(需记忆)Newton-Cotes 公式系数表:}
\begin{table}[H]
    \centering
    \begin{tabular}{c|ccccc}
        \hline
        n & weighted factor & & & \\
        \hline
        1 & 1 & 1 & & & \\
        2 & 1 & 4 & 1 & & \\
        3 & 1 & 3 & 3 & 1 & \\
        4 & 7 & 32 & 12 & 32 & 7 \\
        \hline
    \end{tabular}
\end{table}

数值积分结果
\begin{align}
    I_n(f)=(b-a)\sum_{k=0}^n C_k^(n) f(x_k)
\end{align}

低阶公式:
\begin{itemize}
    \item 一阶:梯形公式:
    \begin{align}
        \tilde{I}_1(f)=& \frac{b-a}{2}[f(a)+f(b)]\\
        R_1(x)=& f(x)-P_1(x) =f[a,b,x](x-a)(x-b)\\
        I(f)-I_1(f)=& \int_a^b R_1(x)dx = \frac{f''(\xi)}{2!}\int_a^b (x-a)(x-b)dx = -\frac{(b-a)^3}{12} f''(\xi)
    \end{align}
    \item 二阶:Simpson公式:
    \begin{align}
        \tilde{I}_2(f)=& \frac{b-a}{6}[f(x_0)+4f(x_1)+f(x_2)]\\
        I(f)-\tilde{I}_2(f) =& -\frac{1}{90}\left(\frac{b-a}{2}\right)^5 f^{(4)}(\xi) 
    \end{align}
    \textbf{注意:Simpson公式的误差是4阶导数,而不是3阶。}
    \item 三阶:3/8 - 规则:
    \begin{align}
        \tilde{I}_3(f)=& \frac{b-a}{8}[f(x_0)+3f(x_1)+3f(x_2)+f(x_3)]\\
        I(f)-\tilde{I}_3(f) =& -\frac{3}{80}\left(\frac{b-a}{3}\right)^5 f^{(4)}(\xi)
    \end{align}
    \item 奇数阶的Newton-Cotes公式的代数精确度是n;偶数阶的Newton-Cotes公式的代数精确度是n+1。这是由于取等距节点插值时,将均差转化为余项后,误差中的多项式在中点之前的误差和中点之后的误差恰好抵消。
\end{itemize}


\paragraph{开型Newton-Cotes公式}
取等距节点$a=x_{-1}<x_0<x_1<\ldots <x_n<x_{n+1}=b$,\textbf{仅使用中间的$x_0<x_1<\ldots <x_n$作为插值节点},则称所得求积公式为开型Newton-Cotes公式。

\begin{align}
    I(f)=&(b-a)\sum_{k=0}^n b_k(n)f(x_k)+\bar{R}_n(f)\\
    b_k^{(n)}=&\frac{1}{b-a}\int_a^b L_k(x)\mathrm{d}x=\frac{1}{b-a}\int_a^b \prod_{\substack{j=0 \\ j\neq k}}^n \frac{x - x_j}{x_k - x_j} dx
\end{align}

n=0的开型Newton-Cotes公式称为\textbf{中点公式}:
\begin{align}
    I(f)=(b-a)f(x_0)+\bar{R}_0(f), \quad x_0=\frac{a+b}{2}
\end{align}

中点公式的误差:
\begin{align}
    \bar{R}_0[f]=\int_a^b f[x_0,x](x-x_0)\mathrm{d}x=\frac{1}{3}h^3f''(\xi), \quad h=\frac{b-a}{2}
\end{align}

\subsubsection{Hermite积分法}

类似Newton-Cotes公式的思想,采用更高阶的近似:

在区间$[a,b]$上,用f的三次Hermite插值近似$f(x)$,再对插值多项式进行积分,即可得到Hermite积分公式。

仅在区间端点上采样时,得到下式:
\begin{align}
    I(f)=\frac{b-a}{2}[f(a)+f(b)]+\frac{(b-a)^2}{12}[f'(a)-f'(b)]+\frac{1}{720}(b-a)^5 f^{(4)}(\xi)
\end{align}





\subsubsection {复合求积方法}
在区间内取n个采样点,\textbf{不要求均匀},即取
\begin{align}
    a=x_0<x_1<\ldots <x_n=b
\end{align}
并记第k个区间长度为\textbf{下一个点减去这一个点}:
\begin{align}
    h_k = x_{k+1} - x_{k}, \quad k=0,1,\ldots,n-1
\end{align}


同时,为了统一表示形式,约定$h_{-1}=h_n=0$。




\paragraph{复合中点公式}
\begin{align}
    I(f)=\sum_{k=0}^{n-1} h_k f\left(\frac{x_k+x_{k+1}}{2}\right) + \bar{R}_n(f)
\end{align}

机械求积公式形式:
\begin{align}
    H_n(f)=\sum_{k=0}^{n} A_k^{(n)} f(x_k)\\
    A_k=h_k
\end{align}


\paragraph {复合梯形公式}
\begin{align}
    I(f)=\sum_{k=0}^{n-1} \frac{h_k}{2}(f(x_k)+f(x_{k+1})) + R_n(f)
\end{align}

机械求积公式形式(约定$h_{-1}=h_n=0$):
\begin{align}
    T_n(f)=\sum_{k=0}^{n} A_k^{(n)} f(x_k)\\
    A_k=\frac{h_{k-1}+h_k}{2}
\end{align}

误差分析:
\begin{align}
    R_n(f)=&-\frac{1}{12}\sum_{k=0}^{n-1}h_k^3f''(\xi_k),\quad \xi_k\in [x_k,x_{k+1}]\\
    =&-\frac{1}{12}\sum_{k=0}^{n-1}h_k^3 f''(\eta),\quad \eta\in [a,b]
\end{align}
设$h=\max_{0\leq k \leq n-1} h_k$,则
\begin{align}
    |R_n(f)| \leq \frac{b-a}{12} h^2  |f''(\eta)|
\end{align}

递推关系:
\begin{align}
    T_{2n}(f)=\frac{T_n(f)+H_n(f)}{2}
\end{align}
即在每个积分区间上增加一个节点后,复合梯形公式即为增加节点前的复合梯形公式与复合中点公式的平均值。


\paragraph{改造复合梯形公式}
使用Hermite积分法,考虑等距节点的复合梯形公式,可以将复合梯形公式的积分精度改造到四阶:
\begin{align}
    I(f)=&\sum_{k=0}^{n-1} \frac{h_k}{2}[f(x_k)+f(x_{k+1})] + \frac{h_k^2}{12}[f'(x_k)-f'(x_{k+1})] + \frac{1}{720}(b-a)^5 f^{(4)}(\xi)\\
\end{align}

机械求积公式:
\begin{align}
    \tilde{T}_n(f)=&\sum_0^n A_k^{(n)} f(x_k)\\
    A_k=&\begin{cases}
        \frac{5}{12}h,\quad k=0,n\\
        \frac{13}{12}h,\quad k=1,n-1\\
        h,\quad \text{其他}
    \end{cases}
\end{align}



\paragraph {复合 Simpson 公式:一致稳定的求积公式}
\begin{align}
    I(f)=&\sum_{k=0}^{n-1} \frac{h_k}{6}[f(x_k)+4f(\frac{x_k+x_{k+1}}{2})+f(x_{k+1})] + R_n(f)
\end{align}

机械求积公式为
\begin{align}
    S_{2n}(f)=\frac{1}{3}T_n(f)+\frac{2}{3}H_n(f)=\frac{4T_{2n}(f)-T_n(f)}{3}
\end{align}

误差分析:
\begin{align}
    |R_{2n}(f)| \leq \frac{b-a}{2880}h^4|f^{(4)}(\eta)|,\quad \eta\in[a,b]
\end{align}
其中$h=\max_{0\leq k \leq n-1} h_k$。







\subsubsection {Romberg加速方法}

目的:在提升积分区间取点数时,充分利用利用已有的积分结果,通过外插法提升积分精度。

\paragraph{外插法}

在取等距节点时,成立欧拉-麦克劳林公式:
\begin{theorem}
    [Euler-Maclaurin 公式]
    设$f\in C^{2m+2}[a,b]$,则有
    \begin{align}
        I(f)=&T_n(f)+\sum_{k=1}^m \frac{B_{2k}}{(2k)!} h^{2k} [f^{(2k-1)}(b)-f^{(2k-1)}(a)] + R_{m+1}\\
    \end{align}
    其中,$T_n(f)$为复合梯形公式,$h=\frac{b-a}{n}$,$B_{2k}$为Bernoulli数,且
    \begin{align}
        r_{m+1} = -\frac{B_{2m+2}}{(2m+2)!}(b-a) f^{(2m+2)}(\eta) h^{2m+2},\quad \eta\in [a,b]
    \end{align}
\end{theorem}

由Euler-Maclaurin公式可知(只要将上式的级数移到左侧),复合梯形公式可以写成关于h的渐进级数形式,且0次项即为积分结果:
\begin{align}
    T_n(f)\sim& \tau_0+\tau_1h^2+\tau_2h^4+\ldots,\\
    \tau_0&=I(f)
\end{align}

在逐步提升采样节点密度时,我们总是采用等距节点,则提升密度过程可以导出一个关于h的单调递减序列:
\begin{align}
    h_i=\frac{b-a}{n_i},\quad n_i\text{单调递增}
\end{align}

于是可以考虑一个关于$h^2$的m次插值多项式(具体m取决于精度要求):
\begin{align}
    \tilde{T}_{mm}=a_0+a_1 h^2 + a_2 h^4 + \ldots + a_m h^{2m}\\
    \tilde{T}_{mm}(h_i)=T(h_i), \quad i=0,1,\ldots,m
\end{align}
于是将采样点的增加转化为了插值点的增加。

\paragraph{Neville算法:外插法的计算}
记$\tilde{T}_{ik}$为关于$h^2$的k次插值多项式,满足的插值条件为最后的k+1个$h_i$采样点:
\begin{align}
    \tilde{T}_{ik}(h_j)=T(h_j),\quad j=i-k,i-k+1,\ldots i.
\end{align}

则可以通过低一阶的插值多项式构造高一阶的插值多项式:
\begin{align}
    \tilde{T}_{ik}(h)=\frac{(h^2-h_{i-k})^2 \tilde{T}_{i,k-1}(h)+(h_i^2-h^2)\tilde{T}_{i-1,k-1}(h)}{h_i^2 - h_{i-k}^2}
\end{align}

我们仅关注这一递推关系在$h=0$处的取值,于是得到\textcolor{blue}{(重要递推关系,需记忆!)}
\begin{align}
    T_{ik}=T_{i,k-1}+\frac{T_{i,k-1}-T_{i-1,k-1}}{\dfrac{h_{i-k}^2}{h_i^2}-1}
\end{align}

插值多项式在0处的值$T_{ik}$有一个需要注意的性质:
\begin{itemize}
    \item $T_{i0}=T_{n_i}$,即0阶插值多项式即为对应h的复合梯形公式积分结果,或取$n_i$个等距节点时的复合梯形公式积分结果
\end{itemize}


\paragraph {Romberg 方法}
对于最简单的二分加密过程,有$n_i=2^i$,则$h_i=\frac{b-a}{2^i}$。

于是递推式化为
\begin{align}
    T_{ik}=T_{i,k-1}+\frac{T_{i,k-1}-T_{i-1,k-1}}{4^k-1}
\end{align}

特别地,当k=1时,有:
\begin{align}
    T_{i1}=\frac{4T_{i0}-T_{i-1,0}}{3}=\frac{4T_{2n}-T_n}{3}
\end{align}
即为Simpson公式。

\textbf{Romberg方法的列表求解}
见例题。逐步外推过程。

\textbf{Romberg方法的代数精度}

若用Romberg方法计算出一个$T_{mm}$,并将其作为积分结果,则该积分方法的代数精度至少为$2m+1$。




\subsubsection {Gauss 型求积}
\begin{definition}
    若$x_1,\ldots,x_n$取为权函数$\rho(x)$对应的正交多项式$\phi_n(x)$的n个不同实根,则所得求积公式
    \begin{align}
        I_n(f)=\sum_{k=1}^nA_k^{(n)}f(x_k),\quad A_k^{(n)}=I(L_k^{(n)})=\int_a^b L_k^{(n)}(x)\rho(x)\mathrm{d}x
    \end{align}
    有2n-1阶代数精度,相应的公式称为Gauss公式,$x_k$称为Gauss点。
\end{definition}

\paragraph{Gauss积分公式的基本性质}
\begin{itemize}
    \item Gauss积分公式是一致稳定的
    \item Gauss积分公式是收敛的,即当采样点个数趋于无穷大时,积分结果趋于真实值
\end{itemize}

\paragraph{设计Gauss积分公式的步骤}
\begin{itemize}
    \item 求正交多项式$\phi_n$
    \item 求正交多项式的n个不同实根,作为Gauss点
    \item 计算积分系数$A_k=I(L_k)$
    \item 得到求积公式
    \begin{align}
        I_n(f)=\sum_{k=1}^n A_k f(x_k)
    \end{align}
\end{itemize}

\paragraph{Gauss积分公式的应用}
应用优势区:函数光滑性足够好。如果函数光滑性不太好,可根据奇点分片。

两种典型积分:
\begin{itemize}
    \item Gauss-Legendre 求积:适用于$\rho(x)=1$的情形
    \begin{align}
        I(f)=\int_{-1}^1 f(x)\mathrm{d}x 
    \end{align}
    \item Gauss-Chebyshev 求积:适用于$\rho(x)=\frac{1}{\sqrt{1-x^2}}$的情形
    \begin{align}
        I(f)=\int_{-1}^1 \frac{f(x)}{\sqrt{1-x^2}}\mathrm{d}x
    \end{align}
\end{itemize}

在一般积分区间上,可以通过线性变换将积分区间映射到$[-1,1]$上:
\begin{align}
    I(f)=\int_a^bf(x)\mathrm{d}x=\frac{b-a}{2}\int_{-1}^1 f \left(\frac{b-a}{2}t+\frac{a+b}{2}\right)\mathrm{d}t
\end{align}






\subsubsection {特殊积分处理}

\paragraph {奇异积分}
常用方法:
\begin{itemize}
    \item 变量替换消奇点:如$\int_0^1x^{-1/n}f(x)\mathrm{d}x=n\int_0^1f(t^n)t^{n-2}\mathrm{d}t$
    \item 奇性分离:将函数分为解析可积的有奇性部分和一个光滑部分,解析处理奇性部分,数值处理光滑部分
    \item 无穷区间:在Laguerre多项式零点或者Hermite多项式零点进行Gauss积分
\end{itemize}

\paragraph {振荡积分}

对于形如
\begin{align}
    \int_a^b f(x)e^{i\omega x}\mathrm{d}x
\end{align}
的问题,常用两种方法处理:

\begin{itemize}
    \item 将问题区域分片,用高精度的分片多项式插值函数g逼近f,再(解析或者半解析地)准确计算g的积分
    \item 若f为足够光滑的周期函数,则可用高次三角函数近似
\end{itemize}




\subsubsection {高维积分}




\paragraph {蒙特卡洛方法}





\section {数值微分}

\subsection {基础方法}
\subsubsection {向前差商:误差$O(h)$}
\begin{align}
    f'(x)=\frac{f(x+h)-f(x)}{h}+O(h)=f[x,x+h]+O(h)
\end{align}

\subsubsection {向后差商:误差$O(h)$}
\begin{align}
    f'(x)=\frac{f(x)-f(x-h)}{h}+O(h)=f[x-h,x]+O(h)
\end{align}

\subsubsection {中心差商:误差$O(h^2)$}
\begin{align}
    f'(x)=\frac{f(x+h)-f(x-h)}{2h}+O(h^2)=f[x-h,x+h]+O(h^2)
\end{align}

高阶导数:
\begin{align}
    f''(x)=&\frac{f(x+h)-2f(x)+f(x-h)}{h^2}+O(h^2)=2f[x-h,x,x+h]+O(h^2)\\
\end{align}


\subsection {高精度方法}
\subsubsection {插值型数值微分}
思路:设函数g是满足在$\{x_k\}_{k=0}^n$上插值条件的f的近似函数,插值型求导方法即令
\begin{align}
    f^{(n)}(x)\approx g^{(n)}(x)
\end{align}
作为f的导数的近似。

误差分析:
\begin{align}
    f(x)-p_n(x)=& f[x_0,x_1,\ldots,x_n,x]w_{n+1}(x)\\
    w_{n+1}(x)=&(x-x_0)(x-x_1)\ldots(x-x_n)\\
\end{align}
求导可得
\begin{align}
    f'(x_k)-p_n'(x_k)=& f[x_0,x_1,\ldots,x_n,x_k]w_{n+1}'(x_k)\\
    f''(x_k)-p_n''(x_k)=& 2f[x_0,x_1,\ldots,x_n,x_k,x_k]w_{n+1}'(x_k) + f[x_0,x_1,\ldots,x_n,x_k]w_{n+1}''(x_k)\\
\end{align}
即通常情况下,一阶导有$O(h^n)$误差,二阶导有$O(h^{n-1})$误差。

\paragraph{线性插值公式}
\begin{align}
    p_1(x)=&f[x_0]+f[x_0,x_1](x-x_0)\\
    p_1'(x)=& f[x_0,x_1]\\
\end{align}

\paragraph{抛物插值}
\begin{align}
    p_2(x)=&f[x_0]+f[x_0,x_1](x-x_0)+f[x_0,x_1,x_2](x-x_0)(x-x_1)\\
    p_2'(x)=& f[x_0,x_1]+f[x_0,x_1,x_2](2x - x_0 - x_1)\\
    p_2''(x)=& 2f[x_0,x_1,x_2]
\end{align}

等距节点时,有 
\begin{align}
    f'(x_0)=&\frac{1}{2h}[-3f(x_0)+4f(x_1)-f(x_2)]+O(h^2)\\
    f'(x_1)=&\frac{1}{2h}[-f(x_0)+f(x_2)]+O(h^2)\\
    f'(x_2)=&\frac{1}{2h}[f(x_0)-4f(x_1)+3f(x_2)]+O(h^2)\\
    f''(x_0)=&\frac{1}{h^2}[f(x_0)-2f(x_1)+f(x_2)]+O(h^2)\\
\end{align}
Note:由于抛物插值有一定对称性,中心点的二阶导精度可以加一阶。


\subsubsection {Richardson 外推加速}
类似数值积分的外推法,若有多个不同区间长度下的中点微分公式近似结果,则可进行数值外推

进行二分加密时,公式如下:
\begin{align}
    G_1(h)=&G(h)\\
    G_{m+1}(h)=&G_m(h/2)+\frac{G_m(h/2)-G_m(h)}{2^{m}-1}=\frac{4^mG_m(h/2)-G_m(h)}{4^m-1}
\end{align}


\chapter {非线性方程求根 (Nonlinear Equations)}


求解 $f(x)=0$,其中 $f: [a,b] \to \mathbb{R}$ 为连续函数。
\begin{itemize}
    \item \textbf{二分法}:基于介值定理,线性收敛但可靠。
    \item \textbf{迭代法}:将问题转化为等价不动点问题 $x = \phi(x)$。利用迭代格式$x^{(k+1)} = \phi(x^{(k)})$得到。
\end{itemize}






\section{不动点迭代法 (Fixed Point Iteration)}
\subsection{基本概念}
\begin{itemize}
    \item \textbf{不动点}:若 $\xi = \phi(\xi)$,则称 $\xi$ 为 $\phi$ 的不动点。
    \item \textbf{等价性}:$f(x)=0 \iff x=\phi(x)$。例如取 $\phi(x) = x - c f(x)$。
\end{itemize}

\subsection{收敛理论}

%局部收敛性的定义

\begin{theorem}[Brouwer 不动点定理]
    设 $\phi: D \to \mathbb{R}^n$ 是凸紧集 $D$ 上的连续映射,且 $\phi(D) \subseteq D$。则 $\phi$ 在 $D$ 内至少存在一个不动点。
    (注:在一维情形下,即若 $\phi: [a,b] \to [a,b]$ 连续,则必有不动点。)
\end{theorem}

\begin{theorem}[压缩映射原理 / Banach Fixed Point Theorem]
    设 $\phi: D \to D$ 是完备度量空间上的压缩映射,即存在常数 $0 \le L < 1$,使得:
    \[
        ||\phi(x) - \phi(y)|| \le L ||x-y||, \quad \forall x, y \in D
    \]
    则:
    \begin{enumerate}
        \item $\phi$ 在 $D$ 内存在\textbf{唯一}不动点 $x^*$。
        \item 对任意初值 $x_0 \in D$,迭代序列 $x_{k+1} = \phi(x_k)$ 均收敛于 $x^*$。
        \item \textbf{误差估计}:
        \[
            ||x_k - x^*|| \le \frac{L^k}{1-L} ||x_1 - x_0||
        \]
    \end{enumerate}
\end{theorem}
\begin{proofstep}{压缩映射证明思路}

    1. 证明 $\{x_k\}$ 是柯西序列:$||x_{k+m} - x_k|| \le \frac{L^k}{1-L}||x_1-x_0||$。

    2. 利用空间完备性,得极限 $x^*$ 存在。

    3. 利用连续性证明 $x^*$ 是不动点。

    4. 利用反证法证明唯一性。
\end{proofstep}




\subsection{局部收敛性定理}
\begin{theorem}
    设 $x^*$ 为 $\phi(x)$ 的不动点,且 $\phi'(x)$ 在 $x^*$ 邻域内连续。若
    \[
        |\phi'(x^*)| < 1
    \]
    则不动点迭代在 $x^*$ 的某个邻域内局部收敛。
\end{theorem}

\begin{proofstep}{}
    由微分中值定理:
    \[
        x_{k+1} - x^* = \phi(x_k) - \phi(x^*) = \phi'(\xi_k)(x_k - x^*)
    \]
    其中 $\xi_k$ 介于 $x_k$ 和 $x^*$ 之间。
    取绝对值:
    \[
        |x_{k+1} - x^*| = |\phi'(\xi_k)| |x_k - x^*|
    \]
    若 $|\phi'(x^*)| < 1$,由于 $\phi'$ 连续,存在邻域使得 $|\phi'(x)| \le L < 1$。
    于是 $|x_{k+1} - x^*| \le L |x_k - x^*|$,即收敛。
\end{proofstep}



\subsection{收敛阶 (Order of Convergence)}
\begin{definition}
    若迭代序列 $\{x_k\}$ 收敛于 $x^*$,且
    \[
        \lim_{k \to \infty} \frac{|x_{k+1} - x^*|}{|x_k - x^*|^p} = C \ne 0
    \]
    则称该迭代过程是 $p$ 阶收敛的。
\end{definition}
\begin{theorem}[收敛阶判定]
    设 $\phi(x)$ 在 $x^*$ 附近足够光滑。
    \begin{itemize}
        \item 若 $0 < |\phi'(x^*)| < 1$,则为\textbf{线性收敛} ($p=1$)。
        \item 若 $\phi'(x^*) = 0$ 且 $\phi''(x^*) \ne 0$,则为\textbf{平方收敛} ($p=2$)。
        \item 一般地,若 $\phi^{(k)}(x^*) = 0 (k=1,\dots,p-1)$ 且 $\phi^{(p)}(x^*) \ne 0$,则为 $p$ 阶收敛。
    \end{itemize}
\end{theorem}



\begin{proofstep}{推导过程 (利用泰勒展开)}
    将 $\phi(x_k)$ 在 $x^*$ 处展开:
    \[
        x_{k+1} = \phi(x_k) = \phi(x^*) + \phi'(x^*)(x_k - x^*) + \frac{\phi''(x^*)}{2!}(x_k - x^*)^2 + \dots
    \]
    注意到 $\phi(x^*) = x^*$,代入误差 $e_k = x_k - x^*$:
    \[
        e_{k+1} = \phi'(x^*) e_k + \frac{\phi''(x^*)}{2!} e_k^2 + \dots + \frac{\phi^{(p)}(x^*)}{p!} e_k^p + O(e_k^{p+1})
    \]
    \begin{itemize}
        \item 若 $\phi'(x^*) \ne 0$,则 $e_{k+1} \approx \phi'(x^*) e_k$,即线性收敛。
        \item 若 $\phi'(x^*) = 0$ 但 $\phi''(x^*) \ne 0$,则 $e_{k+1} \approx \frac{\phi''(x^*)}{2} e_k^2$,即平方收敛。
    \end{itemize}
\end{proofstep}



\subsection{收敛域 (Convergence Domain)}
\begin{definition}[收敛域]
    对于给定的不动点 $x^*$,使得迭代序列 $x_{k+1} = \phi(x_k)$ 收敛于 $x^*$ 的所有初始点 $x_0$ 的集合,称为该不动点的收敛域。
\end{definition}

\subsubsection{求解收敛域}
理论上求精确的收敛域很难,但通常我们需要找到一个\textbf{保证收敛的区间}。
\begin{enumerate}
    \item \textbf{解不等式}:
    首先解不等式 $|\phi'(x)| < 1$。
    这将给出一个或多个开区间 $I$。在这些区间内,$\phi$ 是压缩的。
    
    \item \textbf{验证封闭性}:
    选取上述区间中包含不动点 $x^*$ 的最大区间 $(a, b)$。
    检查该区间是否满足:
    \[
        \forall x \in [a, b], \quad \phi(x) \in [a, b]
    \]
    如果是,则 $[a, b]$ 是一个收敛区间。

\end{enumerate}





\section{牛顿法 (Newton's Method)}
\subsection{构造}
\begin{itemize}
    \item \textbf{几何视角}:在当前点 $(x_k, f(x_k))$ 作切线,取切线与 $x$ 轴交点为下一次迭代值 $x_{k+1}$。
    \item \textbf{泰勒展开视角}:将 $f(x)$ 在 $x_k$ 处展开,忽略高阶项:
    \[
        f(x) \approx f(x_k) + f'(x_k)(x - x_k) = 0 \implies x = x_k - \frac{f(x_k)}{f'(x_k)}
    \]
\end{itemize}
\textbf{迭代公式}:
\[
    x_{k+1} = x_k - \frac{f(x_k)}{f'(x_k)}
\]

\subsection{ 收敛性分析}
特殊的不动点迭代, $\phi(x) = x - \frac{f(x)}{f'(x)}$。
\begin{theorem}[局部二阶收敛性]
    设 $x^*$ 是 $f(x)=0$ 的单根(即 $f(x^*)=0, f'(x^*) \ne 0$),且 $f(x)$ 在 $x^*$ 邻域内二阶连续可微。则存在 $x^*$ 的邻域,使得对任意 $x_0$ 在此邻域内,牛顿法生成的序列 $\{x_k\}$ 收敛于 $x^*$,且为\textbf{二阶收敛}。
\end{theorem}
\begin{proofstep}{证明}

    1. 考察迭代函数 $\phi(x) = x - \frac{f(x)}{f'(x)}$ 的导数:
    \[
        \phi'(x) = 1 - \frac{(f'(x))^2 - f(x)f''(x)}{(f'(x))^2} = \frac{f(x)f''(x)}{(f'(x))^2}
    \]
    2. 在单根 $x^*$ 处,$f(x^*)=0$,故 $\phi'(x^*) = 0$。

    3. 根据不动点收敛阶判定定理,由于 $\phi'(x^*) = 0$,只要 $\phi''(x^*) \ne 0$,收敛阶至少为 2。
    
    \textbf{误差渐进式}:利用泰勒展开可得
    \[
        e_{k+1} \approx \frac{f''(x^*)}{2f'(x^*)} e_k^2 \implies \lim_{k\to\infty} \frac{|e_{k+1}|}{|e_k|^2} = \left| \frac{f''(x^*)}{2f'(x^*)} \right|
    \]
\end{proofstep}

\subsection{ 重根情形修正}
若 $x^*$ 是 $f(x)=0$ 的 $m$ 重根 ($m > 1$),则 $f(x) = (x-x^*)^m g(x)$,且 $g(x^*) \ne 0$。
此时,$f(x)$的前$m-1$阶导数为0.
标准牛顿法退化为线性收敛,收敛因子为 $1 - 1/m$。


\textbf{修正牛顿法}:为了恢复二阶收敛,可使用:
\[
    x_{k+1} = x_k - m \frac{f(x_k)}{f'(x_k)}
\]
m为不动点的重数。


另:


令$u:= \frac{f}{f'}$,对$u$使用牛顿迭代法。


\section{加速方法}




\subsection{割线法 (Secant Method)}
\textbf{构造思想}:
为了避免牛顿法中计算导数 $f'(x_k)$ 的代价,用差商来近似导数:
\[
    f'(x_k) \approx \frac{f(x_k) - f(x_{k-1})}{x_k - x_{k-1}}
\]
将此代入牛顿公式,得到\textbf{迭代公式}:
\[
    x_{k+1} = x_k - f(x_k) \frac{x_k - x_{k-1}}{f(x_k) - f(x_{k-1})}
\]
此方法需要\textbf{两个初值} $x_0, x_1$。

\textbf{收敛性分析}:
\begin{theorem}[割线法收敛阶]
    若 $x^*$ 是单根,且 $f$ 二阶连续可微,初值充分接近 $x^*$,则割线法是\textbf{超线性收敛}的,收敛阶为黄金分割比:
    \[
        p = \frac{1+\sqrt{5}}{2} \approx 1.618
    \]
\end{theorem}
\begin{proofstep}{误差方程推导}
    设误差 $e_k = x_k - x^*$。割线法的误差关系满足:
    \[
        e_{k+1} \approx C e_k e_{k-1}
    \]
    设 $|e_{k+1}| \sim |e_k|^p$,则 $|e_k| \sim |e_{k-1}|^p \implies |e_{k-1}| \sim |e_k|^{1/p}$。
    代入误差方程:
    \[
        |e_k|^p \sim |e_k| \cdot |e_k|^{1/p} = |e_k|^{1 + 1/p}
    \]
    比较指数:$p = 1 + \frac{1}{p} \implies p^2 - p - 1 = 0$。
    解得正根 $p = \frac{1+\sqrt{5}}{2} \approx 1.618$。
\end{proofstep}


\subsection{Aitken $\Delta^2$ 加速}
\textbf{构造}:
假设序列 $\{x_k\}$ 线性收敛于 $x^*$,且渐进误差常数为 $C$:
\[
    x_k - x^* \approx C(x_{k-1} - x^*)
\]
我们有三个连续点 $x_{k}, x_{k+1}, x_{k+2}$。近似认为 $C$ 在这几步是不变的:
\[
    \frac{x_{k+2} - x^*}{x_{k+1} - x^*} \approx \frac{x_{k+1} - x^*}{x_k - x^*} \approx C
\]
消去 $C$,解出 $x^*$ 的估计值 $\hat{x}_k$:
\[
    (x_{k+2} - x^*)(x_k - x^*) \approx (x_{k+1} - x^*)^2
\]
解得 \textbf{Aitken 加速公式}:
\[
    \hat{x}_k = x_k - \frac{(x_{k+1} - x_k)^2}{x_{k+2} - 2x_{k+1} + x_k} = x_k - \frac{(\Delta x_k)^2}{\Delta^2 x_k}
\]

即使 $\{x_k\}$ 只是线性收敛,$\{\hat{x}_k\}$ 的收敛速度通常会快得多(高阶收敛)。

\subsection{Steffensen 方法}
Aitken 加速实质上是原不动点迭代步骤完成后,进一步提高精度的处理。

Steffensen 方法则将二阶差分直接结合到不动点迭代过程中,构成一种无需导数的二阶收敛算法。

\textbf{构造过程}:
对于不动点迭代 $x = \phi(x)$,每一步迭代如下:

1. 给定 $x_k$,计算两步辅助点:
   \[ y_k = \phi(x_k), \quad z_k = \phi(y_k) \]
2. 将 $x_k, y_k, z_k$ 视为 Aitken 加速中的三个点,直接计算加速值作为 $x_{k+1}$:
   \[
       x_{k+1} = x_k - \frac{(y_k - x_k)^2}{z_k - 2y_k + x_k}
   \]
   其中分母 $z_k - 2y_k + x_k = \phi(\phi(x_k)) - 2\phi(x_k) + x_k$。
   

\textbf{Steffensen方法对应的迭代函数} 

本质上是将单步不动点迭代函数$\phi$换成了
$\psi(x)$:
   \[ \psi(x) = x - \frac{(\phi(x) - x)^2}{\phi(\phi(x)) - 2\phi(x) + x}  = \frac{x\phi(\phi(x)) - [\phi(x)]^2}{\phi(\phi(x)) - 2\phi(x) + x} \]
   即 $x_{k+1} = \psi(x_k)$。

\textbf{收敛性分析}:
\begin{theorem}
    Steffensen 方法实际上等价于对函数 $g(x) = x - \phi(x)$ 使用牛顿法的一个近似形式(用差商代替导数)。
    若 $\phi(x)$ 三阶连续可微,且 $\phi'(x^*) \ne 1$,则 Steffensen 方法是\textbf{二阶收敛}的。
    \[
        \lim_{k \to \infty} \frac{|e_{k+1}|}{|e_k|^2} = C \ne 0
    \]
\end{theorem}

不需要计算导数,却能达到牛顿法的二阶收敛速度。



\section{非线性方程组}

\subsection{基础理论与向量值微积分}
\subsubsection{非线性方程组}
求解 $\mathbf{F}(\mathbf{x}) = \mathbf{0}$,其中 $\mathbf{F}: D \subset \mathbb{R}^n \to \mathbb{R}^n$ 为向量值函数,$\mathbf{x} = (x_1, \dots, x_n)^T$。
\[
    \mathbf{F}(\mathbf{x}) = \begin{pmatrix} f_1(x_1, \dots, x_n) \\ \vdots \\ f_n(x_1, \dots, x_n) \end{pmatrix}
\]

\subsubsection{向量值函数的导数 (Jacobian Matrix)}
\begin{definition}[Jacobian 矩阵]
    设 $\mathbf{F}$ 在 $\mathbf{x}$ 处可微,其导数(Jacobian 矩阵)定义为:
    \[
        \mathbf{J}(\mathbf{x}) = \mathbf{F}'(\mathbf{x}) = \begin{pmatrix}
            \frac{\partial f_1}{\partial x_1} & \cdots & \frac{\partial f_1}{\partial x_n} \\
            \vdots & \ddots & \vdots \\
            \frac{\partial f_n}{\partial x_1} & \cdots & \frac{\partial f_n}{\partial x_n}
        \end{pmatrix} \in \mathbb{R}^{n \times n}
    \]
\end{definition}

\subsubsection{多元 Taylor 展开}
\begin{theorem}[多元 Taylor 公式]
    设 $\mathbf{F}$ 二阶连续可微,则在 $\mathbf{x}$ 的邻域内:
    \[
        \mathbf{F}(\mathbf{x} + \mathbf{h}) = \mathbf{F}(\mathbf{x}) + \mathbf{J}(\mathbf{x})\mathbf{h} + O(||\mathbf{h}||^2)
    \]
    其中线性主部 $\mathbf{J}(\mathbf{x})\mathbf{h}$ 是 $\mathbf{F}$ 在 $\mathbf{x}$ 处的微分。
\end{theorem}

\subsubsection{多元压缩映射原理}
\begin{theorem}[Banach Fixed Point Theorem in $\mathbb{R}^n$]
    设 $D \subset \mathbb{R}^n$ 是闭集,$\mathbf{\Phi}: D \to D$ 是压缩映射,即存在 $0 \le L < 1$ 使得:
    \[
        ||\mathbf{\Phi}(\mathbf{x}) - \mathbf{\Phi}(\mathbf{y})|| \le L ||\mathbf{x} - \mathbf{y}||, \quad \forall \mathbf{x}, \mathbf{y} \in D
    \]
    则 $\mathbf{\Phi}$ 在 $D$ 内存在唯一不动点 $\mathbf{x}^*$。
\end{theorem}
\begin{corollary}[局部收敛判据]
    设 $\mathbf{x}^*$ 是不动点。若 $\rho(\boldsymbol{\Phi}'(\mathbf{x}^*)) < 1$(谱半径小于1),则不动点迭代 $\mathbf{x}_{k+1} = \mathbf{\Phi}(\mathbf{x}_k)$ 在 $\mathbf{x}^*$ 邻域内局部收敛。
\end{corollary}

\subsection{多元牛顿法 (Newton's Method for Systems)}
\textbf{构造思想}:
对 $\mathbf{F}(\mathbf{x}) = \mathbf{0}$ 在 $\mathbf{x}_k$ 处进行线性化(Taylor 展开保留一项):
\[
    \mathbf{F}(\mathbf{x}) \approx \mathbf{F}(\mathbf{x}_k) + \mathbf{J}(\mathbf{x}_k)(\mathbf{x} - \mathbf{x}_k) = \mathbf{0}
\]
解出 $\mathbf{x}$ 即为下一次迭代值 $\mathbf{x}_{k+1}$。

\textbf{迭代格式}:
\[
    \mathbf{x}_{k+1} = \mathbf{x}_k - [\mathbf{J}(\mathbf{x}_k)]^{-1} \mathbf{F}(\mathbf{x}_k)
\]
实际计算中,\textbf{不求逆矩阵},而是求解线性方程组:

1. 计算残差 $\mathbf{r}_k = -\mathbf{F}(\mathbf{x}_k)$。

2. 求解线性方程组 $\mathbf{J}(\mathbf{x}_k) \mathbf{s}_k = \mathbf{r}_k$ (得到修正量 $\mathbf{s}_k$)。

3. 更新 $\mathbf{x}_{k+1} = \mathbf{x}_k + \mathbf{s}_k$。


\subsubsection{多元牛顿法的收敛性分析}

\begin{theorem}[多元牛顿法局部二阶收敛]

    设 $\mathbf{F}: D \subset \mathbb{R}^n \to \mathbb{R}^n$ 满足以下条件:
    \begin{enumerate}
        \item 存在零点 $\mathbf{x}^* \in D$,即 $\mathbf{F}(\mathbf{x}^*) = \mathbf{0}$。
        \item $\mathbf{F}$ 在 $\mathbf{x}^*$ 的邻域内二阶连续可微($C^2$)。
        \item Jacobian 矩阵 $\mathbf{J}(\mathbf{x}^*)$ 非奇异(即 $\det(\mathbf{J}(\mathbf{x}^*)) \ne 0$,$\mathbf{x}^*$ 为单根)。
    \end{enumerate}
    则存在 $\mathbf{x}^*$ 的邻域 $S$,使得对任意初始点 $\mathbf{x}_0 \in S$,牛顿迭代产生的序列 $\{\mathbf{x}_k\}$ 收敛于 $\mathbf{x}^*$,且收敛阶至少为 2(Quadratic Convergence)。即存在常数 $C > 0$,使得:
    \[
        \|\mathbf{x}_{k+1} - \mathbf{x}^*\| \le C \|\mathbf{x}_k - \mathbf{x}^*\|^2
    \]
\end{theorem}

\begin{proofstep}[证明思路]

    \textbf{1. 误差递推关系}
    设 $\mathbf{e}_k = \mathbf{x}_k - \mathbf{x}^*$ 为第 $k$ 步的误差。
    由牛顿法迭代公式:
    \[ \mathbf{x}_{k+1} = \mathbf{x}_k - [\mathbf{J}(\mathbf{x}_k)]^{-1} \mathbf{F}(\mathbf{x}_k) \]
    两边减去 $\mathbf{x}^*$,得:
    \[ \mathbf{e}_{k+1} = \mathbf{e}_k - [\mathbf{J}(\mathbf{x}_k)]^{-1} \mathbf{F}(\mathbf{x}_k) \]
    
    \textbf{2.  Taylor 展开}

    将 $\mathbf{F}(\mathbf{x}^*)$ 在 $\mathbf{x}_k$ 处展开(注意方向是 $\mathbf{x}^* = \mathbf{x}_k - \mathbf{e}_k$):
    \[ \mathbf{0} = \mathbf{F}(\mathbf{x}^*) = \mathbf{F}(\mathbf{x}_k) + \mathbf{J}(\mathbf{x}_k)(\mathbf{x}^* - \mathbf{x}_k) + O(\|\mathbf{x}^* - \mathbf{x}_k\|^2) \]
    即:
    \[ \mathbf{F}(\mathbf{x}_k) = \mathbf{J}(\mathbf{x}_k)\mathbf{e}_k + O(\|\mathbf{e}_k\|^2) \]
    或者更精确地写成积分余项形式:
    \[ \mathbf{F}(\mathbf{x}_k) = \mathbf{J}(\mathbf{x}_k)\mathbf{e}_k - \int_0^1 (1-t) \mathbf{F}''(\mathbf{x}_k - t\mathbf{e}_k)(\mathbf{e}_k, \mathbf{e}_k) dt \]
    

    将 $\mathbf{F}(\mathbf{x}_k)$ 的展开式代入误差递推式:
    \[
    \begin{aligned}
        \mathbf{e}_{k+1} &= \mathbf{e}_k - [\mathbf{J}(\mathbf{x}_k)]^{-1} \bigl(\mathbf{J}(\mathbf{x}_k)\mathbf{e}_k + O(\|\mathbf{e}_k\|^2)\bigr) \\
        &= \mathbf{e}_k - \bigl(\mathbf{e}_k + [\mathbf{J}(\mathbf{x}_k)]^{-1} O(\|\mathbf{e}_k\|^2)\bigr) \\
        &= - [\mathbf{J}(\mathbf{x}_k)]^{-1} O(\|\mathbf{e}_k\|^2)
    \end{aligned}
    \]
    
    两边取范数:
    \[ \|\mathbf{e}_{k+1}\| \le \|[\mathbf{J}(\mathbf{x}_k)]^{-1}\| \cdot \|O(\|\mathbf{e}_k\|^2)\| \]
    

    由于 $\mathbf{J}(\mathbf{x})$ 在 $\mathbf{x}^*$ 处连续且非奇异,存在 $\mathbf{x}^*$ 的邻域 $S$ 和常数 $M_1$,使得对任意 $\mathbf{x} \in S$,$\mathbf{J}(\mathbf{x})$ 均可逆且 $\|[\mathbf{J}(\mathbf{x})]^{-1}\| \le M_1$(逆算子的一致有界性)。
    又因为 $\mathbf{F}$ 二阶可微,二阶导数在闭邻域上有界,即存在常数 $M_2$ 使得 Taylor 展开的余项满足 $\|O(\|\mathbf{e}_k\|^2)\| \le M_2 \|\mathbf{e}_k\|^2$。
    于是:
    \[ \|\mathbf{e}_{k+1}\| \le M_1 M_2 \|\mathbf{e}_k\|^2 = C \|\mathbf{e}_k\|^2 \]
    其中 $C = M_1 M_2$。
    
    \textbf{结论}:这就证明了误差是按照平方速度衰减的。只要初值 $\mathbf{x}_0$ 足够接近 $\mathbf{x}^*$,使得 $C\|\mathbf{e}_0\| < 1$,迭代就会收敛。
\end{proofstep}


\subsubsection{拟牛顿法}
为了避免计算昂贵的 Jacobian 矩阵及其逆,拟牛顿法使用近似矩阵 $B_k \approx \mathbf{J}(\mathbf{x}_k)$ 并满足\textbf{拟牛顿方程}:
\[
    B_{k+1}(\mathbf{x}_{k+1} - \mathbf{x}_k) = \mathbf{F}(\mathbf{x}_{k+1}) - \mathbf{F}(\mathbf{x}_k)
\]

\textbf{Broyden 秩1 方法}:
最常用的秩1更新公式:
\[
    B_{k+1} = B_k + \frac{(\mathbf{y}_k - B_k \mathbf{s}_k)\mathbf{s}_k^T}{\mathbf{s}_k^T \mathbf{s}_k}
\]
其中 $\mathbf{s}_k = \mathbf{x}_{k+1} - \mathbf{x}_k,\ \mathbf{y}_k = \mathbf{F}(\mathbf{x}_{k+1}) - \mathbf{F}(\mathbf{x}_k)$。



\subsubsection{往年题:牛顿法计算特征值与归一化特征向量}
\paragraph{特征值问题的非线性方程组转化}
对于 $n$ 阶实矩阵 $\mathbf{A} \in \mathbb{R}^{n \times n}$,特征值-特征向量问题 $\mathbf{A}\mathbf{x} = \lambda \mathbf{x} (\mathbf{x} \neq \mathbf{0})$ 结合归一化约束 $\mathbf{x}^T\mathbf{x} = 1$,可转化为求解非线性方程组 $\mathbf{F}(\mathbf{z}) = \mathbf{0}$,其中 $\mathbf{z} = (\mathbf{x}^T, \lambda)^T \in \mathbb{R}^{n+1}$,且:
\[
\mathbf{F}(\mathbf{z}) = \mathbf{F}(\mathbf{x}, \lambda) = \begin{pmatrix}
(\mathbf{A} - \lambda \mathbf{I})\mathbf{x} \\
\frac{1}{2}(\mathbf{x}^T\mathbf{x} - 1)
\end{pmatrix}
\]

\paragraph{Jacobian矩阵构造}
$\mathbf{F}(\mathbf{z})$ 的 $(n+1) \times (n+1)$ 阶Jacobian矩阵为:
\[
\mathbf{J}(\mathbf{z}) = \begin{pmatrix}
\mathbf{A} - \lambda \mathbf{I} & -\mathbf{x} \\
\mathbf{x}^T & 0
\end{pmatrix}
\]

\paragraph{牛顿迭代格式}


\tofill{以下构造过程就是一样的了,抄上面方程组牛顿迭代格式即可}




%%%%%%%%%%%%%%%%%%%%%%%%%%%%%%%%%%%%%%%%%%%%%%%%%%%%%%%%%%%%%%%%%%%%%%%%%%%%%%%%%



\chapter {常微分方程初值问题数值解法}

\section*{总结}
由于ODE的单步法与多步法的理论多有相似,因此将理论统一整理。

\subsection*{重要概念}


\begin{itemize}
    \item Lipschitz条件:$\|f(t,y_1)-f(t,y_2)\|\leq L\|y_1-y_2\|$
    \item 零稳定性:两组数值解的最大差值不超过二者初值差值的常数倍
    \item 强稳定性:多步法的系数矩阵范数不大于1
    \item 绝对稳定性:在给定的步长下,对试验方程的解是趋于0的
    \item 局部截断误差:单步法或多步法推进一步时产生的误差
    \item 相容性与相容阶:局部截断误差小于等于$Mh^{p+1}$,则具有p阶相容性。注意,由于整个数值解要推进$1/h$步,对局部截断误差的阶数要求要+1
    \item 收敛性与收敛阶:步长h趋于0时,整体误差按照$h^p$收敛到初值误差,则称具有p阶收敛性。
\end{itemize}




% \subsection*{单步法}
% \begin{itemize}
%     \item 一般形式
%     \item 求解方法
%     \begin{itemize}
%         \item Euler折线法(Euler公式)
%         \item 向后Euler法
%         \item 中点格式及其显式化
%         \item 梯形格式及其显式化
%     \end{itemize}
%     \item 一步误差
%     \item 局部截断误差 
%     \item 相容性与相容阶
% \end{itemize}

\section {通用理论}




\subsection {问题模型}

一阶初值问题提法:
\begin{align}
    \begin{cases}
        y' = f(t,y),\quad t\in [a,b],\quad y\in\boldsymbol{R}^d\\
        y(a) = y_0
    \end{cases}
\end{align}

半线性高阶常微分方程初值问题提法:
\begin{align}
    y^{(n)}+F(t,y,y',\ldots,y^{(n-1)})=0,\quad t\in [a,b]\\
    y(a)=y_0,\,y'(a)=y_0^{(1)},\ldots,y^{(n-1)}(a)=y_0^{(n-1)}
\end{align}

从一阶初值问题到高阶初值问题的转化:
考虑导数向量
\begin{align}
    Y=[y,y',\ldots,y^{(n-1)}]^T
\end{align}

则该导数向量满足矩阵关系:
\begin{align}
    Y'=&\begin{bmatrix}
        0 & I & 0 & \ldots & 0\\
        0 & 0 & I & \ldots & 0\\
        \vdots & \vdots & \vdots & \ddots & \vdots\\
        0 & 0 & 0 & \ldots & I\\
        0 & 0 & 0 & \ldots & 0
    \end{bmatrix}
Y + \begin{bmatrix}
        0\\
        0\\
        \vdots\\
        0\\
        -F(t,y,y',\ldots,y^{(n-1)})
    \end{bmatrix}\\
\end{align}


\subsection {数值解法前提:Lipschitz 条件}

对比:线性方程组求解的压缩映像原理;但这里\textbf{不}要求L<1。
\begin{definition}
    [Lipschitz 条件]
    设函数$f:[a,b]\times \boldsymbol{R}^d \to \boldsymbol{R}^d$,若存在常数$L>0$,使得对任意$t\in[a,b]$及任意$y_1,y_2\in \boldsymbol{R}^d$,都有
    \begin{align}
        ||f(t,y_1)-f(t,y_2)|| \leq L ||y_1 - y_2||
    \end{align}
    则称$f$满足Lipschitz条件,L为Lipschitz常数。

    当方程满足Lipschitz条件时,初值问题存在唯一解$y=y(t)\in C^1[a,b]$
\end{definition}


\subsection {基本概念}

\subsubsection{单步法与多步法的一般形式}
\begin{itemize}
    \item 单步法:从n计算出n+1
    \begin{align}
    y_{n+1}=y_n + h_{n+1} \phi(t_n,t_{n+1};y_n,y_{n+1};f)
\end{align}
    \item 多步法:从$(n,n+k-1)$共k个点计算出n+k
    \begin{align}
        \sum_{i=0}^k \alpha_i^{(n+k)}y_{n+i}=h_{n+k}\phi(t_n,t_{n+1},\ldots,t_{n+k};y_n,y_{n+1},\ldots,y_{n+k};f)
    \end{align}
    式中,$\alpha_k^{(n+k)}=1$
    \item 多步法的等价形式:

考虑k点数值向量
\begin{align}
    Y_n=[y_n,\,y_{n+1},\,\ldots,\,y_{n+k-1}]^T
\end{align}
则多步法可表示为
\begin{align}
    Y_{n+1}=\begin{bmatrix}
        0 & 1 & 0 & \ldots & 0\\
        0 & 0 & 1 & \ldots & 0\\
        \vdots & \vdots & \vdots & \ddots & \vdots\\
        0 & 0 & 0 & \ldots & 1\\
        -\alpha_0^{(n+k)} & -\alpha_1^{(n+k)} & -\alpha_2^{(n+k)} & \ldots & -\alpha_{k-1}^{(n+k)}
    \end{bmatrix}
    Y_n + h_{n+k} \begin{bmatrix}
        0\\
        0\\
        \vdots\\
        0\\
        \phi(t_n,t_{n+1},\ldots,t_{n+k};y_n,y_{n+1},\ldots,y_{n+k};f)
    \end{bmatrix}
\end{align}

\end{itemize}

对于单步法和多步法,增量函数$\phi$均需要满足以下三条性质:
\begin{itemize}
    \item 连续性:$\phi$在所有的(除了函数f外)的变量上连续
    \item 零值性:当$f=0$时,$\phi(\cdot;f)=0$
    \item Lipschitz条件:
    \begin{itemize}
        \item 对于单步法:
        \begin{align}
            \|\phi(t,\tau;u_1,v_1;f)-\phi(t,\tau;u_2,v_2;f)\|\leq L_f(\|u_1-u_2\|+\|v_1-v_2\|)
        \end{align}
        \item 对于多步法,需要满足一致Lipschitz条件:
        \begin{align}
            &\|\phi(t_n,t_{n+1},\ldots,t_{n+k};u_0,u_1,\ldots,u_k;f)-\phi(t_n,t_{n+1},\ldots,t_{n+k};v_0,v_1,\ldots,v_k;f)\| \\
            &\leq L_f \sum_{i=0}^k \|u_i - v_i\|
        \end{align}
    \end{itemize}
\end{itemize}


\subsubsection{多步法的特征多项式和稳定性多项式}

\begin{theorem}
    [多步法的特征多项式]
    设多步法的系数为$\alpha_i^{(n+k)}$,则其特征多项式定义为
    \begin{align}
    \rho(\xi)=\sum_{i=0}^k \alpha_i^{(n+k)} \xi^i
\end{align}

\end{theorem}


\begin{definition}
    [多步法的稳定性多项式]
    线性多步法用于求解试验方程$y'=\lambda y$得到 
    \begin{align}
        \sum_{i=0}^k (\alpha_i-h\lambda\beta_i)y_{n+i}=0
    \end{align}
    记 
    \begin{align}
        \mu=&h\lambda \\
        \Pi(\xi,\mu)=&\rho(\xi)-\mu\sigma(\xi)
    \end{align}
    为线性多步法的稳定性多项式。
\end{definition}

\subsubsection{零稳定性}
零稳定性考虑的是方程右端函数$f(t,y)$受到一个小扰动变为$\tilde{f}(t,y)$后,数值解的变化情况。相应地,初始条件也可能被扰动,从$y_0$变为$\tilde{y}_0$.

若当扰动量控制在一个范围内时,数值解的偏差也能被控制在一个与扰动量成正比的范围内,则称该数值方法是零稳定的。
\begin{definition}
    [零稳定性的一般定义形式]
    若存在正常数$K^*$和$\eta^*$,使得对任意$\epsilon \leq \eta^*$,当对原方程施加一个扰动
    \begin{align}
        &\|y_0-\tilde{y}_0\| \leq \epsilon \\
        &\|f(t,y)-\tilde{f}(t,y)\| \leq \epsilon,\quad t\in [a,b],\,y\in \boldsymbol{R}^d
    \end{align}
    得到新的方程\footnote{老登在这里的符号规范仿佛是精神分裂了一样,$\tilde{y}_0$表示准确值,$y_0$表示扰动值;f则反过来,$\tilde{f}$表示扰动后的函数,$f$表示准确的函数。}
    \begin{align}
        \begin{cases}
            v'=\tilde{f}(t,v),\quad t\in [a,b]\\
            v(a)=y_0
        \end{cases}
    \end{align}
    若该方程有解存在,且满足
    \begin{align}
        \|y(t)-v(t)\|\leq K^* \epsilon, \quad t\in [a,b]
    \end{align}
    则称该数值方法是零稳定的。
\end{definition}


\paragraph{单步法}

\begin{definition}
    [单步法的零稳定性]
    若对于给定的时间演化区间$(0,T]$,存在常数$h_0$和K,使得当 
\begin{align}
    0<h\leq h_0
\end{align}
时,任意两数值解满足
\begin{align}
    \max_{t_n\leq T}|y_n-z_n|\leq K|y_0-z_0|
\end{align}
则称单步法是零稳定的。
\end{definition}






\paragraph{多步法}
\begin{definition}
    [多步法的零稳定性]
    若存在正常数$C$和$h_0$,当$0<h<h_0$时,多步法的任意两解$u_n$和$v_n$满足不等式
\begin{align}
    \max |u_n-v_n| \leq C \max_{j=0,1,\ldots,k-1} |u_j - v_j|
\end{align}
则称该多步法是零稳定的。
\end{definition}

\subsubsection{多步法的强稳定性}
\begin{definition}
    [多步法的强稳定性]
    若存在某个范数,使得多步法系数矩阵的范数不大于1,则称多步法满足强稳定性条件。
\end{definition}



\subsubsection{M-条件稳定与绝对稳定}
\begin{definition}
    [M-条件稳定]
    设$M>0$,给定一个确定的ODE初值问题,若$\forall h\in(0,h_0]$,两个数值解$\|y_n\|$和$\|z_n\|$满足 
    \begin{align}
        \max_{n\leq \frac{b-a}{h}}\|y_n-z_n\|\leq M^{b-a}\|y_0-z_0\|
    \end{align}
    则称这个方法关于步长$h_0$是 M-条件稳定的。
\end{definition}

\begin{itemize}
    \item $M^{b-a}$称为稳定常数
    \item $M(h_0)$:对于一个给定的$h_0$,让M稳定性成立的最小常数M。或者说,$M(h_0)$是一个数值方法关于$h_0$条件稳定的最小 单位区间稳定常数。
    \item 零稳定与M-条件稳定的关系:零稳定是指存在一个步长$h_0$,使得$h_0$是条件稳定的。
\end{itemize}

\begin{definition}
    [绝对稳定]
    一个数值方法用于求解$\lambda$-试验方程
    \begin{align}
        y'=&\lambda y,\quad t\in [a,b]\\
        y(a)=&\tilde{y}_0
    \end{align}
    若对于任意的初值,对于给定的步长h得到的数值解$\{y_n\}$满足
    \begin{align}
        \lim_{n\to \infty} y_n=0,\quad \text{当} \operatorname{Re}(\lambda)<0
    \end{align}
    则称该数值方法是绝对稳定的。
\end{definition}

\begin{itemize}
    \item 对所有的方法,$(h,\lambda)$以乘积的方式出现在稳定性分析中。
    \item 绝对稳定域:$h-\lambda$的复平面区域中,所有使得数值解趋于0的点$(h,\lambda)$的集合称为该方法的绝对稳定域。
    \item 绝对稳定区间:绝对稳定域与实轴的交集
    \item 若当$n\rightarrow \infty$时$y_n$无界,则称该方法数值不稳定。
\end{itemize}

此外,还有两种条件稳定,即A-条件稳定和L-条件稳定:
\begin{itemize}
    \item A-条件稳定:若绝对稳定区域包含整个复平面的左半边,则称方法是A-条件稳定的。
    \item L-条件稳定:单步法求解试验方程会得到$y_{n+1}=E(h\lambda)y_n$。若当$h\lambda \rightarrow -\infty$时,$|E(h\lambda)|\rightarrow 0$,则称该方法是L-条件稳定的。
\end{itemize}




\subsubsection {一步误差与局部截断误差}
单步法:

\begin{itemize}
    \item 一步误差:
    考虑已知一点的数值解$(t_n,y_n)$,设$y=y(t;t_n,y_n)$是点$(t_n,y_n)$所在的积分曲线(即该数值解点对应的一条精确解),
    用单步法得到$y_{n+1}$处的数值解,则一步误差定义为由点$(t_n,y_n)$所确定的$t_{n+1}$处所确定的“一步精确值”与数值结果导出的“一步近似值”之间的差值:
    \begin{align}
        \tilde{R}_{n+1}=y(t_{n+1};t_n,y_n)-y_{n+1}=y(t_{n+1};t_n,y_n)-y_n - h_{n+1}\phi(t_n,t_{n+1};y_n,y_{n+1};f)
    \end{align}
    \item 局部截断误差:在$t_{n+1}$处,假设前一步的数值解是准确的,即$y_n=y(t_n)$,则在$t_{n+1}$的局部截断误差定义为
\begin{align}
    R_{n+1}=y(t_{n+1})-y_{n+1}=y(t_{n+1})-y(t_n)-h_{n+1}\phi(t_n,t_{n+1};y(t_n),y(t_{n+1});f)
\end{align}
\end{itemize}

二者满足关系
\begin{align}
    (1-h_{n+1}L)|\tilde{R}_{n+1}|\leq  |R_{n+1}| \leq (1+h_{n+1}L)|\tilde{R}_{n+1}|
\end{align}

多步法:

\begin{align}
    R_{n+k}=\sum_{i=0}^k \alpha_i^{(n+k)}y(t_{n+i}) - h_{n+k}\phi(t_n,t_{n+1},\ldots,t_{n+k};y(t_n),y(t_{n+1}),\ldots,y(t_{n+k});f)
\end{align}






\subsubsection {相容性}
单步法:
对任意积分曲线$y=y(t)$,若 
\begin{align}
    R(h)=y(t+h)-y(t)-h\phi(t,t+h;y(t),y(t+h);f)=o(h)
\end{align}
则称该方法是相容的。此外,若 
\begin{align}
    \|R(h)\|\leq Mh^{p+1}
\end{align}
则称该方法至少p阶相容。

若局部截断误差可以展开为
\begin{align}
    R(h)=\psi(t;y)h^{p+1}+O(h^{p+2})
\end{align}
则称$\psi(t;y)h^{p+1}$为主局部截断误差。

此外,对于隐式方法,局部截断误差和一步误差有相同的主项。

多步法:

若局部截断误差$R^{n+k}=o(h_{n+k})$,则称多步法是相容的。此外,若 
\begin{align}
    R^{n+k}=O(h^{p+1}_{n+k})
\end{align}
则称其为p阶相容,主项称为主局部截断误差。



\subsubsection {收敛性与收敛阶}
\begin{definition}
    [单步法的收敛性]
    若单步法满足
    \begin{align}
        \sup_n \|y(t_n)-y_n\|\rightarrow 0, \quad h\rightarrow 0^+, \quad e_0\rightarrow 0
    \end{align}
\end{definition}

\begin{definition}
    [多步法的收敛性]
    若对任意初值
    \begin{align}
        y_r=s_r,\quad r=0,1,\ldots,k-1
    \end{align}
    满足
    \begin{align}
        s_r(h)\rightarrow \tilde{y}_0,\quad h\rightarrow 0^+
    \end{align}
    时总有
    \begin{align}
        \sup_n \|y_n-y(t_n)\|\rightarrow 0,\quad h\rightarrow 0^+
    \end{align}
    则称多步法是收敛的。
\end{definition}




\section{相关定理}

\subsection{相容性}
\subsubsection{单步法相容的充要条件}
\begin{align}
    f(t,y(t))=\phi(t,t;y(t),y(t);f)
\end{align}

\subsubsection{线性k步法相容的充要条件}
\begin{align}
    \rho'(\xi)=\sigma(\xi)\\
    \rho(1)=0
\end{align}

\subsection{整体误差估计}

记第n步的误差为$e_n=y(t_n)-y_n$。若函数$\phi$满足前面条件,则存在$h_0>0$和$C>0$,使得当$0<h<h_0$时,有
\begin{align}
    \|e_n\|\leq C\left(  \|e_0\|+\sup_m \frac{\|R_m\|}{h_m}  \right)
\end{align}
若方法是p阶相容的,则 
\begin{align}
    \|e_n\|=O(\|e_0\|+h^p)
\end{align}

\subsection{收敛性定理}

\subsubsection{单步法}
若方法是相容的,且 
\begin{align}
    y_0\rightarrow \tilde{y}_0,\quad h\rightarrow 0^+
\end{align}
则方法是收敛的。

同时,若$y_0-\tilde{y}_0=O(h^p)$且方法是p阶相容的,则收敛阶为p.

\subsubsection{多步法}
\begin{theorem}
    [多步法收敛的充分条件]
    多步法收敛的充分条件是特征多项式在1处的值为0,即$\rho(1)=0$
\end{theorem}

\begin{theorem}
    [多步法收敛性定理]

    多步法若强稳定且相容,则多步法收敛。进一步,若方法p阶相容,并且初值的近似误差也是p阶的,则
    \begin{align}
        \sup_n \|y_n-y(t_n)\|=O(h^p)
    \end{align}
\end{theorem}


\subsection {稳定性定理}

\subsubsection{单步法}

\paragraph{零稳定性定理}
单步法是零稳定的。

(虽然PPT上就这么一句话,但Gemini说其实这要求增量函数$\phi$和方程右侧函数$f$都满足连续性和Lipschitz条件。)

\paragraph{绝对稳定性判据}
\begin{theorem}
    [单步法的绝对稳定性判据]
    若单步法用于求解试验方程得到
    \begin{align}
        y_{n+1}=E(h\lambda)y_n 
    \end{align}
    则当且仅当
    \begin{align}
        |E(h\lambda)|<1
    \end{align}
    时,单步法是绝对稳定的。
\end{theorem}

\subsubsection{多步法}



\paragraph{多步法强稳定性的一个充分条件}
若对于等步长特征多项式
\begin{align}
    \rho(\xi)=\sum_{i=0}^k \alpha_i \xi^i
\end{align}
除了1这一个单根外,所有根的模值都小于1,则存在一个常数$\kappa>1$,使得当$h_m/h_n\leq \kappa$时,多步法是强稳定的。

\paragraph{多步法的强稳定性与零稳定性}
强稳定性可以确保多步法是零稳定的。

\paragraph{多步法的绝对稳定性判据}
\begin{theorem}
    [多步法的绝对稳定新判据]
    线性多步法用于求解试验方程得到
    \begin{align}
        \sum_{i=0}^k(\alpha_i-h\lambda \beta_i)y_{n+i}=0
    \end{align}
    记 $\mu=h\lambda$,定义稳定性多项式如下:
    \begin{align}
        \Pi(\xi;\mu)=\rho(\xi)-\mu \sigma(\xi)=\sum_{i=0}^k (\alpha_i - \mu \beta_i)\xi^i
    \end{align}
    对于给定的$\mu=h\lambda$,当且仅当稳定性多项式的k个根的模都小于1时,该多步法是绝对稳定的。
\end{theorem}


\section {具体求解方法}

区间划分规定:
\begin{align}
    a=t_0<t_1<\ldots <t_N=b\\
\end{align}

记号:

区间长度:
\begin{align}
        h_i=t_i-t_{i-1},\quad i=1,2,\ldots,N
\end{align}
注意,此时的区间长度为这个下标减去前一个下标,与数值微积分一节不同。

此外,记 
\begin{align}
    h=\max_i h_i
\end{align}

准确值与近似值:
\begin{itemize}
    \item 准确值记为$y(t_i)$
    \item 近似值记为$y_i$
\end{itemize}

\subsection {单步法}

基本方法:通过各种不同数值表达式得到前一时刻和后一时刻的函数值之间的关系,进行时间步进

一般形式:
\begin{align}
    y_{n+1}=y_n + h_{n+1} \phi(t_n,t_{n+1};y_n,y_{n+1};f)
\end{align}


\subsubsection {Euler 方法}
\paragraph{欧拉折线法:显格式}
由每一点的向前差分近似导数得到。
\begin{align}
    y_{n+1}=y_n+h_{n+1}f(t_n,y_n)
\end{align}

\begin{itemize}
    \item 一阶相容性
    \item 主局部截断误差$\frac{1}{2}h^2y''(t)$
\end{itemize}

\paragraph {向前欧拉:显格式}
将问题转化为积分形式,用左矩形公式近似积分,得到Euler公式:
\begin{align}
    y_{n+1}=y_n+h_{n+1}f(t_n,y_n)
\end{align}


\paragraph {向后欧拉:隐格式}
若用右矩形求积公式近似积分,则得到向后欧拉方法:
\begin{align}
    y_{n+1}=y_n+h_{n+1}f(t_{n+1},y_{n+1})
\end{align}

\subsubsection{数值积分导出的其他方法}

\paragraph{中点格式:可以显式化的隐格式}

用中点求积公式近似积分,得到中点格式:
\begin{align}
    y_{n+1}=y_n+h_{n+1}f(t_{n+1/2},y_{n+1/2})
\end{align}
若中点处$y_n$取一阶近似,即
\begin{align}
    t_{n+1/2}=&\frac{t_n+t_{n+1}}{2}\\
    y_{n+1/2}=&y_n+\frac{h_{n+1}}{2}f(t_n,y_n)
\end{align}
即为显式中点方法。

\paragraph{梯形公式:隐格式}
用梯形公式近似积分,得到梯形格式:
\begin{align}
    y_{n+1}=y_n+\frac{h_{n+1}}{2}[f(t_n,y_n)+f(t_{n+1},y_{n+1})]
\end{align}

\begin{itemize}
    \item 二阶相容性
    \item 主局部截断误差$-\frac{1}{12}h^3 y^{(3)}(t)$
\end{itemize}

\paragraph{显式梯形公式}
同样用一阶近似代入梯形公式,得到显式梯形公式:
\begin{align}
    y_{n+1}=&y_n+\frac{h_{n+1}}{2}[f(t_n,y_n)+f(t_{n+1},y_{n+1}^*)]\\
    y_{n+1}^*=&y_n+h_{n+1}f(t_n,y_n)
\end{align}


% \subsubsection {改进欧拉方法}
\subsubsection {龙格 - 库塔(RK)方法}
\textcolor{blue}{明确说了必考显式龙格库塔方法的推导}

思想:用s个点的斜率的线性组合来近似平均斜率,而每一个点的确定又都与其他点相关。

一般形式:
\begin{align}
    y_{n+1}=y_n+h_{n+1}\sum_{i=1}^s b_i k_i
\end{align}
式中,
\begin{align}
    k_i=f(t_n+c_ih_{n+1},y_n+h_{n+1} \underbrace{\sum_{j=1}^sa_{ij}k_j}_{\substack{\text{用s个点得到的}\\\text{近似平均斜率}}})
\end{align}
且满足
\begin{align}
    \sum_{j=1}^s a_{ij}=c_i
\end{align}

一般形式下的参数或者约束:
\begin{align}
    c_k=\sum_{j=1}^s a_{kj} \text{(行和)}\\
    \sum_{i=1}^s b_i=1 \text{(权重参数)}
\end{align}

这种一般形式的等价形式为
\begin{align}
    y_{n+1}=y_n+h_{n+1}\sum_{i=1}^s b_if(t_n+c_ih_{n+1},Y_i)\\
    Y_i=y_n+h_{n+1}\sum_{j=1}^s a_{ij}f(t_n+c_jh_{n+1},Y_j)
\end{align}
可以理解为引入一个$y_n$的代数函数$Y_i$作为中间变量。

\paragraph{显式与隐式RK方法}
若矩阵 $A=(a_{ij})$ 是严格下三角矩阵,则称该RK方法为显式RK方法,否则为隐式RK方法。


\paragraph {显式2 阶 RK 方法}
参数组合:
\begin{align}
    b_1+b_2=1\\
    b_2c_2=\frac{1}{2}
\end{align}

\paragraph {3 阶 RK 方法}

\begin{align}
    b_1+b_2+b_3=1\\
    b_2c_2+b_3c_3=\frac{1}{2}\\
    b_2c_2^2+b_3c_3^2=\frac{1}{3}\\
    b_3a_{32}c_2=\frac{1}{6}
\end{align}

\paragraph{经典RK方法}
\begin{itemize}
    \item 修正欧拉法::即显式中点方法 
    \item 改进欧拉法:显式的梯形方法
\end{itemize}


\subsection {多步法}
设计方法:数值积分

先对所有节点进行插值,得到$f(t,y)$的插值多项式,再对所需要的区域进行积分即可得到多步法形式。

\subsubsection {Adams 方法:上一步函数值和q步导数值}
\paragraph {显式 Adams 公式}
\begin{align}
    y_{n+1}=y_n+h(\beta_{q0}f_n+\beta_{q1}f_{n-1}+\ldots +\beta_{qq}f_{n-q})
\end{align}

\paragraph {隐式 Adams 公式}
\subsubsection {Nyström 方法:隔一步函数值和q步导数值,显式}

\subsubsection {Milne-Simpson 方法:隔一步函数值和q步导数值,隐式}








\chapter {线性代数方程组数值解法}


\section{线性代数方程组直接解法 (Direct Methods)}

\subsection{基本概念}
求解 $Ax=b$,其中 $A \in \mathbb{R}^{n \times n}$ 非奇异。

\subsubsection{Cramer 法则 (Cramer's Rule)}

即直接求逆

\begin{theorem}[Cramer 法则]
    若 $\det(A) \ne 0$,则 $x_i = \det(A_i)/\det(A)$。
\end{theorem}
\begin{remark}
    理论重要但计算不可行($O((n+1)!)$),仅用于极小规模或理论证明。
\end{remark}

\subsection{Gauss 消去法 (Gauss Elimination)}
\subsubsection{前向消去与回代}
求解 $Ax=b$ 分为两个阶段:
\begin{enumerate}
    \item \textbf{消元过程 (Forward Elimination)}:
    将增广矩阵 $[A|b]$ 变换为上三角形式 $[U|y]$。
    对于第 $k$ 步 ($k=1, 2, \dots, n-1$):
    \begin{itemize}
        \item \textbf{计算乘子}:设主元 $a_{kk}^{(k)} \ne 0$,对所有 $i = k+1, \dots, n$:
        \[
            m_{ik} = \frac{a_{ik}^{(k)}}{a_{kk}^{(k)}}
        \]
        \item \textbf{行变换}:将第 $i$ 行减去第 $k$ 行的 $m_{ik}$ 倍,消去 $a_{ik}^{(k)}$:
        \[
            a_{ij}^{(k+1)} = a_{ij}^{(k)} - m_{ik} a_{kj}^{(k)}, \quad j=k+1, \dots, n
        \]
        \[
            b_i^{(k+1)} = b_i^{(k)} - m_{ik} b_k^{(k)}
        \]
    \end{itemize}
    经过 $n-1$步后,得到上三角矩阵 $U = A^{(n)}$ 和变换后的右端项 $y = b^{(n)}$。

    \item \textbf{回代过程 (Backward Substitution)}:
    求解上三角方程组 $Ux=y$。
    \[
        \begin{cases}
            u_{11}x_1 + u_{12}x_2 + \dots + u_{1n}x_n = y_1 \\
            \dots \\
            u_{nn}x_n = y_n
        \end{cases}
    \]
    计算公式为:
    \[
        x_n = \frac{y_n}{u_{nn}}
    \]
    \[
        x_i = \frac{1}{u_{ii}} \left( y_i - \sum_{j=i+1}^n u_{ij} x_j \right), \quad i = n-1, \dots, 1
    \]
\end{enumerate}

\subsubsection{选主元策略 (Pivoting)}
若主元 $|a_{kk}^{(k)}|$ 过小,乘子 $m_{ik}$ 会很大,导致舍入误差剧烈放大。
\begin{itemize}
    \item \textbf{列主元消去法}:交换行,使 $\max_{i \ge k} |a_{ik}^{(k)}|$ 所在行成为当前主行。
    \item \textbf{全主元消去法}:交换行和列,选全子阵最大值。
    \item \textbf{数值稳定性}:选主元是保证 Gauss 消去法数值稳定的必要条件。
\end{itemize}

\subsection{矩阵三角分解 (Matrix Factorization)}
\subsubsection{LU 分解}
\begin{theorem}[存在唯一性]
    若 $A$ 的所有顺序主子式非零,则 $A$ 可唯一分解为 $A=LU$,其中 $L$ 为单位下三角阵,$U$ 为上三角阵。
\end{theorem}


\begin{proofstep}{消去步的矩阵表示}


    在 Gauss 消去的第 $k$ 步,我们通过行变换将第 $k$ 列对角线以下的元素消为 0。这一步等价于左乘一个\textbf{初等下三角矩阵}(Atomic Lower Triangular Matrix)$L_k$:
    \[
        L_k = I - \mathbf{m}_k \mathbf{e}_k^T = 
        \begin{bmatrix}
        1 & & & & \\
        & \ddots & & & \\
        & & 1 & & \\
        & & -m_{k+1,k} & 1 & \\
        & & \vdots & & \ddots & \\
        & & -m_{n,k} & & & 1
        \end{bmatrix}
    \]
    其中 $\mathbf{m}_k = [0, \dots, 0, m_{k+1,k}, \dots, m_{n,k}]^T$ 是由第 $k$ 步的乘子构成的向量,$\mathbf{e}_k$ 是单位向量。
    
    整个消去过程可以写成:
    \[
        L_{n-1} L_{n-2} \dots L_1 A = U
    \]
\end{proofstep}

\begin{proofstep}{逆矩阵的性质}


    $L_k$ 的逆矩阵 $L_k^{-1}$ 非常简单,只需将乘子的符号取反:
    \[
        L_k^{-1} = (I - \mathbf{m}_k \mathbf{e}_k^T)^{-1} = I + \mathbf{m}_k \mathbf{e}_k^T
    \]
    这是因为 $(I - \mathbf{m}_k \mathbf{e}_k^T)(I + \mathbf{m}_k \mathbf{e}_k^T) = I - \mathbf{m}_k (\mathbf{e}_k^T \mathbf{m}_k) \mathbf{e}_k^T = I$(注意 $\mathbf{e}_k^T \mathbf{m}_k = 0$,因为 $\mathbf{m}_k$ 前 $k$ 个元素为 0)。
\end{proofstep}

\begin{proofstep}{构造 L 矩阵}


    由 $L_{n-1} \dots L_1 A = U$,我们有:
    \[
        A = (L_{n-1} \dots L_1)^{-1} U = L_1^{-1} L_2^{-1} \dots L_{n-1}^{-1} U
    \]
    令 $L = L_1^{-1} L_2^{-1} \dots L_{n-1}^{-1}$。神奇的是,计算这个乘积不需要复杂的矩阵运算。由于 $L_k^{-1}$ 的非零乘子位于第 $k$ 列,$L_{j}^{-1}$ 的位于第 $j$ 列($j > k$),它们的乘积只是简单地将各列的乘子\textbf{填充}到单位矩阵对应的位置,互不干扰:
    \[
        L = \left( I + \sum_{k=1}^{n-1} \mathbf{m}_k \mathbf{e}_k^T \right) = 
        \begin{bmatrix}
        1 & & & & \\
        m_{21} & 1 & & & \\
        m_{31} & m_{32} & 1 & & \\
        \vdots & \vdots & \ddots & \ddots & \\
        m_{n1} & m_{n2} & \dots & m_{n,n-1} & 1
        \end{bmatrix}
    \]
    \textbf{结论}:$L$ 矩阵的下三角部分正是 Gauss 消去过程中计算出的所有乘子 $m_{ik}$。
\end{proofstep}

\paragraph{LU 分解的应用形式}
一旦得到 $A=LU$,求解 $Ax=b$ 变为两步:
\begin{enumerate}
    \item 解下三角方程组 $Ly=b$ (前代)。
    \item 解上三角方程组 $Ux=y$ (回代)。
\end{enumerate}


\subsubsection{Cholesky 分解 ($LL^T$)}
针对\textbf{对称正定矩阵 (SPD)} 的高效分解。

矩阵 $A$ 存在 $LL^T$ 分解(其中 $l_{ii} > 0$)$\iff$ $A$ 是对称正定的。


\subsubsection{附:对称正定矩阵 (SPD) 的判定方法}
\begin{enumerate}
    \item \textbf{定义法}:
    $A$ 是对称矩阵 ($A^T=A$),且对任意非零向量 $x \ne 0$,都有二次型 $x^T A x > 0$。
    \item \textbf{特征值判别法}:
    $A$ 是对称矩阵,且 $A$ 的所有特征值 $\lambda_i$ 均严格大于零 ($\lambda_i > 0, \forall i$)。
    \item \textbf{顺序主子式判别法 (Sylvester 准则)}:
    $A$ 的所有顺序主子式 $\det(A_k)$ 均严格大于零 ($k=1, \dots, n$)。
    \item \textbf{Cholesky 分解存在性}:
    若 Cholesky 分解算法能够顺利执行到底,且所有对角元 $l_{ii}$ 均为实数且非零(即开方过程未遇到负数或零),则 $A$ 必为对称正定矩阵。这也是计算机程序中判断 SPD 的最实用方法。
\end{enumerate}

\begin{theorem}
    若 $A$ 对称正定,则存在唯一的对角元 $>0$ 的下三角阵 $L$ 使得 $A=LL^T$。
    此外,Cholesky 分解是\textbf{数值稳定}的,无需选主元(因为 $|l_{ij}|^2 \le \sum l_{ik}^2 = a_{ii}$,元素不会增长)。
\end{theorem}

\begin{itemize}
    \item \textbf{算法推导}:
    比较 $A = LL^T$ 两边的元素:
    \[
        a_{ij} = \sum_{k=1}^n l_{ik} (L^T)_{kj} = \sum_{k=1}^{\min(i,j)} l_{ik} l_{jk}
    \]
    \item \textbf{计算公式}(按列计算):
    对于 $j=1, \dots, n$:
    \begin{enumerate}
        \item \textbf{对角元}:
        \[
            l_{jj} = \sqrt{a_{jj} - \sum_{k=1}^{j-1} l_{jk}^2}
        \]
        \item \textbf{非对角元} ($i > j$):
        \[
            l_{ij} = \frac{1}{l_{jj}} \left( a_{ij} - \sum_{k=1}^{j-1} l_{ik}l_{jk} \right)
        \]
    \end{enumerate}
    \item \textbf{改进:$LDL^T$ 分解(针对非正定对称阵)}:
    如果 $A$ 仅仅是对称的(可能是未定或不定的),但其顺序主子式非零,可以进行 $A = L D L^T$ 分解,其中 $L$ 是单位下三角,$D$ 是对角阵(对角元可正可负)。
    \[
        d_j = a_{jj} - \sum_{k=1}^{j-1} d_k l_{jk}^2, \quad l_{ij} = \frac{1}{d_j} \left( a_{ij} - \sum_{k=1}^{j-1} d_k l_{ik} l_{jk} \right)
    \]
    这种方法避免了开方,适用于更广泛的对称矩阵,但若遇到 $d_j \approx 0$ 仍需选主元(对称选主元 $P A P^T = L D L^T$)。
\end{itemize}

\subsubsection{追赶法 (Thomas Algorithm)}

针对三对角矩阵的 $LU$ 分解简化版(待补充)。

\subsection{误差分析与条件数}
\subsubsection{条件数 (Condition Number)}
衡量方程组解对数据扰动的敏感程度。
\[ \text{cond}(A) = ||A|| \cdot ||A^{-1}|| \]
\begin{property}
    1. $\text{cond}(A) \ge 1$。\\

    2. 若 $A$ 酉相似于 $B$,则 $\text{cond}_2(A) = \text{cond}_2(B)$。\\

    3. 只与矩阵“形态”有关,数乘不改变条件数:$\mathrm{cond}(\alpha A)=\mathrm{cond}(A)$,$\alpha \ne 0$。\\

    4. 正交阵的2-条件数为1:若A为正交阵,则$\mathrm{cond}_2(A)=1$。\\

    5. 正交阵不改变矩阵条件数:若U为正交阵,则$\mathrm{cond}_2(UA)=\mathrm{cond}_2(AU)=\mathrm{cond}_2(A)$。\\

    6. 条件数与特征值的关系:
        \begin{align}
            \mathrm{cond}(A)\geq \frac{\max_i |\lambda_i|}{\min_i |\lambda_i|}
        \end{align}\\
    对于对称矩阵(不要求正定!),$\text{cond}_2(A) = \frac{|\lambda|_{\max}}{|\lambda|_{\min}}$。\\
\end{property}



\subsubsection{扰动误差界定理}
设 $Ax=b, A$ 非奇异。考虑 $(A+\delta A)(x+\delta x) = b+\delta b$。
若满足条件 $||\delta A|| \cdot ||A^{-1}|| < 1$(保证 $A+\delta A$ 非奇异),则有如下相对误差界:
\[
    \frac{||\delta x||}{||x||} \le \frac{\text{cond}(A)}{1 - \text{cond}(A)\frac{||\delta A||}{||A||}} \left( \frac{||\delta A||}{||A||} + \frac{||\delta b||}{||b||} \right)
\]

\begin{proof}

    \[
        (A+\delta A)(x+\delta x) = b + \delta b
    \]
    展开得:
    \[
        Ax + A\delta x + \delta A x + \delta A \delta x = b + \delta b
    \]
    由于 $Ax=b$,消去得:
    \[
        (A+\delta A)\delta x = \delta b - \delta A x
    \]
    

    两边左乘 $(A+\delta A)^{-1}$:
    \[
        \delta x = (A+\delta A)^{-1}(\delta b - \delta A x)
    \]
    
    \textbf{ 利用范数放缩}
    取范数:
    \[
        ||\delta x|| \le ||(A+\delta A)^{-1}|| \cdot (||\delta b|| + ||\delta A|| \cdot ||x||)
    \]
    

    根据前面的逆矩阵扰动定理:
    \[
        ||(A+\delta A)^{-1}|| \le \frac{||A^{-1}||}{1 - ||A^{-1}||\cdot||\delta A||}
    \]
    代入上式:
    \[
        ||\delta x|| \le \frac{||A^{-1}||}{1 - ||A^{-1}||\cdot||\delta A||} (||\delta b|| + ||\delta A|| \cdot ||x||)
    \]
    

    两边除以 $||x||$:
    \[
        \frac{||\delta x||}{||x||} \le \frac{||A^{-1}||}{1 - ||A^{-1}||\cdot||\delta A||} \left( \frac{||\delta b||}{||x||} + ||\delta A|| \right)
    \]
    利用 $b=Ax \implies ||b|| \le ||A|| \cdot ||x|| \implies \frac{1}{||x||} \le \frac{||A||}{||b||}$:
    \[
        \frac{||\delta x||}{||x||} \le \frac{||A^{-1}||}{1 - ||A^{-1}||\cdot||\delta A||} \left( \frac{||\delta b|| \cdot ||A||}{||b||} + ||\delta A|| \right)
    \]
    
    \textbf{引入条件数}
    提取 $||A||$,并利用 $\text{cond}(A) = ||A|| \cdot ||A^{-1}||$:
    \[
        \frac{||\delta x||}{||x||} \le \frac{||A^{-1}|| \cdot ||A||}{1 - ||A^{-1}|| \cdot ||A|| \cdot \frac{||\delta A||}{||A||}} \left( \frac{||\delta b||}{||b||} + \frac{||\delta A||}{||A||} \right)
    \]
    整理得:
    \[
        \frac{||\delta x||}{||x||} \le \frac{\text{cond}(A)}{1 - \text{cond}(A)\frac{||\delta A||}{||A||}} \left( \frac{||\delta A||}{||A||} + \frac{||\delta b||}{||b||} \right)
    \]
    证毕。
\end{proof}

\subsubsection{后验误差估计 (Posteriori Error Estimation)}
\textit{问题:如何利用计算出的近似解 $\tilde{x}$ 来估计真实误差 $x - \tilde{x}$?}

设 $\tilde{x}$ 为近似解,定义残差(Residual)为:
\[
    r = b - A\tilde{x}
\]


\begin{theorem}[基于残差的误差界]
    设 $A$ 非奇异,则相对误差与相对残差之间满足如下不等式:
    \[
        \frac{1}{\text{cond}(A)} \frac{||r||}{||b||} \le \frac{||x-\tilde{x}||}{||x||} \le \text{cond}(A) \frac{||r||}{||b||}
    \]
\end{theorem}

\begin{proof}
    \textbf{1. 上界}:
    由 $r = b - A\tilde{x} = Ax - A\tilde{x} = A(x-\tilde{x})$,可得:
    \[
        x - \tilde{x} = A^{-1}r \implies ||x-\tilde{x}|| \le ||A^{-1}|| \cdot ||r||
    \]
    又由 $Ax=b \implies ||b|| \le ||A|| \cdot ||x|| \implies \frac{1}{||x||} \le \frac{||A||}{||b||}$。
    两式相乘:
    \[
        \frac{||x-\tilde{x}||}{||x||} \le ||A^{-1}|| \cdot ||r|| \cdot \frac{||A||}{||b||} = \text{cond}(A) \frac{||r||}{||b||}
    \]
    
    \textbf{2. 下界}:
    由 $r = A(x-\tilde{x}) \implies ||r|| \le ||A|| \cdot ||x-\tilde{x}|| \implies ||x-\tilde{x}|| \ge \frac{||r||}{||A||}$。
    又由 $x = A^{-1}b \implies ||x|| \le ||A^{-1}|| \cdot ||b|| \implies \frac{1}{||x||} \ge \frac{1}{||A^{-1}|| \cdot ||b||}$。
    两式相乘:
    \[
        \frac{||x-\tilde{x}||}{||x||} \ge \frac{||r||}{||A|| \cdot ||A^{-1}|| \cdot ||b||} = \frac{1}{\text{cond}(A)} \frac{||r||}{||b||}
    \]
    证毕。
\end{proof}

\begin{remark}[物理意义]
    \begin{enumerate}
        \item \textbf{残差小 $\ne$ 误差小}:如果 $\text{cond}(A)$ 很大(矩阵病态),即使残差 $||r||$ 很小(计算机解出的方程组几乎“满足”方程),解的误差 $||x-\tilde{x}||$ 也可能非常大。这说明对于病态方程组,仅仅看残差是不够的。
        \item \textbf{迭代改善 (Iterative Refinement)}:利用残差可以提高解的精度。
        计算 $r = b - A\tilde{x}$(需用高精度计算),解修正方程 $A e = r$,则新解 $x_{\text{new}} = \tilde{x} + e$ 会更精确。
    \end{enumerate}
\end{remark}

\subsubsection{病态矩阵 (Ill-conditioned Matrix)}
条件数很大的矩阵。
\begin{example}[Hilbert 矩阵]
\[ H_{ij} = \frac{1}{i+j-1} \]
$H_n$ 高度病态。当 $n$ 稍大时,$\text{cond}(H_n)$ 极大,直接法解将完全失真。
\end{example}

\subsection{广义逆与正则化 (Generalized Inverse \& Regularization)}
\textit{针对奇异或极度病态方程组 (参考第二章讲义)}

\subsubsection{Moore-Penrose 广义逆}
对于 $A \in \mathbb{C}^{m \times n}$ (秩 $r$),存在唯一的 $A^\dagger$ 满足 Penrose 方程。
\begin{itemize}
    \item \textbf{最小二乘解}:方程组 $Ax=b$ 的最小二乘解集为 $S = \{x | A^H Ax = A^H b\}$。
    \item \textbf{广义逆解}:$x = A^\dagger b$ 是 $S$ 中 2-范数最小的解。
    \item \textbf{SVD 计算}:若 $A=U\Sigma V^H$,则
    \[ A^\dagger = V \Sigma^\dagger U^H \]
    其中 $\Sigma^\dagger = \text{diag}(\sigma_1^{-1}, \dots, \sigma_r^{-1}, 0, \dots, 0)$。
\end{itemize}

\subsubsection{Tikhonov 正则化}
用于处理严重的病态问题(如第一类 Fredholm 积分方程离散化)。
\begin{itemize}
    \item \textbf{变分原理}:最小化带罚项的泛函:
    \[ J_\alpha(x) = ||Ax-b||_2^2 + \alpha ||x||_2^2 \]
    其中 $\alpha > 0$ 为正则化参数。
    \item \textbf{正规方程}:
    \[ (\alpha I + A^H A) x_\alpha = A^H b \]
    \item \textbf{效果}:矩阵 $\alpha I + A^H A$ 的条件数比 $A^H A$ 改善很多(特征值整体平移 $\alpha$),解更稳定。
    \item \textbf{收敛性}:当 $\alpha \to 0$ 时,$x_\alpha \to A^\dagger b$。
\end{itemize}

\vspace{1cm}


\section{单步定常线性迭代解法 (Stationary Linear Iterative Methods)}

\subsection{迭代法基础理论}

\subsubsection{迭代法的基本概念}
\begin{itemize}
    \item \textbf{构造思想}:对于大规模稀疏方程组 $Ax=b$,直接法计算代价过高。迭代法通过构造向量序列 $\{x^{(k)}\}$,使其极限为方程的解。
    \[
        \lim_{k \to \infty} x^{(k)} = x^*
    \]
    \item \textbf{单步定常线性迭代}:
    迭代格式为:
    \[
        x^{(k+1)} = B x^{(k)} + f
    \]
    其中 $B$ 称为\textbf{迭代矩阵},$f$ 为常向量。这种格式只依赖于前一步 $x^{(k)}$(单步),且 $B$ 和 $f$ 不随 $k$ 变化(定常)。
    
    \item \textbf{构造方法(矩阵分裂)}:
    将 $A$ 分裂为 $A = M - N$,其中 $M$ 是非奇异的“近似”矩阵(易于求逆,如对角阵或三角阵)。
    \[
        Ax = b \iff (M-N)x = b \iff Mx = Nx + b
    \]
    由此导出的迭代格式为:
    \[
        Mx^{(k+1)} = Nx^{(k)} + b \implies x^{(k+1)} = M^{-1}N x^{(k)} + M^{-1}b
    \]
    此时迭代矩阵 $B = M^{-1}N$,常向量 $f = M^{-1}b$。
    
    \item \textbf{Richardson 迭代}:
    最简单的迭代格式,取 $M = \frac{1}{\omega} I$。
    \[
        x^{(k+1)} = x^{(k)} + \omega (b - Ax^{(k)}) = (I - \omega A)x^{(k)} + \omega b
    \]
    迭代矩阵 $B = I - \omega A$。收敛条件:$0 < \omega < \frac{2}{\lambda_{\max}(A)}$(假设 A 为 SPD)。
    



    \item \textbf{预处理迭代}:
    若引入预处理矩阵 $M \approx A$,则 Richardson 迭代变为:
    \[
        x^{(k+1)} = x^{(k)} + \omega M^{-1}(b - Ax^{(k)})
    \]
    这等价于对预处理后的方程 $M^{-1}Ax = M^{-1}b$ 进行 Richardson 迭代。
\end{itemize}

\subsubsection{向量序列的收敛性}
\begin{definition}[序列收敛]
    设 $\{x^{(k)}\}_{k=0}^\infty$ 是赋范空间 $(\mathbb{R}^n, ||\cdot||)$ 中的向量序列。若存在 $x \in \mathbb{R}^n$,使得
    \[
        \lim_{k \to \infty} ||x^{(k)} - x|| = 0
    \]
    则称序列 $\{x^{(k)}\}$ 收敛于 $x$,记为 $\lim_{k\to\infty} x^{(k)} = x$。
\end{definition}

\begin{theorem}[收敛性的等价性]

    \begin{enumerate}
        \item \textbf{范数无关性}:由于有限维空间 $\mathbb{R}^n$ 上的所有范数都是等价的,因此向量序列的收敛性与具体选择的范数无关。如果一个序列在某种范数下收敛,它在任意范数下都收敛。
        
        \item \textbf{分量收敛性}:向量序列 $\{x^{(k)}\}$ 收敛于 $x$ 的充要条件是其每一个分量序列都收敛于 $x$ 的对应分量。
        \[
            \lim_{k \to \infty} x^{(k)} = x \iff \lim_{k \to \infty} x_i^{(k)} = x_i, \quad \forall i=1,\dots,n
        \]
    \end{enumerate}
\end{theorem}



\subsubsection{迭代误差与收敛速度}
设精确解为 $x^*$,第 $k$ 步误差为 $e^{(k)} = x^{(k)} - x^*$。
\begin{itemize}
    \item \textbf{误差传播方程}:
    \[
        e^{(k+1)} = x^{(k+1)} - x^* = (Bx^{(k)} + f) - (Bx^* + f) = B(x^{(k)} - x^*) = B e^{(k)}
    \]
    递推可得:
    \[
        e^{(k)} = B^k e^{(0)}
    \]
    \item \textbf{Jordan 标准型分析与收敛充要条件}:
    将 $B$ 转化为 Jordan 标准型 $B = PJP^{-1}$,则 $B^k = P J^k P^{-1}$。
    对于 Jordan 块 $J_i(\lambda_i)$:
    \[
        J_i^k = \begin{bmatrix}
        \lambda_i^k & \binom{k}{1}\lambda_i^{k-1} & \dots \\
        & \lambda_i^k & \binom{k}{1}\lambda_i^{k-1} \\
        & & \ddots
        \end{bmatrix}
    \]
    要使 $B^k \to 0$(即 $e^{(k)} \to 0$ 对任意初值成立),充要条件是所有特征值的模 $|\lambda_i| < 1$。
    即 \textbf{谱半径条件}:$\rho(B) < 1$。
    
    \item \textbf{平均收敛率 (Average Rate of Convergence)}:
    考查前 $k$ 步误差范数的平均衰减:
    \[
        \frac{||e^{(k)}||}{||e^{(0)}||} \le ||B^k|| \approx \rho(B)^k
    \]
    定义 $k$ 步平均收敛率为:
    \[
        R_k(B) = -\frac{1}{k} \ln ||B^k||
    \]
    
    \item \textbf{渐进收敛率 (Asymptotic Rate of Convergence)}:
    当 $k \to \infty$ 时,利用 Gelfand 半径公式 $\lim_{k\to\infty} ||B^k||^{1/k} = \rho(B)$,可得:
    \[
        R(B) = \lim_{k \to \infty} R_k(B) = -\ln \rho(B)
    \]
    \textbf{意义}:$R(B)$ 越大(即 $\rho(B)$ 越小),收敛越快。若要使误差减小 $10^{-m}$ 倍(增加 $m$ 位有效数字),所需迭代次数 $k$ 约为:
    \[
        k \approx \frac{m \ln 10}{R(B)}
    \]
\end{itemize}

\subsection{经典迭代方法}
将 $A$ 分解为 $A = D - L - U$:
\begin{itemize}
    \item $D$:对角部分。
    \item $-L$:严格下三角部分。
    \item $-U$:严格上三角部分。
\end{itemize}

\subsubsection{Jacobi 迭代 }
\begin{itemize}
    \item \textbf{分裂}:$M = D, \quad N = L+U$。
    \item \textbf{公式}:$D x^{(k+1)} = (L+U) x^{(k)} + b$。
    \item \textbf{分量形式}:
    \[
        x_i^{(k+1)} = \frac{1}{a_{ii}} \left( b_i - \sum_{j \ne i} a_{ij} x_j^{(k)} \right)
    \]
    \item \textbf{迭代矩阵}:$B_J = D^{-1}(L+U)$。
\end{itemize}

\subsubsection{Gauss-Seidel 迭代 }
\begin{itemize}
    \item \textbf{思想}:计算 $x_i^{(k+1)}$ 时,利用已更新的 $x_1^{(k+1)}, \dots, x_{i-1}^{(k+1)}$。
    \item \textbf{分裂}:$M = D-L, \quad N = U$。
    \item \textbf{公式}:$(D-L) x^{(k+1)} = U x^{(k)} + b$。
    \item \textbf{分量形式}:
    \[
        x_i^{(k+1)} = \frac{1}{a_{ii}} \left( b_i - \sum_{j < i} a_{ij} x_j^{(k+1)} - \sum_{j > i} a_{ij} x_j^{(k)} \right)
    \]
    \item \textbf{迭代矩阵}:$B_{GS} = (D-L)^{-1}U$。
    \item \textbf{收敛性}:若 $A$ 为 \textbf{SPD} 矩阵,GS 法必收敛。
\end{itemize}

\subsubsection{超松弛迭代 (SOR法)}
\begin{itemize}
    \item \textbf{思想}:对 GS 迭代进行加权平均(外推),加速收敛。
    \[
        x^{(k+1)} = (1-\omega)x^{(k)} + \omega x_{GS}^{(k+1)}
    \]
    其中 $\omega$ 为松弛因子。
    \item \textbf{分裂}:$M = \frac{1}{\omega}(D-\omega L)$。
    \item \textbf{迭代矩阵}:
    \[
        B_{SOR} = (D-\omega L)^{-1} [(1-\omega)D + \omega U]
    \]
    \item \textbf{收敛性}:
    \begin{enumerate}
        \item \textbf{必要条件}:$0 < \omega < 2$(Kahan定理)。
        \item 若 $A$ 为 SPD,则 $0 < \omega < 2 \iff$ SOR 收敛。
    \end{enumerate}
    \item \textbf{最优松弛因子}:
    对于具有 Property A 的矩阵(如三对角阵):
    \[
        \omega_{opt} = \frac{2}{1 + \sqrt{1 - \rho(B_J)^2}}
    \]
    此时 $\rho(B_{SOR}) = \omega_{opt} - 1$。
\end{itemize}

\subsubsection*{经典迭代方法计算收敛性的小寄巧}
\begin{align}
    \det(\lambda I-B)=\det (\lambda I-M^{-1}N)=\det(M^{-1})\cdot \det(\lambda M-N)
\end{align}
令$\det (\lambda I-B)=0$即等价于求$\det(\lambda M-N)=0$,从而避免计算$M^{-1}$。

\vspace{1cm}



\section{非定常迭代法 (Krylov Subspace Methods)}

\subsection{变分原理与 Ritz 方法基础}

\subsubsection{算子方程与变分问题的等价性}
考虑线性方程组 $Ax=b$,其中 $A \in \mathbb{R}^{n \times n}$ 是**对称正定 (SPD)** 矩阵。
\begin{theorem}[变分原理]
    求解 $Ax=b$ 等价于寻找二次泛函 $\phi(x)$ 的极小值点:
    \[
        \phi(x) = \frac{1}{2}(Ax, x) - (b, x) = \frac{1}{2} x^T A x - b^T x
    \]
    即 $x^* = A^{-1}b \iff x^* = \arg\min_{x \in \mathbb{R}^n} \phi(x)$。
\end{theorem}
\begin{proof}
    计算 $\phi(x)$ 的梯度:
    \[
        \nabla \phi(x) = \frac{1}{2}(A+A^T)x - b = Ax - b \quad (\text{因 } A=A^T)
    \]
    令梯度为零,即 $Ax - b = 0 \implies Ax=b$。
    由于 $A$ 正定,Hessian 矩阵 $\nabla^2 \phi(x) = A$ 正定,故驻点为严格极小值点。
\end{proof}

\subsubsection{Ritz 方法与子空间逼近}
\begin{itemize}
    \item \textbf{基本思想}:在全空间 $\mathbb{R}^n$ 中求极值可能太慢。Ritz 方法试图在一系列逐渐扩大的低维子空间 $V_k$ 中寻找 $\phi(x)$ 的近似极小点。
    \item \textbf{Ritz 问题}:设 $V_k = \text{span}\{p_0, p_1, \dots, p_{k-1}\}$。寻找 $x_k \in x_0 + V_k$ 使得
    \[
        \phi(x_k) = \min_{x \in x_0 + V_k} \phi(x)
    \]
    \item \textbf{几何解释}:$x_k$ 是精确解 $x^*$ 在子空间 $x_0 + V_k$ 上的**A-正交投影**。
    即误差 $e_k = x_k - x^*$ 满足 $e_k \perp_A V_k$($e_k^T A v = 0, \forall v \in V_k$)。
\end{itemize}

\subsubsection{梯度与下降方向}
\begin{itemize}
    \item \textbf{残差与梯度}:$r(x) = b - Ax = - \nabla \phi(x)$。残差即负梯度方向。
    \item \textbf{最速下降方向}:$p_k = r_k$。这是局部下降最快的方向,但在椭球狭长山谷中会导致“锯齿形”振荡,收敛极慢。
    \item \textbf{A-共轭方向}:为克服锯齿效应,我们希望搜索方向 $p_0, p_1, \dots$ 能够使得每一步的搜索互不干扰。这就引入了 **A-正交(共轭)** 的概念:
    \[
        (p_i, p_j)_A = p_i^T A p_j = 0, \quad i \ne j
    \]
    在 A-共轭方向系下,二次型 $\phi(x)$ 的等高线被“拉圆”了,从而可以快速收敛。
\end{itemize}

\subsubsection{一维搜索理论 (One-Dimensional Search)}
在变分方法中,给定当前点 $x_k$ 和下降方向 $p_k$,我们需要确定一个步长 $\alpha_k$,使得目标函数在沿该方向上达到极小。
\[
    x_{k+1} = x_k + \alpha_k p_k
\]
\begin{itemize}
    \item \textbf{一维搜索问题}:
    寻找 $\alpha_k$ 使得 $f(\alpha) = \phi(x_k + \alpha p_k)$ 最小。
    
    \item \textbf{最优步长推导}:
    将 $f(\alpha)$ 对 $\alpha$ 求导并令其为 0:
    \[
        f'(\alpha) = \frac{d}{d\alpha} \phi(x_k + \alpha p_k) = (\nabla \phi(x_k + \alpha p_k), p_k)
    \]
    代入 $\nabla \phi(x) = Ax - b = -r(x)$,得:
    \[
        -(r(x_k + \alpha p_k), p_k) = 0
    \]
    利用残差的线性性质 $r(x_k + \alpha p_k) = b - A(x_k + \alpha p_k) = (b - Ax_k) - \alpha Ap_k = r_k - \alpha Ap_k$:
    \[
        (r_k - \alpha Ap_k, p_k) = 0 \implies (r_k, p_k) - \alpha (Ap_k, p_k) = 0
    \]
    解得最优步长:
    \[
        \alpha_k = \frac{(r_k, p_k)}{(Ap_k, p_k)}
    \]
    \item \textbf{结论}:无论是在最速下降法还是共轭梯度法中,每一步的步长 $\alpha_k$ 都是由上述公式确定的,它保证了新残差 $r_{k+1}$ 与当前搜索方向 $p_k$ 正交。
\end{itemize}

\subsubsection{单调性与收敛性分析}
\textbf{命题}:在给定下降方向 $p_k$(满足 $p_k^T r_k > 0$)并采用最优步长 $\alpha_k$ 后,目标函数 $\phi(x)$、残差范数(在特定条件下)及误差范数(在特定度量下)均呈现单调下降趋势。

\begin{proof}
    \textbf{1. 目标函数值下降 ($\phi(x_{k+1}) < \phi(x_k)$)}:
    将 $x_{k+1} = x_k + \alpha_k p_k$ 代入二次泛函展开:
    \[
        \phi(x_{k+1}) = \phi(x_k) + \alpha_k (Ax_k - b, p_k) + \frac{1}{2} \alpha_k^2 (Ap_k, p_k)
    \]
    由于 $Ax_k - b = -r_k$,所以 $(Ax_k - b, p_k) = -(r_k, p_k)$。
    代入最优步长 $\alpha_k = \frac{(r_k, p_k)}{(Ap_k, p_k)}$:
    \[
    \begin{aligned}
        \phi(x_{k+1}) &= \phi(x_k) - \frac{(r_k, p_k)^2}{(Ap_k, p_k)} + \frac{1}{2} \left( \frac{(r_k, p_k)}{(Ap_k, p_k)} \right)^2 (Ap_k, p_k) \\
                      &= \phi(x_k) - \frac{(r_k, p_k)^2}{2(Ap_k, p_k)}
    \end{aligned}
    \]
    因为 $A$ 正定,$(Ap_k, p_k) > 0$,且 $(r_k, p_k) \ne 0$(否则已收敛),故减项恒正。
    \[
        \therefore \phi(x_{k+1}) < \phi(x_k)
    \]
    这证明了变分函数值是严格单调下降的。

    \textbf{2. 误差的 $A$-范数下降}:
    定义误差 $e_k = x^* - x_k$。注意这里 $\phi(x)$ 与误差的 $A$-范数有直接关系:
    \[
        \phi(x) = \frac{1}{2}(A(x^*-e), x^*-e) - (b, x^*-e) = \phi(x^*) + \frac{1}{2}(Ae, e) = \phi(x^*) + \frac{1}{2}||e||_A^2
    \]
    由于 $\phi(x^*) = \text{const}$ 且 $\phi(x_{k+1}) < \phi(x_k)$,直接推出:
    \[
        ||e_{k+1}||_A < ||e_k||_A
    \]
    即误差在 $A$-范数意义下是单调递减的。

    \textbf{3. 残差的正交性}:
    由一维搜索条件 $f'(\alpha_k)=0$ 可知:
    \[
        (r_{k+1}, p_k) = 0
    \]
    这意味着新残差与当前的搜索方向正交。在几何上,这是在此方向上能达到的“最低点”的特征。
\end{proof}

\subsection{最速下降法 (Steepest Descent)}
\subsubsection{算法流程与性质}
\begin{itemize}
    \item \textbf{下降方向}:取负梯度方向 $p_k = r_k = b - Ax_k$。
    \item \textbf{最优步长}:$\alpha_k = \frac{(r_k, r_k)}{(Ar_k, r_k)}$。
    \item \textbf{迭代公式}:
    \[ x_{k+1} = x_k + \alpha_k r_k \]
    \item \textbf{残差递推公式}:
    \[ r_{k+1} = r_k - \alpha_k A r_k \]
    这避免了每次迭代计算 $b - Ax_{k+1}$ 的矩阵向量乘法。
\end{itemize}

\begin{theorem}[相邻残差正交性]
    在最速下降法中,相邻两步的残差相互正交,即 $(r_{k+1}, r_k) = 0$。
\end{theorem}
\begin{proof}
    由最优步长条件(一维搜索性质)可知 $(r_{k+1}, p_k) = 0$。
    而在最速下降法中,$p_k = r_k$。
    故 $(r_{k+1}, r_k) = 0$。
\end{proof}
\begin{remark}[锯齿现象]
    由于 $r_{k+1} \perp r_k$,意味着搜索方向在每一步都必须拐 90 度弯。在等高线为狭长椭球(条件数 $\kappa$ 很大)的情况下,这会导致算法在“山谷”底部反复震荡,前进极慢,形成锯齿形路径。
\end{remark}

\subsubsection{收敛性定理与证明 (Kantorovich 不等式)}
\begin{theorem}[最速下降法收敛速度]
    设 $A$ 是 SPD 矩阵,特征值为 $\lambda_1 \ge \lambda_2 \ge \dots \ge \lambda_n > 0$。则最速下降法的误差满足:
    \[
        ||e_{k+1}||_A \le \frac{\kappa - 1}{\kappa + 1} ||e_k||_A
    \]
    其中 $\kappa = \text{cond}_2(A) = \frac{\lambda_1}{\lambda_n} = \frac{\lambda_{\max}}{\lambda_{\min}}$。
\end{theorem}

\begin{proofstep}{证明思路}
    \textbf{1. 误差递推关系}:
    由 $x_{k+1} = x_k + \alpha_k r_k$,两边减 $x^*$ 得 $e_{k+1} = e_k - \alpha_k r_k$。
    注意 $r_k = b - Ax_k = A(x^* - x_k) = Ae_k$,代入得:
    \[ e_{k+1} = e_k - \alpha_k Ae_k = (I - \alpha_k A)e_k \]
    
    \textbf{2. 能量范数递推}:
    计算 $||e_{k+1}||_A^2 = (Ae_{k+1}, e_{k+1})$:
    \[
        ||e_{k+1}||_A^2 = (A(e_k - \alpha_k r_k), e_k - \alpha_k r_k)
    \]
    展开并代入最优步长 $\alpha_k = \frac{(r_k, r_k)}{(Ar_k, r_k)}$(推导略繁琐,直接利用函数值下降公式):
    我们已知 $\phi(x_{k+1}) = \phi(x_k) - \frac{(r_k, r_k)^2}{2(Ar_k, r_k)}$。
    利用 $\phi(x) = \phi(x^*) + \frac{1}{2}||e||_A^2$,得:
    \[
        \frac{1}{2}||e_{k+1}||_A^2 = \frac{1}{2}||e_k||_A^2 - \frac{(r_k, r_k)^2}{2(Ar_k, r_k)}
    \]
    于是:
    \[
        \frac{||e_{k+1}||_A^2}{||e_k||_A^2} = 1 - \frac{(r_k, r_k)^2}{(Ar_k, r_k) \cdot ||e_k||_A^2}
    \]
    注意 $||e_k||_A^2 = (Ae_k, e_k) = (r_k, A^{-1}r_k)$,代入上式:
    \[
        \frac{||e_{k+1}||_A^2}{||e_k||_A^2} = 1 - \frac{(r_k, r_k)^2}{(Ar_k, r_k)(A^{-1}r_k, r_k)}
    \]
    
    \textbf{3. 利用 Kantorovich 不等式}:
    对于 SPD 矩阵 $A$,Kantorovich 不等式指出:
    \[
        \frac{(x, x)^2}{(Ax, x)(A^{-1}x, x)} \ge \frac{4\lambda_1 \lambda_n}{(\lambda_1 + \lambda_n)^2}
    \]
    将 $x$ 替换为 $r_k$,代入递推式:
    \[
        \frac{||e_{k+1}||_A^2}{||e_k||_A^2} \le 1 - \frac{4\lambda_1 \lambda_n}{(\lambda_1 + \lambda_n)^2} = \left( \frac{\lambda_1 - \lambda_n}{\lambda_1 + \lambda_n} \right)^2 = \left( \frac{\kappa - 1}{\kappa + 1} \right)^2
    \]
    开方即得证。
\end{proofstep}

\subsection{共轭梯度法 (Conjugate Gradient, CG)}
\subsubsection{基本思想}
\begin{itemize}
    \item \textbf{共轭方向}:寻找一组关于 $A$ 正交(共轭)的方向 $\{p_0, p_1, \dots\}$,即 $p_i^T A p_j = 0 \ (i \ne j)$。
    \item \textbf{最优性}:第 $k$ 步得到的解 $x_k$ 使得 $\phi(x)$ 在 Krylov 子空间 $\mathcal{K}_k(A, r_0) = \text{span}\{r_0, Ar_0, \dots, A^{k-1}r_0\}$ 上达到极小。
\end{itemize}

\subsubsection{CG 算法流程}
初始化 $x_0$,计算 $r_0 = b - Ax_0$,令 $p_0 = r_0$。
对于 $k=0, 1, \dots$:
\begin{enumerate}
    \item $\alpha_k = \frac{r_k^T r_k}{p_k^T A p_k}$ (计算步长)
    \item $x_{k+1} = x_k + \alpha_k p_k$ (更新解)
    \item $r_{k+1} = r_k - \alpha_k A p_k$ (更新残差)
    \item $\beta_k = \frac{r_{k+1}^T r_{k+1}}{r_k^T r_k}$ (计算方向更新系数)
    \item $p_{k+1} = r_{k+1} + \beta_k p_k$ (更新搜索方向)
\end{enumerate}

\subsubsection{收敛性分析}
\begin{itemize}
    \item \textbf{有限步终止}:理论上至多 $n$ 步收敛到精确解(无舍入误差时)。
    \item \textbf{误差估计}:
    \[
        \frac{||e_k||_A}{||e_0||_A} \le 2 \left( \frac{\sqrt{\kappa}-1}{\sqrt{\kappa}+1} \right)^k
    \]
    收敛速度远快于最速下降法,且依赖于 $\sqrt{\text{cond}(A)}$。
\end{itemize}

\subsection{预处理技术 (Preconditioning)}
\begin{itemize}
    \item \textbf{目的}:改善矩阵条件数,加速 CG 收敛。
    \item \textbf{方法}:求解等价方程 $M^{-1} A x = M^{-1} b$,其中 $M \approx A$ 且 $M$ 易求逆。
    \item \textbf{预优 CG (PCG)}:在 CG算法中引入 $M^{-1}$,实际上是在 $M$-内积下进行正交化。
    \item \textbf{常用预条件子}:Jacobi (对角Scaling), Incomplete Cholesky (IC)。
\end{itemize}

\tofill[]{从PPT和作业来看,预处理CG的细节不是考试重点,大概考不了...艰深而收获小,摆了,建议大家多花点精力复习别的章节(}









\end{document}