\documentclass{book}

% 中文字体支持
\usepackage{ctex}
\usepackage{xeCJK}

% 数学公式支持
\usepackage{amsmath,amssymb,amsthm}
\usepackage{mathtools}
\usepackage{bm} % 粗体数学符号

% 图表支持
\usepackage{graphicx}
\usepackage{tikz}
\usepackage{pgfplots}
\pgfplotsset{compat=1.17}
\usepackage{float}

% 页面布局
\usepackage{geometry}
\geometry{left=2.5cm,right=2.5cm,top=3cm,bottom=3cm}
\usepackage{fancyhdr}
\usepackage{lastpage}

% 目录和超链接
\usepackage{hyperref}
\hypersetup{
    colorlinks=true,
    linkcolor=blue,
    filecolor=magenta,      
    urlcolor=cyan,
    citecolor=red
}

% 代码支持
\usepackage{listings}
\usepackage{xcolor}

% 表格支持
\usepackage{booktabs}
\usepackage{multirow}
\usepackage{array}

% % 副标题
% \usepackage{titling}

% 定理环境
\newtheorem{theorem}{定理}[section]
\newtheorem{lemma}[theorem]{引理}
\newtheorem{corollary}[theorem]{推论}
\newtheorem{proposition}[theorem]{命题}
\newtheorem{definition}[theorem]{定义}
\newtheorem{example}[theorem]{例题}
\newtheorem{exercise}[theorem]{练习}

% 待填写内容标记
\NewDocumentCommand{\tofill}{O{} m}{
    \textbf{\textit{\textcolor{red}{待填写:\IfValueT{#1}{(#1)}#2}}}
}

% 页眉页脚设置
\setlength{\headheight}{14.5pt}
\pagestyle{fancy}
\fancyhf{}
\fancyhead[L]{数值分析课程总结}
\fancyhead[R]{\today}
\fancyfoot[C]{\thepage/\pageref{LastPage}}

% 代码样式设置
\lstset{
    basicstyle=\ttfamily\small,
    keywordstyle=\color{blue}\bfseries,
    commentstyle=\color{green!60!black},
    stringstyle=\color{red},
    showstringspaces=false,
    numbers=left,
    numberstyle=\tiny\color{gray},
    frame=single,
    breaklines=true
}

% 文档信息
\title{Numerical Analysis \\ Exam Minimum}
\author{Astral Projection}
\date{\today}


\begin{document}

\maketitle
\tableofcontents
\newpage

\chapter {数学基础知识}

\section {核心概念与理论}

\subsection {线性空间}
\subsubsection {定义与性质}


\begin{definition}[线性空间]
    设S是一个集合,P是一个数域($\mathbb{R}$或$\mathbb{C}$). 定义两种映射关系:
    \begin{itemize}
        \item 向量加法:$+: S \times S \to S$
        \item 数乘:$\cdot : P \times S \to S$
    \end{itemize}
    如果对任意的$u,v,w \in S$和$a,b \in P$,满足以下八条公理,则称(S,P)为一个线性空间(向量空间):
    \begin{enumerate}
        \item 加法交换律:$u + v = v + u$
        \item 加法结合律:$(u + v) + w = u + (v + w)$
        \item 存在加法单位元:存在零向量$0 \in S$,使得对任意$v \in S$,有$v + 0 = v$
        \item 存在加法逆元:对任意$v \in S$,存在$-v \in S$,使得$v + (-v) = 0$
        \item 数乘结合律:$a(bv) = (ab)v$
        \item 数乘分配律1:$a(u + v) = au + av$
        \item 数乘分配律2:$(a + b)v = av + bv$
        \item 数乘单位元:$1v = v$
        \end{enumerate}
        则称(S,P)构成一个线性空间。
\end{definition}
此外,如果对于给定空间的运算法则和数域是不言自明的,则通常简写为S是一个线性空间。
如我们说$\mathbb{R}^n$是一个线性空间,通常指$(\mathbb{R}^n,\mathbb{R})$是一个线性空间或$(\mathbb{R}^n,\mathbb{C})$是一个线性空间,具体取决于数域的选择。

\subsubsection {线性无关与相关}
\tofill[定义]{线性无关与线性相关}

\subsubsection {基、框架与维数}

\tofill[定义]{基、框架与维数}

性质:
\begin{itemize}
    \item 空间的维度是一个内蕴量,与基的选择无关
    \item 多项式空间$P_N$中,$\{1,x,x^2,\ldots,x^N\}$构成其一组基,维数为$\dim P_N=N+1$
    \item 连续函数空间$C[a,b]$中,$\forall N$,$\{1,x,x^2,\ldots,x^N\}$是线性无关的,但不能构成其基,因其维数为无穷大
\end{itemize}

\subsection {度量与赋范空间}
\subsubsection {距离空间}
\begin{definition}[距离空间]
    设M是一个集合,$d: M \times M \to \mathbb{R}$是一个映射,如果对任意的$x,y,z \in M$,满足以下三条公理,则称(M,d)为一个距离空间:
    \begin{enumerate}
        \item 非负性与分离性:$d(x,y) \geq 0$,且当且仅当$x=y$时,$d(x,y)=0$
        \item 对称性:$d(x,y) = d(y,x)$
        \item 三角不等式:$d(x,z) \leq d(x,y) + d(y,z)$
    \end{enumerate}
    则称(M,d)构成一个距离空间。
\end{definition}

\subsubsection{距离空间的完备性}

\tofill[定义]{完备性}

$\mathbb{R}$是完备的,且任意有限维赋范空间都是完备的。

\paragraph{构造方法:距离空间的完备化}
设$(M,d)$是一个距离空间,可以按照如下过程构造其完备化空间:
\begin{enumerate}
    \item 构造对偶的柯西列空间
    \begin{align}
        \tilde{M}={\{(x_n)\,|\,x_n \in M,\quad (x_n) \text{为柯西列}\}}
    \end{align}
    \item 在柯西列空间$\tilde{M}$中定义等价关系
    \begin{align}
        \tilde{x}\sim\tilde{y}\leftrightarrow \lim_{n\to\infty} d(x_n,y_n)=0
    \end{align}
    即这两个柯西列按照角标顺序,交叉放在一起,还是柯西列。
    \item 构造商空间:
    \begin{align}
        \hat{M}=\tilde{M}/\sim=\{ [\tilde{x}] \}
    \end{align}
    式中,$[\tilde{x}]$表示柯西列$\tilde{x}$的等价类,即$[\tilde{x}]$是一个集合,
    集合中的所有元素在等价关系$\sim$下都是等价的。
    \item 在商空间$\hat{M}$中定义距离
    \begin{align}
        \hat{d}([\tilde{x}],[\tilde{y}])=\lim_{n\to\infty} d(x_n,y_n)
    \end{align}
    \item 则$(\hat{M},\hat{d})$即为距离空间$(M,d)$的完备化空间。
\end{enumerate}

嵌入映射:可以在原空间$M$与完备化空间$\hat{M}$之间定义一个单射$i$:
\begin{align}
    i: M \to \hat{M},\quad i(x)=[(x,x,x,\ldots)]
\end{align}
该映射将原空间中的每个点$x$映射为完备化空间中由常值序列$(x,x,x,\ldots)$所构成的等价类,
且\textbf{映射前后任意两元素的距离不变}





\subsubsection {赋范空间与 Banach 空间}
\tofill[定义]{赋范空间} 

完备的赋范空间称为Banach空间,或者B空间。


\subsubsection {等价范数}
\begin{definition}[范数的等价性]
    设$V$是一个线性空间,$\|\cdot\|_1$和$\|\cdot\|_2$是$V$上的两个范数,如果存在正常数$c$和$C$,使得对任意$v \in V$,都有
    \begin{align}
        c\|v\|_1 \leq \|v\|_2 \leq C\|v\|_1
    \end{align}
    则称这两个范数是等价的。

    若存在正数C,使得对任意$v \in V$,都有
    \begin{align}
        \|v\|_2 \leq C\|v\|_1
    \end{align}
    则称范数$\|\cdot\|_1$强于$\|\cdot\|_2$。
\end{definition}

性质:
\begin{itemize}
    \item 在有限维线性空间上,任意两个范数都是等价的
    \item 在无限维线性空间上,范数不一定是等价的
    \item 若一个点列在较强的范数下是Cauchy列,则在较弱的范数下也是Cauchy列;反之不必然。
\end{itemize}

\subsubsection{常用的范数}
\paragraph{$\mathbb{R}^n$上的范数} 记$x=(x_1,x_2,\ldots,x_n)^T \in \mathbb{R}^n$,则常用的范数有:
\begin{itemize}
    \item 无穷范数:所有元素的最大值
    \begin{align}
        \|x\|_{\infty}=\max_{1 \leq i \leq n} |x_i|
    \end{align}
    \item 1-范数:所有元素的绝对值之和
    \begin{align}
        \|x\|_{1}=\sum_{i=1}^{n} |x_i|
    \end{align}
    \item 2-范数:欧几里得范数,即所有元素的平方和的平方根
    \begin{align}
        \|x\|_{2}=\left(\sum_{i=1}^{n} |x_i|^2\right)^{\frac{1}{2}}
    \end{align}
\end{itemize}

\paragraph{$C[a,b]$(有界闭区间上连续函数空间)上的范数}
\begin{itemize}
    \item 无穷范数:函数在区间上的最大绝对值
    \begin{align}
        \|f\|_{\infty}=\max_{a \leq x \leq b} |f(x)|
    \end{align}
    \item 1-范数:函数在区间上的绝对值积分
    \begin{align}
        \|f\|_{1}=\int_{a}^{b} |f(x)| \, dx
    \end{align}
    \item 2-范数:函数在区间上的平方积分的平方根
    \begin{align}
        \|f\|_{2}=\left(\int_{a}^{b} |f(x)|^2 \, dx\right)^{\frac{1}{2}}
    \end{align}
\end{itemize}

$C[a,b]$上三个范数的性质:

\begin{itemize}
    \item 任意两个范数不等价
    \item 无穷范数强于2范数,2范数强于1范数
    \item 只有无穷范数对应的赋范空间是完备的
    \item 1-范数对应的完备化空间为$L^1(a,b)$,2-范数对应的完备化空间为$L^2(a,b)$ 
        \footnote{$L^1(a,b)$为(a,b)上的可积函数空间,$L^2(a,b)$为(a,b)上的平方可积函数空间。}
\end{itemize}

\subsection {内积空间}
\subsubsection {内积定义与性质}
\begin{definition}[内积]
    设$(S,P)$是一个线性空间,如果对任意的$u,v,w \in S$和$a,b \in P$,存在一个映射$S\times S\rightarrow P$,满足 
    \begin{enumerate}
        \item 共轭对称性:$\langle u,v \rangle = \overline{\langle v,u \rangle}$
        \item 线性性:$\langle au + bv,w \rangle = a\langle u,w \rangle + b\langle v,w \rangle$
        \item 正定性:$\langle v,v \rangle \geq 0$,且当且仅当$v=0$时,$\langle v,v \rangle = 0$
    \end{enumerate}
    则称该映射为内积,$(S,P)$构成一个内积空间。
\end{definition}

若$\langle x,y \rangle =0$,则称$x$与$y$正交。

几个常用空间上的内积:
\begin{itemize}
    \item $\mathbb{R}^n$或$\mathbb{C}^n$上的内积
    \begin{align}
        \langle x,y \rangle = \sum_{i=1}^{n} x_i \overline{y_i}
    \end{align}
    \item $C[a,b]$(有界闭区间上连续函数空间)上的内积
    \begin{align}
        \langle f,g \rangle = \int_{a}^{b} f(x) \overline{g(x)} \, dx
    \end{align}
    \item $C[a,b]$上的带权内积:
    \begin{align}
        (f,g)=\int_a^b \rho(x) f(x) \overline{g(x)} \, dx
    \end{align}
    其中,权函数$\rho(x)$需要满足条件:
    \begin{itemize}
        \item $\rho(x)\in C[a,b]$
        \item $\rho(x)$几乎处处为正
        \item $\int_a^b \rho(x) \, dx < +\infty$
        \item $\forall q(x) \in P_n$,$\int_a^b \rho(x) |q(x)|\mathrm{d}x < \infty$
    \end{itemize}
    带权内积所研究的空间称为加权内积空间:
    \begin{align}
        L^2_{\rho}(a,b) = \{ f(x) \,|\, \int_a^b \rho(x) |f(x)|^2 \, dx < +\infty \}
    \end{align}
\end{itemize}

常用的权函数有:
\begin{align}
    \rho(x) = 1, & \quad [a,b] = [-1,1]\\
    \rho(x)=\frac{1}{1-x^2}, & \quad [a,b] = [-1,1]\\
\end{align}


\subsubsection {正交性与 Schmidt 正交化}
\tofill[定义]{正交性}

\tofill[方法]{用Grammer矩阵判断内积空间中向量组的线性无关性}

\tofill[方法]{Schmidt 正交化过程:从一个线性无关向量组构造一个正交向量组:让每个向量减去与已有空间垂直的分量}

用Schmidt正交化过程得到的正交向量组具有以下性质:
\begin{align}
    \Phi_{k-1} \subset \Phi_k\\
    y_k \perp \Phi_{k-1}
\end{align}

\subsubsection {由内积诱导的范数}
\begin{definition}[诱导范数]
    设$(S,P)$是一个内积空间,则可以定义范数$\|\cdot\|$如下:
    \begin{align}
        \|v\|=\sqrt{\langle v,v \rangle}
    \end{align}
    则称该范数为由内积诱导的范数。
\end{definition}

\begin{itemize}
    \item 任何内积均能诱导对应的范数
    \item 当且仅当范数满足平行四边形法则时
    \begin{align}
        \|f+g\|^2 + \|f-g\|^2 = 2\|f\|^2 + 2\|g\|^2
    \end{align}
    范数可以诱导内积:
    \begin{align}
        (x,y)=\frac{1}{4}(\|x+y\|^2 - \|x-y\|^2 + i\|x+iy\|^2 - i\|x-iy\|^2)
    \end{align}
\end{itemize}


\subsection {正交多项式}
\begin{definition}
[正交多项式]
    设$\{\phi_n(x)\}$是定义在区间$[a,b]$上的一组多项式,且每个多项式的次数为$n$,如果对任意$m \neq n$,都有
    \begin{align}
        \int_a^b \rho(x) \phi_m(x) \phi_n(x) \, dx = 0
    \end{align}
    则称$\{\phi_n(x)\}$为区间$[a,b]$上关于权函数$\rho(x)$的正交多项式。
\end{definition}

正交多项式的性质:
\begin{itemize}%[]
    \item $\deg \phi_i=i$
    \item $(\phi_i,\,\phi_j)=0,\quad \forall i\neq j$
    \item $\phi_n$为实系数多项式
    \item $\phi_n$在开区间$(a,b)$内恰有n个实单根
\end{itemize}

\tofill[证明]{$\phi_n$在开区间$(a,b)$内恰有n个实单根的证明。证法:分别证明实根、单根、全在$(a,b)$内。3个命题均可用反证法。}


\subsubsection {$\rho=1$:Legendre 多项式}
产生方法:
\begin{itemize}
    \item 权函数$\rho(x)=1$
    \item 区间$[-1,1]$
\end{itemize}

表达式:
\begin{align}
    P_n(x)=\frac{1}{2^nn!}\frac{\mathrm{d}^n}{\mathrm{d}x^n}[(x^2-1)^n]
\end{align}

性质:
\begin{itemize}
    \item 首项系数:
    \begin{align}
        k_n=\frac{(2n)!}{2^n(n!)^2}
    \end{align}
    \item 正交归一化:
    \begin{align}
        \int_{-1}^{1} P_n(x)P_m(x) \, dx = \frac{2}{2n+1} \delta_{mn}
    \end{align}
    \item 三项递推关系:
    \begin{align}
        (n+1)P_{n+1}(x)=(2n+1)xP_n(x)-nP_{n-1}(x)
    \end{align}
    \item 奇偶性:
    \begin{align}
        P_n(-x)=(-1)^n P_n(x)
    \end{align}
    \item 导数关系:
    \begin{align}
        \frac{\mathrm{d}}{\mathrm{d}x} P_n(x) = \frac{n}{x^2-1} [xP_n(x) - P_{n-1}(x)]
    \end{align}
    \item 前五项:
    \begin{align}
        &P_0(x)=1\\
        &P_1(x)=x\\
        &P_2(x)=\frac{1}{2}(3x^2-1)\\
        &P_3(x)=\frac{1}{2}(5x^3-3x)\\
        &P_4(x)=\frac{1}{8}(35x^4-30x^2+3)
    \end{align}
\end{itemize}



\subsubsection {Chebyshev 多项式}
产生方法:
\begin{itemize}
    \item 权函数$\rho(x)=\frac{1}{\sqrt{1-x^2}}$
    \item 区间$[-1,1]$
\end{itemize}

表达式:
\begin{align}
    T_n(x)=\cos(n \arccos x)
\end{align}

性质:
\begin{itemize}
    \item 首项系数:$2^{n-1}$
    \item 正交归一化:
    \begin{align}
        \int_{-1}^{1} \frac{T_n(x)T_m(x)}{\sqrt{1-x^2}} \, dx = \begin{cases}
            \pi, & n=m=0\\
            \frac{\pi}{2}, & n=m\neq 0\\
            0, & n\neq m
        \end{cases}
    \end{align}
    \item 三项递推关系:
    \begin{align}
        T_{n+1}=2xT_n-T_{n-1}
    \end{align}
    \item 奇偶性:
    \begin{align}
        T_n(-x)=(-1)^n T_n(x)
    \end{align}
    \item 前五项:
    \begin{align}
        &T_0(x)=1\\
        &T_1(x)=x\\
        &T_2(x)=2x^2-1\\
        &T_3(x)=4x^3-3x\\
        &T_4(x)=8x^4-8x^2+1
    \end{align}
    \item 零点:
    \begin{align}
        x_k=\cos\left(\frac{2k-1}{2n}\pi\right),\quad k=1,2,\ldots,n
    \end{align}
    \item 极值点:
    \begin{align}
        x_k=\cos\left(\frac{k\pi}{n}\right),\quad k=0,1,\ldots,n
    \end{align}
    \item 简单表达式:当$|x|\geq 1$时,
    \begin{align}
        T_n(x)=\frac{1}{2}\left[ \left( x + \sqrt{x^2-1} \right)^n + \left( x - \sqrt{x^2-1} \right)^n \right]
    \end{align}
\end{itemize}




\subsection {矩阵空间}
\tofill[性质]{矩阵空间的基本性质:线性空间、乘法运算、代数性质}


\subsubsection {矩阵范数}
\begin{definition}[矩阵范数]
    矩阵空间$\mathbb{C}^{n\times n}$上的范数$\|\cdot\|$称为矩阵范数,如果对任意的$A,B \in \mathbb{C}^{n\times n}$和$a \in \mathbb{C}$,满足以下性质:
    \begin{enumerate}
        \item 非负性与分离性:$\|A\| \geq 0$,且当且仅当$A=0$时,$\|A\|=0$
        \item 齐次性:$\|aA\| = |a| \|A\|$
        \item 三角不等式:$\|A+B\| \leq \|A\| + \|B\|$
        \item 次乘性:$\|AB\| \leq \|A\| \cdot \|B\|$
    \end{enumerate}
\end{definition}

Note: 矩阵范数是定义在矩阵代数而非矩阵空间上的,必须与矩阵乘法相容。

\begin{definition}[矩阵范数与向量范数的相容性]
    设$\|\cdot\|_v$是向量空间$\mathbb{C}^n$上的一个范数,$\|\cdot\|_m$是矩阵空间$\mathbb{C}^{n\times n}$上的一个范数,如果对任意的$A \in \mathbb{C}^{n\times n}$和$x \in \mathbb{C}^n$,都有
    \begin{align}
        \|Ax\|_v \leq \|A\|_m \cdot \|x\|_v
    \end{align}
    则称矩阵范数$\|\cdot\|_m$与向量范数$\|\cdot\|_v$是相容的。
\end{definition}

矩阵范数的两种常见构造方法:
\begin{itemize}
    \item 直接构造:Frobenius范数
    \begin{align}
        \|A\|_F=\left( \sum_{i=1}^{n} \sum_{j=1}^{n} |a_{ij}|^2 \right)^{\frac{1}{2}}
    \end{align}
    Frobenius范数的性质:
    \begin{itemize}
        \item Frobenius范数与向量2-范数相容
        \item $\|I\|_F=\sqrt{n}$
    \end{itemize}
    \item 向量范数诱导:算子范数
    \begin{definition}
    [算子范数]
        设$\|\cdot\|_v$是向量空间$\mathbb{C}^n$上的一个范数,则可以定义矩阵空间$\mathbb{C}^{n\times n}$上的算子范数$\|\cdot\|_m$如下:
        \begin{align}
            \|A\|_m = \max_{x \neq 0} \frac{\|Ax\|_v}{\|x\|_v} = \max_{\|x\|_v=1} \|Ax\|_v
        \end{align}
        则称该范数为由向量范数$\|\cdot\|_v$诱导的算子范数。
    \end{definition}
\end{itemize}

常用的几个算子范数:
\begin{itemize}
    \item 无穷范数:行和最大值
    \begin{align}
        \|A\|_{\infty} = \max_{1 \leq i \leq n} \sum_{j=1}^{n} |a_{ij}|
    \end{align}
    \item 1-范数:列和最大值
    \begin{align}
        \|A\|_{1} = \max_{1 \leq j \leq n} \sum_{i=1}^{n} |a_{ij}|
    \end{align}
    \item 2-范数(谱范数):$A$的最大奇异值,即$A^HA$的最大特征值的平方根
    \begin{align}
        \|A\|_{2} = \sqrt{\lambda_{\max}(A^HA)}
    \end{align}
\end{itemize}


\subsubsection {谱半径}
\begin{definition}[谱半径]
    谱半径定义为矩阵所有特征值模的最大值,即
    \begin{align}
        \rho(A) = \max_{1 \leq i \leq n} |\lambda_i|
    \end{align}
\end{definition}

谱半径和矩阵范数的关系:
\begin{itemize}
    \item 矩阵范数下界:
    \begin{theorem}
        对任意$A\in\mathbb{C}^{n\times n}$,有
        \begin{align}
            \rho(A) \leq \|A\|
        \end{align}
    \end{theorem}
    \item 无穷接近范数的存在性:
    \begin{theorem}
        对任意$A\in\mathbb{C}^{n\times n}$,存在一个矩阵范数$\|\cdot\|$,使得
        \begin{align}
            \rho(A) \leq \|A\| \leq \rho(A) + \varepsilon
        \end{align}
        其中,$\varepsilon$为任意给定的正常数。
    \end{theorem}
\end{itemize}


\subsubsection {可逆矩阵相关定理}
\begin{theorem}[扰动引理I]
    给定$B\in\mathbb{C}^{n\times n}$。设$\|B\|<1$,则$I+B$可逆,且
    \begin{align}
        \|(I+B)^{-1}\| \leq \frac{1}{1-\|B\|}
    \end{align}
\end{theorem}

\begin{theorem}
    [扰动引理II]
    设$A,\,C\in\mathbb{C}^{n\times n}$,且$A$可逆。若 
    \begin{align}
        \|C-A\| < \frac{1}{\|A^{-1}\|}
    \end{align}
    则$C$也可逆,且
    \begin{align}
        \|C^{-1}\| \leq \frac{\|A^{-1}\|}{1 - \|A^{-1}\| \cdot \|C-A\|}
    \end{align}
\end{theorem}


\chapter {函数逼近}
\section {通用理论}
\subsection {问题模型}
\subsection {逼近准则}
\subsubsection {最小二乘准则(L₂范数)}
\subsubsection {一致逼近准则(L∞范数)}
\subsection {核心定理}
\section {具体逼近方法}
\subsection {最优平方逼近}
\subsubsection {线性最小二乘}
\subsubsection {加权平方逼近}
\subsubsection {正交多项式逼近}
\subsection {最优一致逼近}


\chapter {函数插值与重构}
\section {通用理论}
\subsection {问题模型}
\subsection {插值空间}
\subsection {误差分析与收敛性}
\section {具体插值方法}
\subsection {一维多项式插值}
\subsubsection {Lagrange 插值}
\paragraph {基函数构造}
\paragraph {插值公式}
\paragraph {余项}
\subsubsection {Newton 插值}
\paragraph {差商定义}
\paragraph {插值公式}
\subsubsection {Hermite 插值}
\subsection {分段插值}
\subsubsection {分段线性插值}
\subsubsection {分段三次 Hermite 插值}
\subsection {Fourier 插值}
\subsubsection {三角多项式插值}
\subsubsection {离散 Fourier 变换关联}



\chapter {数值微积分}
\section {数值积分}
\subsection {通用理论}
\subsubsection {积分问题模型}
\subsubsection {求积公式核心}
\subsubsection {代数精度}
\subsubsection {稳定性}
\subsection {具体求积方法}
\subsubsection {Newton-Cotes 公式}
\paragraph {梯形公式}
\paragraph {Simpson 公式}
\paragraph {3/8 - 规则}
\subsubsection {复合求积方法}
\paragraph {复合梯形公式}
\paragraph {复合 Simpson 公式}
\subsubsection {加速方法}
\paragraph {Romberg 方法}
\subsubsection {Gauss 型求积}
\paragraph {Gauss 点与权系数}
\paragraph {Gauss-Legendre 求积}
\paragraph {Gauss-Chebyshev 求积}
\subsubsection {特殊积分处理}
\paragraph {奇异积分}
\paragraph {振荡积分}
\subsubsection {高维积分}
\paragraph {蒙特卡洛方法}
\section {数值微分}
\subsection {基础方法}
\subsubsection {向前差商}
\subsubsection {向后差商}
\subsubsection {中心差商}
\subsection {高精度方法}
\subsubsection {插值型数值微分}
\subsubsection {Richardson 外推加速}


\chapter {非线性方程求根}


\section {通用理论}
\subsection {问题模型}
\subsection {迭代法基础}
\subsection {收敛性分析}
\subsubsection {收敛阶定义}
\subsubsection {整体收敛性(压缩映像原理)}
\subsubsection {局部收敛性判定}
\subsection {收敛效率}
\section {具体方法}
\subsection {单步法}
\subsubsection {牛顿法}
\paragraph {迭代公式}
\paragraph {收敛性分析}
\paragraph {重根处理}
\subsubsection {不动点迭代}
\subsection {多步法}
\subsubsection {割线法}
\subsection {其他方法}
\subsubsection {Steffensen 方法}
\subsubsection {Broyden 秩 1 方法}




\chapter {常微分方程初值问题数值解法}

\section {通用理论}

\subsection {问题模型}
\subsection {数值解法核心}
\subsection {基本概念}
\subsubsection {局部截断误差与整体截断误差}
\subsubsection {收敛性与收敛阶}
\subsubsection {稳定性与相容性}
\subsection {收敛性与稳定性判定}


\section {具体方法}

\subsection {单步法}
\subsubsection {Euler 方法}
\paragraph {向前欧拉}
\paragraph {向后欧拉}
\subsubsection {改进欧拉方法}
\subsubsection {龙格 - 库塔(RK)方法}
\paragraph {2 阶 RK 方法}
\paragraph {4 阶 RK 方法}

\subsection {多步法}
\subsubsection {Adams 方法}
\paragraph {显式 Adams 公式}
\paragraph {隐式 Adams 公式}
\subsubsection {Nyström 方法}
\subsubsection {Milne-Simpson 方法}





\chapter {线性代数方程组数值解法}
\section {直接解法}
\subsection {通用理论}
\subsubsection {问题模型}
\subsubsection {残差与误差分析}
\subsubsection {数值稳定性与条件数}
\subsection {具体方法}
\subsubsection {Gauss 消元法}
\paragraph {基本步骤}
\paragraph {主元素选取}
\subsubsection {LU 分解}
\subsubsection {特殊方程组解法}
\paragraph {Thomas 算法(三对角方程组)}
\paragraph {Toeplitz 矩阵快速解法}
\section {定常线性迭代解法}
\subsection {通用理论}
\subsubsection {迭代格式构造}
\subsubsection {收敛性判定}
\subsubsection {收敛速度}
\subsection {具体迭代法}
\subsubsection {Jacobi 迭代}
\subsubsection {Gauss-Seidel 迭代}
\subsubsection {SOR 迭代}
\paragraph {松弛因子选取}
\paragraph {最优松弛因子}
\section {非线性迭代解法(基于变分原理)}
\subsection {通用理论}
\subsubsection {Ritz 变分原理}
\subsubsection {迭代核心思想}
\subsection {具体方法}
\subsubsection {最速下降法}
\subsubsection {共轭梯度法(CG)}
\paragraph {迭代公式}
\paragraph {收敛性分析}
\subsubsection {预条件技术}
\paragraph {预条件矩阵选取}
\paragraph {预条件 CG 迭代格式}







\end{document}