\documentclass{book}

% 中文字体支持
\usepackage{ctex}
\usepackage{xeCJK}

% 标题字号设置(使用 titlesec,依次减小字号)
\usepackage{titlesec}
\titleformat{\chapter}{\zihao{-2}\bfseries}{\thechapter}{1em}{}
\titleformat{\section}{\zihao{3}\bfseries}{\thesection}{0.8em}{}
\titleformat{\subsection}{\zihao{4}\bfseries}{\thesubsection}{0.7em}{}
\titleformat{\subsubsection}{\zihao{-4}\bfseries}{\thesubsubsection}{0.6em}{}
\titleformat{\paragraph}{\zihao{5}\bfseries}{\theparagraph}{0.5em}{}
% 各级标题编号格式
% chapter: 一 二 三 ...
\renewcommand{\thechapter}{\zhnum{chapter}}
% section: (一)(二)(三)...
\renewcommand{\thesection}{(\zhnum{section})}
% subsection: 1. 2. 3. ...
\renewcommand{\thesubsection}{\arabic{subsection}.}
% subsubsection: (1)(2)(3)...
\renewcommand{\thesubsubsection}{(\arabic{subsubsection})}
% paragraph: a. b. c. ...
\renewcommand{\theparagraph}{\alph{paragraph}.}
% 允许编号到 paragraph 层级
\setcounter{secnumdepth}{4}
% 数学公式支持
\usepackage{amsmath,amssymb,amsthm}
\usepackage{mathtools}
\usepackage{bm} % 粗体数学符号

% 取消所有公式编号:通过修改计数器深度实现
\usepackage{etoolbox}
\AtBeginEnvironment{equation}{\nonumber}
\AtBeginEnvironment{equation*}{\nonumber}
\AtBeginEnvironment{align}{\nonumber}
\AtBeginEnvironment{align*}{\nonumber}
\AtBeginEnvironment{gather}{\nonumber}
\AtBeginEnvironment{gather*}{\nonumber}
\AtBeginEnvironment{multline}{\nonumber}
\AtBeginEnvironment{multline*}{\nonumber}

% 图表支持
\usepackage{graphicx}
\usepackage{tikz}
\usepackage{pgfplots}
\pgfplotsset{compat=1.17}
\usepackage{float}

% 页面布局
\usepackage{geometry}
\geometry{left=2.5cm,right=2.5cm,top=3cm,bottom=3cm}
\usepackage{fancyhdr}
\usepackage{lastpage}

% 目录和超链接
\usepackage{hyperref}
\hypersetup{
    colorlinks=true,
    linkcolor=blue,
    filecolor=magenta,      
    urlcolor=cyan,
    citecolor=red
}

% 代码支持
\usepackage{listings}
\usepackage{xcolor}

% 表格支持
\usepackage{booktabs}
\usepackage{multirow}
\usepackage{array}

% 特殊形状支持
\usepackage{xcolor}

% % 副标题
% \usepackage{titling}

% 定理环境(取消编号)
\newtheorem*{theorem}{定理}
\newtheorem*{lemma}{引理}
\newtheorem*{corollary}{推论}
\newtheorem*{proposition}{命题}
\newtheorem*{definition}{定义}
\newtheorem*{example}{例题}
\newtheorem*{exercise}{练习}

% 待填写内容标记
\NewDocumentCommand{\tofill}{O{} m}{
    \textbf{\textit{\textcolor{red}{待填写:\IfValueT{#1}{(#1)}#2}}}
}

% 页眉页脚设置
\setlength{\headheight}{14.5pt}
\pagestyle{fancy}
\fancyhf{}
\fancyhead[L]{数值分析课程总结}
\fancyhead[R]{\today}
\fancyfoot[C]{\thepage/\pageref{LastPage}}

% 代码样式设置
\lstset{
    basicstyle=\ttfamily\small,
    keywordstyle=\color{blue}\bfseries,
    commentstyle=\color{green!60!black},
    stringstyle=\color{red},
    showstringspaces=false,
    numbers=left,
    numberstyle=\tiny\color{gray},
    frame=single,
    breaklines=true
}

% 文档信息
\title{Numerical Analysis \\ Exam Minimum}
\author{Astral Projection}
\date{\today}


\begin{document}

\maketitle
\tableofcontents
\newpage

\chapter {数学基础知识}

\section {核心概念与理论}

\subsection {线性空间}
\subsubsection {定义与性质}


\begin{definition}[线性空间]
    设S是一个集合,P是一个数域($\mathbb{R}$或$\mathbb{C}$). 定义两种映射关系:
    \begin{itemize}
        \item 向量加法:$+: S \times S \to S$
        \item 数乘:$\cdot : P \times S \to S$
    \end{itemize}
    如果对任意的$u,v,w \in S$和$a,b \in P$,满足以下八条公理,则称(S,P)为一个线性空间(向量空间):
    \begin{enumerate}
        \item 加法交换律:$u + v = v + u$
        \item 加法结合律:$(u + v) + w = u + (v + w)$
        \item 存在加法单位元:存在零向量$0 \in S$,使得对任意$v \in S$,有$v + 0 = v$
        \item 存在加法逆元:对任意$v \in S$,存在$-v \in S$,使得$v + (-v) = 0$
        \item 数乘结合律:$a(bv) = (ab)v$
        \item 数乘分配律1:$a(u + v) = au + av$
        \item 数乘分配律2:$(a + b)v = av + bv$
        \item 数乘单位元:$1v = v$
        \end{enumerate}
        则称(S,P)构成一个线性空间。
\end{definition}
此外,如果对于给定空间的运算法则和数域是不言自明的,则通常简写为S是一个线性空间。
如我们说$\mathbb{R}^n$是一个线性空间,通常指$(\mathbb{R}^n,\mathbb{R})$是一个线性空间或$(\mathbb{R}^n,\mathbb{C})$是一个线性空间,具体取决于数域的选择。

\subsubsection {线性无关与相关}
\tofill[定义]{线性无关与线性相关}

\subsubsection {基、框架与维数}

\tofill[定义]{基、框架与维数}

性质:
\begin{itemize}
    \item 空间的维度是一个内蕴量,与基的选择无关
    \item 多项式空间$P_N$中,$\{1,x,x^2,\ldots,x^N\}$构成其一组基,维数为$\dim P_N=N+1$
    \item 连续函数空间$C[a,b]$中,$\forall N$,$\{1,x,x^2,\ldots,x^N\}$是线性无关的,但不能构成其基,因其维数为无穷大
\end{itemize}

\subsection {度量与赋范空间}
\subsubsection {距离空间}
\begin{definition}[距离空间]
    设M是一个集合,$d: M \times M \to \mathbb{R}$是一个映射,如果对任意的$x,y,z \in M$,满足以下三条公理,则称(M,d)为一个距离空间:
    \begin{enumerate}
        \item 非负性与分离性:$d(x,y) \geq 0$,且当且仅当$x=y$时,$d(x,y)=0$
        \item 对称性:$d(x,y) = d(y,x)$
        \item 三角不等式:$d(x,z) \leq d(x,y) + d(y,z)$
    \end{enumerate}
    则称(M,d)构成一个距离空间。
\end{definition}

\subsubsection{距离空间的完备性}

\tofill[定义]{完备性}

$\mathbb{R}$是完备的,且任意有限维赋范空间都是完备的。

\paragraph{构造方法:距离空间的完备化}
设$(M,d)$是一个距离空间,可以按照如下过程构造其完备化空间:
\begin{enumerate}
    \item 构造对偶的柯西列空间
    \begin{align}
        \tilde{M}={\{(x_n)\,|\,x_n \in M,\quad (x_n) \text{为柯西列}\}}
    \end{align}
    \item 在柯西列空间$\tilde{M}$中定义等价关系
    \begin{align}
        \tilde{x}\sim\tilde{y}\leftrightarrow \lim_{n\to\infty} d(x_n,y_n)=0
    \end{align}
    即这两个柯西列按照角标顺序,交叉放在一起,还是柯西列。
    \item 构造商空间:
    \begin{align}
        \hat{M}=\tilde{M}/\sim=\{ [\tilde{x}] \}
    \end{align}
    式中,$[\tilde{x}]$表示柯西列$\tilde{x}$的等价类,即$[\tilde{x}]$是一个集合,
    集合中的所有元素在等价关系$\sim$下都是等价的。
    \item 在商空间$\hat{M}$中定义距离
    \begin{align}
        \hat{d}([\tilde{x}],[\tilde{y}])=\lim_{n\to\infty} d(x_n,y_n)
    \end{align}
    \item 则$(\hat{M},\hat{d})$即为距离空间$(M,d)$的完备化空间。
\end{enumerate}

嵌入映射:可以在原空间$M$与完备化空间$\hat{M}$之间定义一个单射$i$:
\begin{align}
    i: M \to \hat{M},\quad i(x)=[(x,x,x,\ldots)]
\end{align}
该映射将原空间中的每个点$x$映射为完备化空间中由常值序列$(x,x,x,\ldots)$所构成的等价类,
且\textbf{映射前后任意两元素的距离不变}





\subsubsection {赋范空间与 Banach 空间}
\tofill[定义]{赋范空间} 

完备的赋范空间称为Banach空间,或者B空间。


\subsubsection {等价范数}
\begin{definition}[范数的等价性]
    设$V$是一个线性空间,$\|\cdot\|_1$和$\|\cdot\|_2$是$V$上的两个范数,如果存在正常数$c$和$C$,使得对任意$v \in V$,都有
    \begin{align}
        c\|v\|_1 \leq \|v\|_2 \leq C\|v\|_1
    \end{align}
    则称这两个范数是等价的。

    若存在正数C,使得对任意$v \in V$,都有
    \begin{align}
        \|v\|_2 \leq C\|v\|_1
    \end{align}
    则称范数$\|\cdot\|_1$强于$\|\cdot\|_2$。
\end{definition}

性质:
\begin{itemize}
    \item 在有限维线性空间上,任意两个范数都是等价的
    \item 在无限维线性空间上,范数不一定是等价的
    \item 若一个点列在较强的范数下是Cauchy列,则在较弱的范数下也是Cauchy列;反之不必然。
\end{itemize}

\subsubsection{常用的范数}
\paragraph{$\mathbb{R}^n$上的范数} 记$x=(x_1,x_2,\ldots,x_n)^T \in \mathbb{R}^n$,则常用的范数有:
\begin{itemize}
    \item 无穷范数:所有元素的最大值
    \begin{align}
        \|x\|_{\infty}=\max_{1 \leq i \leq n} |x_i|
    \end{align}
    \item 1-范数:所有元素的绝对值之和
    \begin{align}
        \|x\|_{1}=\sum_{i=1}^{n} |x_i|
    \end{align}
    \item 2-范数:欧几里得范数,即所有元素的平方和的平方根
    \begin{align}
        \|x\|_{2}=\left(\sum_{i=1}^{n} |x_i|^2\right)^{\frac{1}{2}}
    \end{align}
\end{itemize}

\paragraph{$C[a,b]$(有界闭区间上连续函数空间)上的范数}
\begin{itemize}
    \item 无穷范数:函数在区间上的最大绝对值
    \begin{align}
        \|f\|_{\infty}=\max_{a \leq x \leq b} |f(x)|
    \end{align}
    \item 1-范数:函数在区间上的绝对值积分
    \begin{align}
        \|f\|_{1}=\int_{a}^{b} |f(x)| \, dx
    \end{align}
    \item 2-范数:函数在区间上的平方积分的平方根
    \begin{align}
        \|f\|_{2}=\left(\int_{a}^{b} |f(x)|^2 \, dx\right)^{\frac{1}{2}}
    \end{align}
\end{itemize}

$C[a,b]$上三个范数的性质:

\begin{itemize}
    \item 任意两个范数不等价
    \item 无穷范数强于2范数,2范数强于1范数
    \item 只有无穷范数对应的赋范空间是完备的
    \item 1-范数对应的完备化空间为$L^1(a,b)$,2-范数对应的完备化空间为$L^2(a,b)$ 
        \footnote{$L^1(a,b)$为(a,b)上的可积函数空间,$L^2(a,b)$为(a,b)上的平方可积函数空间。}
\end{itemize}

\subsection {内积空间}
\subsubsection {内积定义与性质}
\begin{definition}[内积]
    设$(S,P)$是一个线性空间,如果对任意的$u,v,w \in S$和$a,b \in P$,存在一个映射$S\times S\rightarrow P$,满足 
    \begin{enumerate}
        \item 共轭对称性:$\langle u,v \rangle = \overline{\langle v,u \rangle}$
        \item 线性性:$\langle au + bv,w \rangle = a\langle u,w \rangle + b\langle v,w \rangle$
        \item 正定性:$\langle v,v \rangle \geq 0$,且当且仅当$v=0$时,$\langle v,v \rangle = 0$
    \end{enumerate}
    则称该映射为内积,$(S,P)$构成一个内积空间。
\end{definition}

若$\langle x,y \rangle =0$,则称$x$与$y$正交。

几个常用空间上的内积:
\begin{itemize}
    \item $\mathbb{R}^n$或$\mathbb{C}^n$上的内积
    \begin{align}
        \langle x,y \rangle = \sum_{i=1}^{n} x_i \overline{y_i}
    \end{align}
    \item $C[a,b]$(有界闭区间上连续函数空间)上的内积
    \begin{align}
        \langle f,g \rangle = \int_{a}^{b} f(x) \overline{g(x)} \, dx
    \end{align}
    \item $C[a,b]$上的带权内积:
    \begin{align}
        (f,g)=\int_a^b \rho(x) f(x) \overline{g(x)} \, dx
    \end{align}
    其中,权函数$\rho(x)$需要满足条件:
    \begin{itemize}
        \item $\rho(x)\in C[a,b]$
        \item $\rho(x)$几乎处处为正
        \item $\int_a^b \rho(x) \, dx < +\infty$
        \item $\forall q(x) \in P_n$,$\int_a^b \rho(x) |q(x)|\mathrm{d}x < \infty$
    \end{itemize}
    带权内积所研究的空间称为加权内积空间:
    \begin{align}
        L^2_{\rho}(a,b) = \{ f(x) \,|\, \int_a^b \rho(x) |f(x)|^2 \, dx < +\infty \}
    \end{align}
\end{itemize}

常用的权函数有:
\begin{align}
    \rho(x) = 1, & \quad [a,b] = [-1,1]\\
    \rho(x)=\frac{1}{1-x^2}, & \quad [a,b] = [-1,1]\\
\end{align}


\subsubsection {正交性与 Schmidt 正交化}
\tofill[定义]{正交性}

\tofill[方法]{用Grammer矩阵判断内积空间中向量组的线性无关性}

\tofill[方法]{Schmidt 正交化过程:从一个线性无关向量组构造一个正交向量组:让每个向量减去与已有空间垂直的分量}

用Schmidt正交化过程得到的正交向量组具有以下性质:
\begin{align}
    \Phi_{k-1} \subset \Phi_k\\
    y_k \perp \Phi_{k-1}
\end{align}

\subsubsection {由内积诱导的范数}
\begin{definition}[诱导范数]
    设$(S,P)$是一个内积空间,则可以定义范数$\|\cdot\|$如下:
    \begin{align}
        \|v\|=\sqrt{\langle v,v \rangle}
    \end{align}
    则称该范数为由内积诱导的范数。
\end{definition}

\begin{itemize}
    \item 任何内积均能诱导对应的范数
    \item 当且仅当范数满足平行四边形法则时
    \begin{align}
        \|f+g\|^2 + \|f-g\|^2 = 2\|f\|^2 + 2\|g\|^2
    \end{align}
    范数可以诱导内积:
    \begin{align}
        (x,y)=\frac{1}{4}(\|x+y\|^2 - \|x-y\|^2 + i\|x+iy\|^2 - i\|x-iy\|^2)
    \end{align}
\end{itemize}


\subsection {正交多项式}
\begin{definition}
[正交多项式]
    设$\{\phi_n(x)\}$是定义在区间$[a,b]$上的一组多项式,且每个多项式的次数为$n$,如果对任意$m \neq n$,都有
    \begin{align}
        \int_a^b \rho(x) \phi_m(x) \phi_n(x) \, dx = 0
    \end{align}
    则称$\{\phi_n(x)\}$为区间$[a,b]$上关于权函数$\rho(x)$的正交多项式。
\end{definition}

正交多项式的性质:
\begin{itemize}%[]
    \item $\deg \phi_i=i$
    \item $(\phi_i,\,\phi_j)=0,\quad \forall i\neq j$
    \item $\phi_n$为实系数多项式
    \item $\phi_n$在开区间$(a,b)$内恰有n个实单根
\end{itemize}

\tofill[证明]{$\phi_n$在开区间$(a,b)$内恰有n个实单根的证明。证法:分别证明实根、单根、全在$(a,b)$内。3个命题均可用反证法。}


\subsubsection {$\rho=1$:Legendre 多项式}
产生方法:
\begin{itemize}
    \item 权函数$\rho(x)=1$
    \item 区间$[-1,1]$
\end{itemize}

表达式:
\begin{align}
    P_n(x)=\frac{1}{2^nn!}\frac{\mathrm{d}^n}{\mathrm{d}x^n}[(x^2-1)^n]
\end{align}

性质:
\begin{itemize}
    \item 首项系数:
    \begin{align}
        k_n=\frac{(2n)!}{2^n(n!)^2}
    \end{align}
    \item 正交归一化:
    \begin{align}
        \int_{-1}^{1} P_n(x)P_m(x) \, dx = \frac{2}{2n+1} \delta_{mn}
    \end{align}
    \item 三项递推关系:
    \begin{align}
        (n+1)P_{n+1}(x)=(2n+1)xP_n(x)-nP_{n-1}(x)
    \end{align}
    \item 奇偶性:
    \begin{align}
        P_n(-x)=(-1)^n P_n(x)
    \end{align}
    \item 导数关系:
    \begin{align}
        \frac{\mathrm{d}}{\mathrm{d}x} P_n(x) = \frac{n}{x^2-1} [xP_n(x) - P_{n-1}(x)]
    \end{align}
    \item 前五项:
    \begin{align}
        &P_0(x)=1\\
        &P_1(x)=x\\
        &P_2(x)=\frac{1}{2}(3x^2-1)\\
        &P_3(x)=\frac{1}{2}(5x^3-3x)\\
        &P_4(x)=\frac{1}{8}(35x^4-30x^2+3)
    \end{align}
    \item 零平方误差最小:
    \begin{theorem}
        在所有首项为1的n次多项式中,Legendre多项式$\tilde{P}_n(x)$在$[-1,1]$上与零的平方误差最小。
    \end{theorem}
\end{itemize}



\subsubsection {$\rho(x)=\frac{1}{\sqrt{1-x^2}}$:Chebyshev 多项式}
产生方法:
\begin{itemize}
    \item 权函数$\rho(x)=\frac{1}{\sqrt{1-x^2}}$
    \item 区间$[-1,1]$
\end{itemize}

表达式:
\begin{align}
    T_n(x)=\cos(n \arccos x)
\end{align}

性质:
\begin{itemize}
    \item 首项系数:$2^{n-1}$
    \item 正交归一化:
    \begin{align}
        \int_{-1}^{1} \frac{T_n(x)T_m(x)}{\sqrt{1-x^2}} \, dx = \begin{cases}
            \pi, & n=m=0\\
            \frac{\pi}{2}, & n=m\neq 0\\
            0, & n\neq m
        \end{cases}
    \end{align}
    \item 三项递推关系:
    \begin{align}
        T_{n+1}=2xT_n-T_{n-1}
    \end{align}
    \item 奇偶性:
    \begin{align}
        T_n(-x)=(-1)^n T_n(x)
    \end{align}
    \item 前五项:
    \begin{align}
        &T_0(x)=1\\
        &T_1(x)=x\\
        &T_2(x)=2x^2-1\\
        &T_3(x)=4x^3-3x\\
        &T_4(x)=8x^4-8x^2+1
    \end{align}
    \item 零点:
    \begin{align}
        x_k=\cos\left(\frac{2k-1}{2n}\pi\right),\quad k=1,2,\ldots,n
    \end{align}
    \item 极值点:
    \begin{align}
        x_k=\cos\left(\frac{k\pi}{n}\right),\quad k=0,1,\ldots,n
    \end{align}
    \item 简单表达式:当$|x|\geq 1$时,
    \begin{align}
        T_n(x)=\frac{1}{2}\left[ \left( x + \sqrt{x^2-1} \right)^n + \left( x - \sqrt{x^2-1} \right)^n \right]
    \end{align}
\end{itemize}




\subsection {矩阵空间}
\tofill[性质]{矩阵空间的基本性质:线性空间、乘法运算、代数性质}


\subsubsection {矩阵范数}
\begin{definition}[矩阵范数]
    矩阵空间$\mathbb{C}^{n\times n}$上的范数$\|\cdot\|$称为矩阵范数,如果对任意的$A,B \in \mathbb{C}^{n\times n}$和$a \in \mathbb{C}$,满足以下性质:
    \begin{enumerate}
        \item 非负性与分离性:$\|A\| \geq 0$,且当且仅当$A=0$时,$\|A\|=0$
        \item 齐次性:$\|aA\| = |a| \|A\|$
        \item 三角不等式:$\|A+B\| \leq \|A\| + \|B\|$
        \item 次乘性:$\|AB\| \leq \|A\| \cdot \|B\|$
    \end{enumerate}
\end{definition}

Note: 矩阵范数是定义在矩阵代数而非矩阵空间上的,必须与矩阵乘法相容。

\begin{definition}[矩阵范数与向量范数的相容性]
    设$\|\cdot\|_v$是向量空间$\mathbb{C}^n$上的一个范数,$\|\cdot\|_m$是矩阵空间$\mathbb{C}^{n\times n}$上的一个范数,如果对任意的$A \in \mathbb{C}^{n\times n}$和$x \in \mathbb{C}^n$,都有
    \begin{align}
        \|Ax\|_v \leq \|A\|_m \cdot \|x\|_v
    \end{align}
    则称矩阵范数$\|\cdot\|_m$与向量范数$\|\cdot\|_v$是相容的。
\end{definition}

矩阵范数的两种常见构造方法:
\begin{itemize}
    \item 直接构造:Frobenius范数
    \begin{align}
        \|A\|_F=\left( \sum_{i=1}^{n} \sum_{j=1}^{n} |a_{ij}|^2 \right)^{\frac{1}{2}}
    \end{align}
    Frobenius范数的性质:
    \begin{itemize}
        \item Frobenius范数与向量2-范数相容
        \item $\|I\|_F=\sqrt{n}$
    \end{itemize}
    \item 向量范数诱导:算子范数
    \begin{definition}
    [算子范数]
        设$\|\cdot\|_v$是向量空间$\mathbb{C}^n$上的一个范数,则可以定义矩阵空间$\mathbb{C}^{n\times n}$上的算子范数$\|\cdot\|_m$如下:
        \begin{align}
            \|A\|_m = \max_{x \neq 0} \frac{\|Ax\|_v}{\|x\|_v} = \max_{\|x\|_v=1} \|Ax\|_v
        \end{align}
        则称该范数为由向量范数$\|\cdot\|_v$诱导的算子范数。
    \end{definition}
\end{itemize}

常用的几个算子范数:
\begin{itemize}
    \item 无穷范数:行和最大值
    \begin{align}
        \|A\|_{\infty} = \max_{1 \leq i \leq n} \sum_{j=1}^{n} |a_{ij}|
    \end{align}
    \item 1-范数:列和最大值
    \begin{align}
        \|A\|_{1} = \max_{1 \leq j \leq n} \sum_{i=1}^{n} |a_{ij}|
    \end{align}
    \item 2-范数(谱范数):$A$的最大奇异值,即$A^HA$的最大特征值的平方根
    \begin{align}
        \|A\|_{2} = \sqrt{\lambda_{\max}(A^HA)}
    \end{align}
\end{itemize}


\subsubsection {谱半径}
\begin{definition}[谱半径]
    谱半径定义为矩阵所有特征值模的最大值,即
    \begin{align}
        \rho(A) = \max_{1 \leq i \leq n} |\lambda_i|
    \end{align}
\end{definition}

谱半径和矩阵范数的关系:
\begin{itemize}
    \item 矩阵范数下界:
    \begin{theorem}
        对任意$A\in\mathbb{C}^{n\times n}$,有
        \begin{align}
            \rho(A) \leq \|A\|
        \end{align}
    \end{theorem}
    \item 无穷接近范数的存在性:
    \begin{theorem}
        对任意$A\in\mathbb{C}^{n\times n}$,存在一个矩阵范数$\|\cdot\|$,使得
        \begin{align}
            \rho(A) \leq \|A\| \leq \rho(A) + \varepsilon
        \end{align}
        其中,$\varepsilon$为任意给定的正常数。
    \end{theorem}
\end{itemize}


\subsubsection {可逆矩阵相关定理}
\begin{theorem}[扰动引理I]
    给定$B\in\mathbb{C}^{n\times n}$。设$\|B\|<1$,则$I+B$可逆,且
    \begin{align}
        \|(I+B)^{-1}\| \leq \frac{1}{1-\|B\|}
    \end{align}
\end{theorem}

\begin{theorem}
    [扰动引理II]
    设$A,\,C\in\mathbb{C}^{n\times n}$,且$A$可逆。若 
    \begin{align}
        \|C-A\| < \frac{1}{\|A^{-1}\|}
    \end{align}
    则$C$也可逆,且
    \begin{align}
        \|C^{-1}\| \leq \frac{\|A^{-1}\|}{1 - \|A^{-1}\| \cdot \|C-A\|}
    \end{align}
\end{theorem}


\begin{theorem}
    [扰动定理II]
    设$A,\,\delta A \in\mathbb{C}^{n\times n}$,且$A$可逆。若$\|A^{-1}\delta A\| < 1$,则$A + \delta A$也可逆,且
    \begin{align}
        \|(A + \delta A)^{-1}\| \leq \frac{\|A^{-1}\|}{1 - \|A^{-1}\| \cdot \|\delta A\|}
    \end{align}
\end{theorem}


\chapter {函数插值与重构}

\section*{总结:基本方法是利用插值基函数构造插值多项式,从而实现对函数的近似与重构。}

问题1:求解插值函数
\begin{itemize}
    \item 整个区间上的连续插值:Lagrange插值、Newton插值、Hermite插值
    \begin{itemize}
        \item Lagrange插值基函数:
        \begin{align}
            L_\alpha(x)=\prod_{\substack{\beta\in I\\ \beta\neq \alpha}} \frac{x - x_\beta}{x_\alpha - x_\beta}, \quad \alpha\in I
        \end{align}
        \item  Newton插值和Hermite插值:构造均差表,列表计算。
        \begin{itemize}
            \item 均差递推关系:
            \begin{align}
                f[x_i] &= f(x_i)\\
                f[x_i,x_{i+1},\ldots,x_{i+k}] &= \frac{f[x_{i+1},\ldots,x_{i+k}] - f[x_i,\ldots,x_{i+k-1}]}{x_{i+k} - x_i}
            \end{align}
            \item 均差构造插值多项式:各项为$f[x_0,x_1,\ldots, x_k]\cdot (x-x_0)(x-x_1)\ldots(x-x_{k-1})$
        \end{itemize}
    \end{itemize}
    \item 分段插值:分片线性插值、分片三次Hermite插值 
    \begin{itemize}
        \item 分片线性插值基函数:
        \begin{align}
            L_{\alpha}(x) =& \begin{cases}
                \frac{x - x_{\alpha-1}}{x_\alpha - x_{\alpha-1}}, & x \in [x_{\alpha-1}, x_\alpha]\\
                \frac{x_{\alpha+1} - x}{x_{\alpha+1} - x_\alpha}, & x \in [x_\alpha, x_{\alpha+1}]\\
                0, & \text{else}
            \end{cases}\\
            \phi(x)=&\sum_{\alpha=0}^{n} f_\alpha L_\alpha(x)
        \end{align}
        \item 分片三次Hermite插值基函数:
        \begin{align}
            \alpha_k =& \left(  1+2\frac{x-x_k}{x_{k+1}-x_k}  \right) \left(  \frac{x_{k+1}-x}{x_{k+1}-x_k}  \right)^2\\
            \beta_k =& (x-x_k) \left(  \frac{x_{k+1}-x}{x_{k+1}-x_k}  \right)^2\\
            \alpha_{k+1} =& \left(  1+2\frac{x_{k+1}-x}{x_{k+1}-x_k}  \right) \left(  \frac{x-x_k}{x_{k+1}-x_k}  \right)^2\\
            \beta_{k+1} =& -(x_{k+1} - x) \left(  \frac{x - x_k}{x_{k+1}-x_k}  \right)^2\\
            \phi(x) =& \sum_{k=0}^{n} \left[ f_k \alpha_k + f_k' \beta_k  \right],\quad x\in[x_k,x_{k+1}]
        \end{align}
    \end{itemize}
\end{itemize}


\section {通用理论}
\subsection {问题模型}
\tofill[数学描述]{采样泛函视角下的插值问题数学描述}
\subsection {插值空间}
常用的插值空间:多项式函数空间、样条函数空间、三角多项式函数空间
\subsection {误差分析与收敛性}

\section {具体插值方法}
\subsection {一维多项式插值}
问题:给定插值数据(采样数据)$(x_\alpha, f_\alpha)$,$\alpha\in I$,确定多项式$P(x)\in P_n$,
$n=|I|-1$,满足插值条件
\begin{align}
    x_\alpha(P)=P(x_\alpha)=f_\alpha, \quad \alpha\in I
\end{align}

\begin{theorem}
    [多项式插值基本定理]

    给定$n+1$个插值条件
    \begin{align}
        (x_\alpha,\,f_\alpha),\quad \alpha\in I, \quad x_\alpha\neq x_\beta \text{ for }\alpha\neq\beta
    \end{align}
    则存在唯一的插值多项式$P\in P_n$满足插值条件。
\end{theorem}

Note:若$x_\alpha$取之于复平面,上述定理依然成立;且上述定理与采样节点的排序无关。 

\subsubsection {Lagrange 插值}
\paragraph {基函数构造}
\begin{definition}
    [Lagrange插值基函数]
    定义
    \begin{align}
        L_\alpha(x)=L_{\alpha;I}(x)=\prod_{\substack{\beta\in I\\ \beta\neq \alpha}} \frac{x - x_\beta}{x_\alpha - x_\beta}, \quad \alpha\in I
    \end{align}
    称为插值基函数。
\end{definition}

若给定3个插值条件$(x_0,f_0)$,$(x_1,f_1)$,$(x_2,f_2)$,则对应的插值基函数为
\begin{align}
    L_0(x) &= \frac{(x - x_1)(x - x_2)}{(x_0 - x_1)(x_0 - x_2)}\\
    L_1(x) &= \frac{(x - x_0)(x - x_2)}{(x_1 - x_0)(x_1 - x_2)}\\
    L_2(x) &= \frac{(x - x_0)(x - x_1)}{(x_2 - x_0)(x_2 - x_1)}
\end{align}

插值基函数天然满足性质:
\begin{align}
    x_\beta(L_\alpha)=L_\alpha(x_\beta)=\delta_{\alpha\beta}
\end{align}

\paragraph {插值公式}
在计算出插值基函数的基础上,插值多项式可写为:
\begin{align}
    P(x)=\sum_{\alpha\in I} f_\alpha L_\alpha(x)
\end{align}

\paragraph {余项}





\paragraph {均差定义}
\begin{definition}
    [均差的递推公式]
    设$f(x)$在区间$[a,b]$上有定义,且给定插值节点$x_0,x_1,\ldots,x_n$,则定义如下均差:
    \begin{align}
        f[x_i] &= f(x_i)\\
        f[x_i,x_{i+1},\ldots,x_{i+k}] &= \frac{f[x_{i+1},\ldots,x_{i+k}] - f[x_i,\ldots,x_{i+k-1}]}{x_{i+k} - x_i}
    \end{align}
    其中,$i=0,1,\ldots,n-k$,$k=1,2,\ldots,n$。
\end{definition}

均差的性质:
\begin{itemize}
    \item $f_{i_0i_1...i_k}$与节点$x_{i_0},x_{i_1},\ldots,x_{i_k}$的顺序无关
    \item 设f是N次多项式,若$k>N$,则对任意节点$x_{i_0},x_{i_1},\ldots,x_{i_k}$,都有
    \begin{align}
        f_{i_0i_1\ldots i_k} = 0
    \end{align}
\end{itemize}


\paragraph {插值公式}
\begin{align}
    P_{i_0i_1...i_k}(x) &= f_{i_0} + f_{i_0i_1}(x - x_{i_0}) + f_{i_0i_1i_2}(x - x_{i_0})(x - x_{i_1}) + \ldots \nonumber\\
    &+ f_{i_0i_1\ldots i_k}(x - x_{i_0})(x - x_{i_1})\cdots(x - x_{i_{k-1}})
\end{align}






\paragraph{Newton插值多项式的列表计算}

以给定4个节点时$x_0,\,x_1,\,x_2,\,x_3$的插值问题为例。可以按照如下表格从左向右逐列填写计算均差:

\begin{center}
\renewcommand{\arraystretch}{1.5}
\begin{tabular}{|c|cccc|}
\hline
$x_i$ & 0阶均差 & 1阶均差 & 2阶均差 & 3阶均差 \\
\hline
$x_0$ & \fcolorbox{red}{white}{$f(x_0)$} & & & \\
& & \fcolorbox{red}{white}{$f[x_0,x_1]=\dfrac{f(x_1)-f(x_0)}{x_1 - x_0}$} & & \\
$x_1$ & $f(x_1)$ & & \fcolorbox{red}{white}{$f[x_0,x_1,x_2]=\dfrac{f[x_1,x_2]-f[x_0,x_1]}{x_2 - x_0}$} & \\
& & $f[x_1,x_2]=\dfrac{f(x_2)-f(x_1)}{x_2 - x_1}$ & & \fcolorbox{red}{white}{$f[x_0,x_1,x_2,x_3]=...$} \\
$x_2$ & $f(x_2)$ & & $f[x_1,x_2,x_3]=\dfrac{f[x_2,x_3]-f[x_1,x_2]}{x_3 - x_1}$ & \\
& & $f[x_2,x_3]=\dfrac{f(x_3)-f(x_2)}{x_3 - x_2}$ & & \\
$x_3$ & $f(x_3)$ & & & \\
\hline
\end{tabular}
\end{center}
之后用插值表最上方一行的均差值逐个组装Newton插值多项式。


\subsubsection {Hermite 插值}
\paragraph{Hermite插值问题}
给定$\xi_i$, $f_i^{(k)}$, $i=0,1,...,m$,  $k=0,1,...,n_i-1$, 其中$\xi_i$两两不同,且 
\begin{align}
    \xi_0<\xi_1<...<\xi_m 
\end{align}
希望确定一个次数为n的多项式函数
\begin{align}
    P_n(x),\quad n=\sum_{i=0}^{m} n_i - 1
\end{align}
满足插值条件
\begin{align}
    P^{(k)}(\xi_i)=f_i^{(k)}, \quad i=0,1,...,m \quad k=0,1,...,n_i-1
\end{align}


\paragraph{拓展均差}
\begin{definition}
    [拓展均差]
    设$f\in C^n(I(x_0,\,x_1,\,...,\,x_n))$,定义
    \begin{align*}
        f[x_0,\,x_1,\,x_n]=\int_0^{t_0} \mathrm{d}t_1 &\int_0^{t_1}\mathrm{d}t_2 \cdots \int_0^{t_{n-1}} \mathrm{d}t_n \\\\
        &f^{(n)}(t_n[x_n-x_{n-1}]+t_{n-1}[x_{n-1}-x_{n-2}]+...+t_1[x_1-x_0]+t_0x_0)
    \end{align*}
    式中,$n\geq 1$且$t_0=1$。
\end{definition}
Note: 这一积分实际上表示了一个单位标准n-维标准型上的积分,或者说积分区域始终是一个插值节点构造的凸组合。
这隐含了一个要求是$1=t_0\geq t_1 \geq t_2 \geq ... \geq t_n \geq 0$。

拓展均差的性质:
\begin{itemize}
    \item 若$x_i$两两不一,则拓展均差等价于普通均差
    \begin{align}
        f[x_0,\,x_1,\,x_n]=f_{x_0,x_1,...,x_n}
    \end{align}
    且具有相同的递推关系:
    \begin{align}
        f[x_0,\,x_1,\,...,\,x_n]=\frac{f[x_0,\,x_1,\,...,\,x_{n-2},\,x_n]-f[x_0,\,x_1,\,...,\,x_{n-1}]}{x_n - x_0}
    \end{align}
    \item 极限性质:若$f$足够光滑,则
    \begin{align}
        \lim_{\epsilon_i\rightarrow 0}f[x_0+\epsilon_0,\,x_1+\epsilon_1,\,...,\,x_n+\epsilon_n]=f[x_0,\,x_1,\,...,\,x_n]
    \end{align}
    \item 导数与重节点:可从极限性质导出
    \begin{align}
        \frac{\mathrm{d}}{\mathrm{d}x}f[x_0,x_1,...,x_n,x]=f[x_0,x_1,...,x_n,x,x]
    \end{align}
    \item 介值定理:若$f\in C^n[a,b]$,$x_0,\,x_1,\,...,\,x_n\in [a,b]$,则存在$\xi\in I(x_0,\,x_1,\,...,\,x_n)$,使得
    \begin{align}
        f[x_0,\,x_1,\,...,\,x_n]=\frac{f^{(n)}(\xi)}{n!}
    \end{align}
    特别地,
    \begin{align}
         f[\underbrace{x, x, \dots, x}_{n+1\text{个}}]=\frac{f^{(n)}(x)}{n!}
    \end{align}
\end{itemize}

\paragraph{Hermite插值多项式}
Hermite插值多项式可表示为
\begin{align}
    P(x)=f[x_0]+f[x_0,x_1](x-x_0)+...+f[x_0,x_1,...,x_n](x-x_0)(x-x_1)...(x-x_{n-1})
\end{align}
其中,$x_0,\,...,\,x_n$为下面序列的任意置换:
\begin{align}
    \underbrace{\xi_0,\,\xi_0,\,...,\,\xi_0}_{n_0\text{个}},\,\underbrace{\xi_1,\,\xi_1,\,...,\,\xi_1}_{n_1\text{个}},\, ...,\,\underbrace{\xi_m,\,\xi_m,\,...,\,\xi_m}_{n_m\text{个}}
\end{align}

\paragraph{列表法求Hermite插值多项式}
假设给定2个节点$\xi_0,\,\xi_1$,对应的插值条件分别为$f_0,\,f_0',\,f_0''$,$f_1$,则可按下表计算均差:


Hermite 插值均差表(节点序列:$\xi_0, \xi_0, \xi_0, \xi_1$):
\[
\begin{array}{c|c|c|c|c}
\text{节点} & \text{0 阶均差} & \text{1 阶均差} & \text{2 阶均差} & \text{3 阶均差} \\
\hline
\xi_0 & f[\xi_0] = f_0 & & & \\
\xi_0 & f[\xi_0] = f_0 & f[\xi_0,\xi_0] = f_0' & & \\
\xi_0 & f[\xi_0] = f_0 & f[\xi_0,\xi_0] = f_0' & f[\xi_0,\xi_0,\xi_0] = \dfrac{f_0''}{2} & \\
\xi_1 & f[\xi_1] = f_1 & f[\xi_0,\xi_1] = \dfrac{f_1 - f_0}{\xi_1 - \xi_0} & 
f[\xi_0,\xi_0,\xi_1] = \dfrac{f[\xi_0,\xi_1] - f[\xi_0,\xi_0]}{\xi_1 - \xi_0} &
f[\xi_0,\xi_0,\xi_0,\xi_1] = \dfrac{f[\xi_0,\xi_0,\xi_1] - f[\xi_0,\xi_0,\xi_0]}{\xi_1 - \xi_0}
\end{array}
\]

对应的插值多项式为:
\[
\begin{aligned}
P(x) &= f[\xi_0] + f[\xi_0,\xi_0](x - \xi_0) + f[\xi_0,\xi_0,\xi_0](x - \xi_0)^2 + f[\xi_0,\xi_0,\xi_0,\xi_1](x - \xi_0)^3 \\
&= f_0 + f_0'(x - \xi_0) + \frac{f_0''}{2}(x - \xi_0)^2 \\
&\quad + \frac{2(f_1 - f_0 - f_0'(\xi_1 - \xi_0)) - f_0''(\xi_1 - \xi_0)^2}{2(\xi_1 - \xi_0)^3}(x - \xi_0)^3
\end{aligned} 
\]


\subsubsection{Lagrange插值和Hermite插值的收敛分析}
\paragraph{插值余项}
若f在区间$[a,b]$上具有$n+1$阶连续导数,则对任意$x\in[a,b]$,存在$\xi\in I(x, x_0, x_1, \ldots, x_n)$,使得
\begin{align}
    R(x) = f(x) - P(x) = \frac{f^{(n+1)}(\xi)}{(n+1)!} \prod_{i=0}^{n} (x - x_i)= \frac{f^{(n+1)}(\xi)}{(n+1)!} \omega_{01...n}(x)
\end{align}

\paragraph{收敛性}
收敛性的定义:当给定插值点的最大间距$h\rightarrow 0$时,插值余项$R(x)\rightarrow 0$,则称插值多项式序列在区间$[a,b]$上收敛于函数$f(x)$。


\begin{definition}
    [多项式插值收敛定义]
    设$f\in C^{\infty}[a,b]$,且存在正常数$M>0$,使得对任意的$n\geq 0$,都有
    \begin{align}
        \max_{x\in[a,b]} |f^{(n)}(x)| \leq M
    \end{align}
    则对任意在$[a,b]$上的插值节点序列$\{x_i^{(n)}\}_{i=0}^{n}$,对应的插值多项式序列$\{P_n(x)\}$在$[a,b]$上均匀收敛于$f(x)$。
\end{definition}


收敛的充分条件:
\begin{theorem}
    记$\delta =|I(x_0,\,x_1,\,x_2, \ldots, x_n)|$, $\tilde{x}$为I的中心。若$f$在$B(\tilde{x},w\delta)$上复解析,则对任意$\bar{x}\in I$,插值法收敛。
\end{theorem}

\subsection {分段插值}
\subsubsection {分段线性插值}
\tofill{分片线性插值问题描述}

\paragraph{分片线性插值的插值基函数}
\begin{definition}
    [分片线性插值基函数]
    设给定插值节点$x_0<x_1<...<x_n$,则定义分片线性插值基函数为
    \begin{align}
        l_i(x) = \begin{cases}
            \frac{x - x_{i-1}}{x_i - x_{i-1}}, & x\in [x_{i-1}, x_i]\\
            \frac{x_{i+1} - x}{x_{i+1} - x_i}, & x\in [x_i, x_{i+1}]\\
            0, & \text{otherwise}
        \end{cases}
    \end{align}
    其中,$i=0,1,...,n$,且约定$x_{-1}=x_0$,$x_{n+1}=x_n$。
\end{definition}
则线性插值基函数为 
\begin{align}
    \phi(x)=\sum_{k=1}^nf_k l_k(x)
\end{align}

\paragraph{分片线性插值的收敛性定理}
定义
\begin{align}
    h=\max_{1\leq i \leq n}(x_i - x_{i-1})
\end{align}

\begin{itemize}
    \item 若$f\in C[a,b]$,则$\lim_{h\rightarrow 0}\|f-\phi\|_\infty\rightarrow 0$
    \item 若$f\in C^1[a,b]$,则$\|f-\phi\|_\infty \leq \frac{h}{2}\|f'\|_\infty$
    \item 若$f\in C^2[a,b]$,则$\|f-\phi\|_\infty \leq \frac{h^2}{8}\|f''\|_\infty$
\end{itemize}

\subsubsection {分段三次 Hermite 插值}
\tofill[]{分片三次Hermite插值的数学描述}


\paragraph{插值基函数}
分段三次Hermite插值的基函数满足如下条件:
\begin{align}
    \alpha_k(x_i)=& \delta_{ik},\quad \alpha_k'(x_i)=0\\
    \beta_k(x_i)=& 0,\quad \beta_k'(x_i)= \delta_{ik}
\end{align}

在单元$[x_k, x_{k+1}]$上,三次Hermite插值基函数为
\begin{align}
    \alpha_k =& \left(  1+2\frac{x-x_k}{x_{k+1}-x_k}  \right) \left(  \frac{x_{k+1}-x}{x_{k+1}-x_k}  \right)^2\\
    \beta_k =& (x-x_k) \left(  \frac{x_{k+1}-x}{x_{k+1}-x_k}  \right)^2\\
    \alpha_{k+1} =& \left(  1+2\frac{x_{k+1}-x}{x_{k+1}-x_k}  \right) \left(  \frac{x-x_k}{x_{k+1}-x_k}  \right)^2\\
    \beta_{k+1} =& -(x_{k+1} - x) \left(  \frac{x - x_k}{x_{k+1}-x_k}  \right)^2
\end{align}

\begin{itemize}
    \item $\alpha_k$满足单位分解性:$\alpha_k+\alpha_{k+1}=1$
    \item $\alpha_k(x_i)=\delta_{ik}$,$\alpha_k'(x_i)=0$
    \item $\beta_k(x_i)=0$,$\beta_k'(x_i)=\delta_{ik}$
\end{itemize}



\paragraph{三次Hermite插值多项式}
\begin{align}
    \phi=\sum_{k=0}^{n} \left[ f_k \alpha_k(x) + f_k' \beta_k(x) \right]
\end{align}

\paragraph{收敛性定理}
\begin{definition}
    [分段三次Hermite插值收敛性定理]
    设$f\in C^1[a,b]$,则分段三次Hermite插值多项式$\phi$满足
    \begin{align}
        \|f - \phi\|_\infty \leq ch\|f'\|_\infty
    \end{align}
\end{definition}

若f有更好的光滑性,则:
\begin{itemize}
    \item 若$f\in C^2[a,b]$,则$\|f-\phi\|_\infty \leq ch^2\|f''\|_\infty$
    \item 若$f\in C^3[a,b]$,则$\|f-\phi\|_\infty \leq ch^3\|f'''\|_\infty$
    \item 若$f\in C^4[a,b]$,则$\|f-\phi\|_\infty \leq \frac{1}{384}h^4\|f^{(4)}\|_\infty$
\end{itemize}

此外,对于不高于三次的多项式,分段三次Hermite插值是精确的。



\subsection {Fourier 插值}
\subsubsection{离散傅里叶变换}
\tofill[定义]{离散傅里叶变换式、变换式系数表达式}

如果f的光滑性满足$f\in C_{per}^M$,则有
\begin{align}
    a_n=O(n^{-M})\\
    b_n=O(n^{-M})
\end{align}
且
\begin{align}
    \|f(x)-\left[  \frac{a_0}{2}+\sum_{n=1}^N a_n\cos nx+\sum_{n=1}^N b_n \sin nx \right]\|_\infty = O(N^{-M})
\end{align}

\subsubsection{三角多项式插值空间}
三角多项式插值空间为:
\begin{align}
    \Phi_{2M+1}:=&\left\{  \frac{A_0}{2} + \sum_{n=1}^M \left( A_n \cos nx + B_n \sin nx \right) \right\}\\
    \Phi_{2M}:=&\left\{  \frac{A_0}{2} + \sum_{n=1}^{M-1} \left( A_n \cos nx + B_n \sin nx \right) + A_M \cos Mx \right\}
\end{align}



\subsubsection {三角多项式插值与一般多项式插值}
\tofill[问题]{三角多项式插值问题的数学描述}

\tofill[问题]{辅助插值问题:找相多项式}

\tofill[理论]{两个插值问题之间的联系:欧拉公式}

\subsubsection{插值定理与三角插值多项式}
\begin{theorem}
    [三角多项式插值定理]
    设给定插值节点
    \begin{align}
        x_k = \frac{2k\pi}{N}, \quad k=0,1,\ldots,N-1
    \end{align}
    则对任意插值数据$f_k$,存在唯一的三角多项式
    \begin{align}
        P(x) = \frac{A_0}{2} + \sum_{n=1}^{M} \left( A_n \cos nx + B_n \sin nx \right)
    \end{align}
    (当N为奇数时,$M=\frac{N-1}{2}$;当N为偶数时,$M=\frac{N}{2}$)满足插值条件
    \begin{align}
        P(x_k) = f_k, \quad k=0,1,\ldots,N-1
    \end{align}
    同样,存在唯一的相多项式
    \begin{align}
        Q(x) = \sum_{n=0}^{N-1} \beta_n e^{inx}
    \end{align}
    满足插值条件
    \begin{align}
        Q(x_k) = f_k, \quad k=0,1,\ldots,N-1
    \end{align}
\end{theorem}


\begin{itemize}
    \item 三角插值多项式:
    \begin{align}
        A_j=\frac{2}{N}\sum_{k=0}^{N-1} f_k \cos\left(  \frac{2\pi jk}{N}  \right), \quad j=0,1,\ldots,M\\
        B_j=\frac{2}{N}\sum_{k=0}^{N-1} f_k \sin\left(  \frac{2\pi jk}{N}  \right), \quad j=1,2,\ldots,M
    \end{align}
    \item 相插值多项式:
    \begin{align}
        \beta_j=\frac{1}{N}\sum_{k=0}^{N-1} f_k w^{-kj}, \quad j=0,1,\ldots,N-1
    \end{align}
    式中,$w=e^{i\frac{2\pi}{N}}$。
\end{itemize}


% \subsubsection {离散 Fourier 变换关联}



\chapter {函数逼近}

\section*{总结}
Note:这里为了让总结看起来更顺畅,没有遵循原PPT和书中的符号规范,转而应用了比较统一的符号。这些符号规范仅在总结一部分使用。
\begin{itemize}
    \item 最佳平方逼近求解:
    \begin{itemize}
        \item 给定一组规范正交函数基时:
        \begin{itemize}
            \item 写出规范正交函数基$\{\phi_m\}_{m=0}^n$
            \item 计算系数$a_m=\dfrac{(f,\phi_m)}{(\phi_m,\phi_m)}$
            \item 写出逼近多项式$\phi^*=\sum_{m=0}^n a_m \phi_m$
        \end{itemize}
        \item 给定一组非规范正交函数基时:
        \begin{itemize}
            \item 写出非规范正交函数基$\{\phi_m\}_{m=0}^n$
            \item 构建法方程组:
            \begin{align}
                m_{ij}=(\phi_j,\phi_i),\quad b_i=(f,\phi_i)\\
            \end{align}
            \item 求解线性方程组$\mathbf{M}\mathbf{a}=\mathbf{b}$,得到系数$a_m$
            \item 写出逼近多项式$\phi^*=\sum_{m=0}^n a_m \phi_m$
        \end{itemize}
    \end{itemize}
    \item Legendre多项式作最佳平方逼近的收敛速度:高阶时控制在$1/\sqrt{n}$以下
    \item 最小二乘逼近求解:计算过程可视作最佳平方逼近的离散形式
    \begin{itemize}
        \item 给定基底$\{\phi_m\}_{m=0}^n$和采样点$\{x_k\}_{k=0}^N$
        \item 估计一个误差系数$\rho(x_i)$
        \item 用半内积构建法方程组:
        \begin{align}
            m_{ij}=\sum_{k=0}^N \rho(x_k) \phi_j(x_k)\phi_i(x_k),\quad b_i=\sum_{k=0}^N \rho(x_k) f(x_k)\phi_i(x_k)\\
        \end{align}
        \item 求解线性方程组$\mathbf{M}\mathbf{a}=\mathbf{b}$,得到系数$a_m$
        \item 写出逼近多项式$\phi^*=\sum_{m=0}^n a_m \phi_m$
    \end{itemize}
    \item 一致逼近求解:
    \begin{itemize}
        \item 给定原函数$f\in C[a,b]$和基底$\{\phi_m\}_{m=0}^n$
        \item 写出待定逼近多项式$\phi(x)=\sum_{m=0}^n a_m \phi_m(x)$
        \item 写出误差函数$E(x)=f(x)-\phi(x)$
        \item 根据切比雪夫交错点组定理,写出求解条件:
        \begin{itemize}
            \item 交错条件:$E(x_i)=-e(x_{x-1})$
            \item 偏差点条件:除端点作为偏差点的情况,必有$f'(x_i)=p_n'(x_i)$
        \end{itemize}
        \item 由偏差点条件可以解出一系列偏差点$x_i(\boldsymbol{a})$,再代入交错条件中,解出系数$a_m$
    \end{itemize}
    
\end{itemize}
\section {通用理论}

\subsection {问题模型}
\tofill[问题]{函数逼近问题的数学模型}

最佳逼近问题:找$\phi^*\in \Phi$,使得 
\begin{align}
    \|f-\phi^*\| = \min_{\phi\in \Phi} \|f-\phi\|
\end{align}


\subsection {逼近准则}
\subsubsection {最小二乘准则(L₂范数)}
\subsubsection {一致逼近准则(L∞范数)}

\subsection {核心定理}

\section {具体逼近方法}

\subsection {最优平方逼近}
\subsubsection{基础理论}

问题:给定一个线性子空间$\Phi$,找$\phi^*\in \Phi$,使得
\begin{align}
    \|f-\phi^*\|_2 = \min_{\phi\in \Phi} \|f-\phi\|_2
\end{align}

问题等价于
\begin{align}
    \|f-\phi^*\|_2^2 = \min_{\phi\in \Phi} \|f-\phi\|_2^2
\end{align}

于是构造出一个辅助函数:
\begin{align}
    I=\|f-\phi\|_2^2=\left( f-\sum_{i=0}^n a_i\phi_i, f-\sum_{i=0}^n a_i\phi_i \right)\\
    =  \sum_{i=0}^n \sum_{j=0}^n a_i a_j (\phi_i, \phi_j) - 2\sum_{i=0}^n a_i (f,\phi_i) +(f,f)
\end{align}

式中,$\{\phi_i\}$为$\Phi$的一组基。

\paragraph{法方程}

多元函数取得最小值的条件:对各变量偏导为0、在这一点的二阶偏导数构成的矩阵正定。

各变量偏导为0即导出法方程:
\begin{align}
    \sum_{j=0}^{n} a_j ( \phi_j, \phi_i ) = ( f, \phi_i ), \quad i=0,1,\ldots,n
\end{align}
或写成矩阵形式,
\begin{align}
    \begin{bmatrix}
        ( \phi_0, \phi_0 ) & ( \phi_1, \phi_0 ) & \cdots & ( \phi_n, \phi_0 ) \\
        ( \phi_0, \phi_1 ) & ( \phi_1, \phi_1 ) & \cdots & ( \phi_n, \phi_1 ) \\
        \vdots & \vdots & \ddots & \vdots \\
        ( \phi_0, \phi_n ) & ( \phi_1, \phi_n ) & \cdots & ( \phi_n, \phi_n )
    \end{bmatrix}
    \begin{bmatrix}
        a_0 \\ a_1 \\ \vdots \\ a_n
    \end{bmatrix}
    =
    \begin{bmatrix}
        ( f, \phi_0 ) \\ ( f, \phi_1 ) \\ \vdots \\ ( f, \phi_n )
    \end{bmatrix}
\end{align}


\begin{theorem}
    [最小二乘逼近法方程]
    设$\Phi$为线性子空间,$\{\phi_0, \phi_1, \ldots, \phi_n\}$为$\Phi$的一组基,则函数$f$在$\Phi$中的最优平方逼近$\phi^*$可表示为
    \begin{align}
        \phi^* = \sum_{j=0}^{n} a_j^* \phi_j
    \end{align}
    其中,系数$c_j^*$满足法方程组
    \begin{align}
        \sum_{j=0}^{n} a_j ( \phi_j, \phi_i ) = ( f, \phi_i ), \quad i=0,1,\ldots,n
    \end{align}
\end{theorem}

\paragraph{最佳平方逼近的性质}
\begin{lemma}
    [最佳平方逼近的正交性质]
    设$\phi^*$为函数$f$在子空间$\Phi$中的最佳平方逼近,则对任意$\phi\in \Phi$,都有
    \begin{align}
        f-\phi^* \perp \Phi
    \end{align}
\end{lemma}

\begin{corollary}[最佳平方逼近的勾股定理]
    设$\phi^*$为函数$f$在子空间$\Phi$中的最佳平方逼近,则有
    \begin{align}
        \|f\|_2^2 = \|f-\phi^*\|_2^2 + \|\phi^*\|_2^2
    \end{align}
\end{corollary}


\subsubsection{幂函数基的逼近}
考虑连续函数在n-次多项式空间$P_n$中的最佳平方逼近问题。

若选取$\{x^m\}_{m=0}^n$作为基函数,在$[0,1]$区间上求解最佳平方逼近,则法方程组的系数矩阵为Hilbert矩阵:
\begin{definition}
    [Hilbert矩阵]
    阶数为$n+1$的Hilbert矩阵定义为
    \begin{align}
        H_{ij} = \frac{1}{i+j-1}, \quad i,j=1,2,\ldots,n+1
    \end{align}
    即
    \begin{align}
        H_n=\begin{bmatrix}
        1 & \frac{1}{2} & \frac{1}{3} & \cdots & \frac{1}{n+1} \\
        \frac{1}{2} & \frac{1}{3} & \frac{1}{4} & \cdots & \frac{1}{n+2} \\
        \frac{1}{3} & \frac{1}{4} & \frac{1}{5} & \cdots & \frac{1}{n+3} \\
        \vdots & \vdots & \vdots & \ddots & \vdots \\
        \frac{1}{n+1} & \frac{1}{n+2} & \frac{1}{n+3} & \cdots & \frac{1}{2n+1}
        \end{bmatrix}
    \end{align}
\end{definition}

这一逼近形式的问题:
\begin{itemize}
    \item Hilbert矩阵病态,随着n的增大,条件数迅速增大,导致数值解不稳定
    \item 幂函数基不正交,导致法方程系数矩阵接近奇异
\end{itemize}


\subsubsection {正交多项式逼近}

\paragraph{广义傅里叶展开}
在$C[a,b]$中,取一个规范正交函数组$\{\phi_m\}_{m=0}^n$,即确定了一个用于逼近的线性子空间$\Phi_n=\text{span}\{\phi_0, \phi_1, \ldots, \phi_n\}$。

相应地,对于任意给定函数f,存在唯一的最佳平方逼近$\phi^*\in \Phi_n$,且
\begin{align}
    \phi^* = \sum_{i=0}^n a_i^* \phi_i, \quad a_i^* = (f, \phi_i)
\end{align}

\begin{definition}
    [广义傅里叶展开]
    设$\{\phi_m\}_{m=0}^\infty$为$C[a,b]$中的规范正交函数组,则对任意$f\in C[a,b]$,都有
    \begin{align}
        f \sim f_\infty= \sum_{i=0}^\infty a_i \phi_i, \quad a_i = (f, \phi_i)
    \end{align}
    $f_\infty$称为函数f在$\{\phi_m\}$下的广义傅里叶展开。
\end{definition}

\begin{theorem}
    [广义傅里叶展开的收敛性]
    设$\{\phi_m\}_{m=0}^\infty$为$C[a,b]$中的规范正交函数组,则对任意$f\in C[a,b]$,其广义傅里叶展开在$L_2$范数下收敛于f,即
    \begin{align}
        \lim_{n\rightarrow \infty} \|f - f_n\|_2 = 0
    \end{align}
    其中,$f_n = \sum_{i=0}^n a_i \phi_i$。
\end{theorem}


\paragraph{Legendre多项式作最佳平方逼近}
设内积为$[-1,1]$上的权1内积,给定$f\in C[-1,1]$,则f在Legendre多项式空间$P_n$中的最佳平方逼近为
\begin{align}
    I_nf(x)=\sum_{k=0}^n \frac{(f,\phi_k)}{\|\phi_k\|^2}\phi_k(x)
\end{align}

性质:
\begin{itemize}
    \item 收敛性:$I_nf\rightarrow f,\quad n\rightarrow \infty$,且收敛到$L^2((-1,1))$空间中
    \item 收敛速度估计:若$f\in C^2[-1,1]$,则$\forall \epsilon>0$,$\exists N>)$,使得
    \begin{align}
        \|f - I_nf\|_\infty \leq \frac{\epsilon}{\sqrt{n}},\quad n\geq N
    \end{align}
    即对于n次的Legendre多项式逼近,误差会被控制在$O(\frac{1}{\sqrt{n}})$范围内
\end{itemize}

\subsubsection{Legender多项式的零平方误差最小性质}
\begin{theorem}
    [Legendre多项式的零平方误差最小性质]
    设$f\in C[-1,1]$,则在所有满足$\deg(p_n)\leq n$且$(p_n, x^k)=0,\quad k=0,1,\ldots,n-1$的多项式中,Legendre多项式$P_n(x)$使得平方误差$\|f - P_n\|_2$最小。
\end{theorem}




\subsection {最小二乘逼近:最优平方逼近的离散化形式}
\tofill[问题]{最小二乘逼近的问题模型:确定的输入-输出关系叠加系统噪声和随机误差,希望拟合输入-输出关系的具体形式}

\subsubsection{基础理论}
核心假设:观测数据$f_i$和正确输 $\phi^*(x_i)$之间的误差$a_i\xi_i$是一个独立同分布的零均值随机变量。

假设导出:当拟合结果$\phi$取得正确的关系$\phi=\phi^*$时,(归一化)误差的方差最小。

数学化:$\phi^*$是以下优化问题的极小解:
\begin{align}
    \phi^*=\arg\min_{\phi\in \Phi} \sigma^2=\arg\min_{\phi\in \Phi} \frac{1}{m+1}\sum_{i=0}^m [f_i - \phi(x_i)]^2
\end{align}


\subsubsection{数学方法:“半”内积}

\begin{theorem}
    [Haar条件]
    若给定一组采样点$\{x_i\}_{i=0}^m$,且$\Phi$中的任意非0元素在这组采样点上的采样结果不全为0,则称这组采样点满足Haar条件。
\end{theorem}


满足Haar条件时,可以利用在采样点上的采样结果,定义一个“半”内积:
\begin{align}
    (f,g)_m = \sum_{i=0}^m \frac{f(x_i) g(x_i)}{a_i^2}
\end{align}
此时这个“半”内积在$\Phi$上是一个真正的内积。因此上述问题等价于求解法方程AU=b:
\begin{align}
    \begin{bmatrix}
        (\phi_0, \phi_0)_m & (\phi_1, \phi_0)_m & \cdots & (\phi_n, \phi_0)_m \\
        (\phi_0, \phi_1)_m & (\phi_1, \phi_1)_m & \cdots & (\phi_n, \phi_1)_m \\
        \vdots & \vdots & \ddots & \vdots \\
        (\phi_0, \phi_n)_m & (\phi_1, \phi_n)_m & \cdots & (\phi_n, \phi_n)_m
    \end{bmatrix}
    \begin{bmatrix}
        u_0 \\ u_1 \\ \vdots \\ u_n
    \end{bmatrix}
    =
    \begin{bmatrix}
        ( f, \phi_0 )_m \\ ( f, \phi_1 )_m \\ \vdots \\ ( f, \phi_n )_m
    \end{bmatrix}
\end{align}

Note: 我们注意到,在求内积过程中,分母上表示采样点随机误差幅度的$a_i^2$仍然是不确定的,此时需要基于经验
或先验知识给出估计。常用的估计有$a_i\propto 1$、$a_i\propto f_i$、$a_i\propto x_i$、$a_i\propto x_i^2$等。


\subsection{最佳一直逼近}
\tofill[问题]{最佳一直逼近的数学描述}

最佳一直逼近中用到一些符号:
\begin{itemize}
    \item $\Delta (f,p_n)=\|f-p_n\|_\infty$称为$p_n$关于f的偏差
    \item $E_n=\inf_{p_n\in P_n} \Delta (f,p_n)$称为f在$P_n$中的最小偏差
    \item 若$f(x_i)-p_n(x_i)=\sigma \Delta (f, p_n)$, $\sigma=\pm 1$,则称$x_i$为$p_n$关于f的偏差点。
    \begin{itemize}
        \item 若$\sigma=1$,则称$x_i$为正偏差点
        \item 若$\sigma=-1$,则称$x_i$为负偏差点
    \end{itemize}
\end{itemize}



\subsubsection{偏差泛函与最优逼近多项式存在性定理}
假设给定一个逼近多项式$p_n=a_0+a_1x+a_2x^2+\ldots +a_nx^n$,则可定义偏差泛函:
\begin{align}
    \phi(f,\boldsymbol{a})=\|f-p_n\|_\infty=\|f-(a_0+a_1x+\ldots +a_nx^n)\|_\infty
\end{align}

这个泛函具有以下性质:
\begin{itemize}
    % \item 正定性:$\phi(f,\boldsymbol{a})\geq 0$,且当且仅当$f=p_n$时取等号
    % \item 齐次性:$\forall \lambda \in \mathbb{R}$,都有
    % \begin{align}
    %     \phi(f, \lambda \boldsymbol{a}) = |\lambda| \phi(f, \boldsymbol{a})
    % \end{align}
    \item 连续性:$\phi$对$\boldsymbol{a}$连续
    \item 对$\boldsymbol{a}$下凸:$\forall \boldsymbol{a_1}, \boldsymbol{a_2}\in \mathbb{R}^{n+1}$,$\forall t\in [0,1]$,都有
    \begin{align}
        \phi(f, t\boldsymbol{a_1} + (1-t)\boldsymbol{a_2}) \leq t\phi(f,\boldsymbol{a_1}) + (1-t)\phi(f,\boldsymbol{a_2})
    \end{align}
    \item 正定性:对于任意$\boldsymbol{a}\neq \boldsymbol{0}$,都有$\phi(f,\boldsymbol{a})>0$
    
    正定性等价于:$\phi(0;\mathbf{a})$在单位球面$\sum_{i=0}^n a_i^2=1$上有正的最小值$\mu$
\end{itemize}

利用偏差泛函的性质,可以证明最佳一直逼近多项式的存在性:

\begin{theorem}
    [最佳一致逼近存在性定理]
    设$f\in C[a,b]$,则在$n$次多项式空间$P_n$中,存在一个多项式$p_n^*\in P_n$,使得
    \begin{align}
        \|f - p_n^*\|_\infty = \min_{p_n\in P_n} \|f - p_n\|_\infty
    \end{align}
    即$P_n$中关于$f\in C[a,b]$的最小偏差是可以达到的。
\end{theorem}

\subsubsection{最佳一致逼近多项式的性质}
\begin{definition}
    [Chebyshev交错点组]
    若$x_1<x_2<\ldots <x_m$是$p_n$关于f的轮流为正负的偏差点,则称其为一个Chebyshev交错点组。
\end{definition}

\begin{theorem}
    [最佳一致逼近多项式的Chebyshev交错定理]
    $p^*_n$是f的最佳一致逼近多项式的充要条件是存在一个元素个数为n+2的Chebyshev交错点组$x_0<x_1<\ldots <x_{n+1}$
\end{theorem}

\begin{corollary}
    最佳一致逼近多项式是一个Lagrange插值多项式,其插值节点在$(a,b)$内。
\end{corollary}

\subsubsection{唯一性定理}
\begin{theorem}
    [最佳一致逼近多项式的唯一性定理]
    设$f\in C[a,b]$,则$f$在$n$次多项式空间$P_n$中的最佳一致逼近多项式$p_n^*$是唯一的。
\end{theorem}


\subsubsection{Chebyshev多项式的零一致误差最小性质}
\begin{theorem}
    [Chebyshev多项式的零一致误差最小性质]
    设$T_n(x)$为n次Chebyshev多项式,则在$[-1,1]$上,任意单位首项系数的n次多项式$p_n(x)$均满足
    \begin{align}
        \|p_n\|_\infty \geq \|T_n\|_\infty = \frac{1}{2^{n-1}}
    \end{align}
    即Chebyshev多项式在$[-1,1]$上具有最小的无穷范数。
\end{theorem}





\chapter {数值微积分}
\section {数值积分}
\subsection {通用理论}
\subsubsection {积分问题模型}
\subsubsection {求积公式核心}
\subsubsection {代数精度}
\subsubsection {稳定性}
\subsection {具体求积方法}
\subsubsection {Newton-Cotes 公式}
\paragraph {梯形公式}
\paragraph {Simpson 公式}
\paragraph {3/8 - 规则}
\subsubsection {复合求积方法}
\paragraph {复合梯形公式}
\paragraph {复合 Simpson 公式}
\subsubsection {加速方法}
\paragraph {Romberg 方法}
\subsubsection {Gauss 型求积}
\paragraph {Gauss 点与权系数}
\paragraph {Gauss-Legendre 求积}
\paragraph {Gauss-Chebyshev 求积}
\subsubsection {特殊积分处理}
\paragraph {奇异积分}
\paragraph {振荡积分}
\subsubsection {高维积分}
\paragraph {蒙特卡洛方法}
\section {数值微分}
\subsection {基础方法}
\subsubsection {向前差商}
\subsubsection {向后差商}
\subsubsection {中心差商}
\subsection {高精度方法}
\subsubsection {插值型数值微分}
\subsubsection {Richardson 外推加速}


\chapter {非线性方程求根}


\section {通用理论}
\subsection {问题模型}
\subsection {迭代法基础}
\subsection {收敛性分析}
\subsubsection {收敛阶定义}
\subsubsection {整体收敛性(压缩映像原理)}
\subsubsection {局部收敛性判定}
\subsection {收敛效率}
\section {具体方法}
\subsection {单步法}
\subsubsection {牛顿法}
\paragraph {迭代公式}
\paragraph {收敛性分析}
\paragraph {重根处理}
\subsubsection {不动点迭代}
\subsection {多步法}
\subsubsection {割线法}
\subsection {其他方法}
\subsubsection {Steffensen 方法}
\subsubsection {Broyden 秩 1 方法}




\chapter {常微分方程初值问题数值解法}

\section {通用理论}

\subsection {问题模型}
\subsection {数值解法核心}
\subsection {基本概念}
\subsubsection {局部截断误差与整体截断误差}
\subsubsection {收敛性与收敛阶}
\subsubsection {稳定性与相容性}
\subsection {收敛性与稳定性判定}


\section {具体方法}

\subsection {单步法}
\subsubsection {Euler 方法}
\paragraph {向前欧拉}
\paragraph {向后欧拉}
\subsubsection {改进欧拉方法}
\subsubsection {龙格 - 库塔(RK)方法}
\paragraph {2 阶 RK 方法}
\paragraph {4 阶 RK 方法}

\subsection {多步法}
\subsubsection {Adams 方法}
\paragraph {显式 Adams 公式}
\paragraph {隐式 Adams 公式}
\subsubsection {Nyström 方法}
\subsubsection {Milne-Simpson 方法}





\chapter {线性代数方程组数值解法}
\section {直接解法}
\subsection {通用理论}
\subsubsection {问题模型}
\subsubsection {残差与误差分析}
\subsubsection {数值稳定性与条件数}
\subsection {具体方法}
\subsubsection {Gauss 消元法}
\paragraph {基本步骤}
\paragraph {主元素选取}
\subsubsection {LU 分解}
\subsubsection {特殊方程组解法}
\paragraph {Thomas 算法(三对角方程组)}
\paragraph {Toeplitz 矩阵快速解法}
\section {定常线性迭代解法}
\subsection {通用理论}
\subsubsection {迭代格式构造}
\subsubsection {收敛性判定}
\subsubsection {收敛速度}
\subsection {具体迭代法}
\subsubsection {Jacobi 迭代}
\subsubsection {Gauss-Seidel 迭代}
\subsubsection {SOR 迭代}
\paragraph {松弛因子选取}
\paragraph {最优松弛因子}
\section {非线性迭代解法(基于变分原理)}
\subsection {通用理论}
\subsubsection {Ritz 变分原理}
\subsubsection {迭代核心思想}
\subsection {具体方法}
\subsubsection {最速下降法}
\subsubsection {共轭梯度法(CG)}
\paragraph {迭代公式}
\paragraph {收敛性分析}
\subsubsection {预条件技术}
\paragraph {预条件矩阵选取}
\paragraph {预条件 CG 迭代格式}







\end{document}